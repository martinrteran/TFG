\chapter{Métodos}
En este capítulo se detallan los criterios seguidos para la elección de la geometría de los espaciadores y su escala, los cálculos realizados para la obtención de las propiedades de los materiales utilizados en las simulaciones de transmisión de calor por conducción según la escala utilizada en el modelado 3D del nano-espaciador, los procedimientos seguidos para realizar las simulaciones y el procedimiento seguido para la extracción de los datos de la simulación de CFD.
\section{Criterios de geometría y escala} 
La base del nano-espaciador es cuadrada por ser la más sencilla de fabricar por deposición catódica y para realizar cálculos por ser sus ecuaciones de área y volumen la más sencilla de los polígonos regulares cerrados. El lado de la base del nano-espaciador son de 3$\mu m$ por ser la unidad mínima que se es capaz de depositar en la célula para disminuir las pérdidas por conducción.\\\\
Para el modelado 3D en Inventor se toma como distancia mínima el milímetro, correspondiente a 100nm de la realidad que es la distancia mínima de todos componentes del sistema, para el caso de altura mínima del nano-espaciador a simular.
\section{Cálculos de las propiedades de los materiales para las simulaciones}
Para obtener los nuevos parámetros de distancias, áreas y volúmenes del modelo 3D del nano-espaciador se procede a obtener las relaciones de escala entre el modelo y la realidad. También se obtienen las relaciones de las propiedades más significativas de los materiales de la realidad y las simulaciones de transmisión de calor por conducción, siendo la principal propiedad la conductividad térmica. Otras propiedades que se toman en cuenta son la densidad y el calor específico, teniendo que destacar la resistencia de contacto por unidad de área entre el nano-espaciador y el emisor.\\\\
Para diferenciar la realidad del modelo 3D o del modelo a simular se utilizará el apostrofe después de la variable correspondiente, por ejemplo, para la longitud real se usa $L$ y para la longitud en el modelo 3D se usa $L'$.\\\\
La relación de longitudes entre el modelo 3D y la realidad es:
\begin{equation}
{L'}/{L}={1mm}/{100nm}=10^4
\label{eq:relacion_longitud}
\end{equation}
%%%  AREA
\subsection{Área}
La sección de los nano-espaciadores es un cuadrado cuya fórmula de área es $A=L^2$, donde $A$ es el área y $L$ el lado del cuadrado, siendo la relación de las áreas la siguiente:
\begin{equation}
	\dfrac{A'}{A}=\left(\dfrac{L'}{L}\right)^2=10^8
	\label{eq:relacion_areas}
\end{equation}
%%% VOLUMEN
\subsection{Volumen}
El volumen de un prisma de base cuadrada se expresa como $V=A\cdot L$, donde $L$ es la altura del prisma.
\begin{equation}
	\dfrac{V'}{V}=\dfrac{A'\cdot L'}{A\cdot L} = 
	\dfrac{A'}{A}\cdot \dfrac{L'}{L} \ \Longrightarrow \ \dfrac{V'}{V} =10^{12}
	\label{eq:relacion_volumen}
\end{equation}
El volumen de cada nano-espaciador en el modelo será $10^{12}$ veces el volumen original.
%%% DENSIDAD
\subsection{Densidad}
La masa de cada elemento es igual entre el modelo($M'$) y la realidad($M$), por lo tanto la densidad varía.
\begin{equation}
\dfrac{\rho '}{\rho}=\dfrac{M'/V'}{M/V}=\dfrac{M'}{M}\cdot \dfrac{V}{V'} \ \Longrightarrow \ 
\dfrac{\rho '}{\rho}=10^{-12}
\label{eq:relacion_densidad}
\end{equation}
La densidad de cada elemento en el modelo será $10^{-12}$ veces la densidad de la realidad.
%%% CONDUCTIVIDAD TERMICA
\subsection{Conductividad Térmica}
La resistencia térmica de los materiales del modelo de simulación se mantiene igual a la de la realidad, por lo tanto, la conductividad térmica de los materiales en el modelo son distintas a la realidad. Sabiendo que la fórmula de la resistencia térmica de conducción es $R=1/k \cdot L/A$, donde $k$ es la conductividad térmica, $L$ la longitud y $A$ es la sección, se puede obtener la relación de las conductividades térmicas del modelo de simulación respecto a la realidad.
\[ \dfrac{R}{R'}= \dfrac{1/k}{1/k'}\cdot \dfrac{L/A}{L'/A'}= \dfrac{k'}{k}\cdot \dfrac{L}{L'}\cdot \dfrac{A'}{A}=1\]
\begin{equation}
\dfrac{k'}{k}=\dfrac{L'}{L}\cdot \dfrac{A}{A'}=\dfrac{10^4}{10^8} \ \Longrightarrow \ \dfrac{k'}{k}=10^{-4}
\label{eq:relacion_conductividadTermica}
\end{equation}
%%% CALOR ESPECIFICO
\subsection{Calor Específico}
El calor específico de los materiales del modelo de simulación es el mismo que el de la realidad porque el calor específico se define como la cantidad de calor necesaria que hay que suministrar a una unidad de masa para elevar su temperatura en una unidad, como la masa del modelo de simulación es igual a la de la realidad, la cantidad de energía necesaria para elevar una unidad de temperatura va a ser igual al del modelo de simulación respecto a la realidad, por lo tanto, el calor específico se mantiene igual al de la realidad.
%%% RESISTENCIA DE CONTACTO
\subsection{Resistencia de contacto}
La resistencia de contacto ($R_c$) es difícil de modelar matemáticamente en una ecuación ya que depende de muchas variables, como la temperatura, presión, entre otros. La resistencia de térmica producida por la resistencia de contacto es $R_{th}=R_{c}/A$, donde A es la superficie de contacto, por lo tanto se ve afectado por la diferencia de escala entre el modelo de simulación y la realidad.
\[ \dfrac{R_{th}'}{R_{th}}=\dfrac{R_c'}{R_c}\cdot \dfrac{A}{A'}=1 \]
\begin{equation}
	\dfrac{R_{th}'}{R_{th}}=\dfrac{R_c'}{R_c}\cdot \dfrac{A}{A'}=1 \ \Longrightarrow \  \dfrac{R_c'}{R_c}=\dfrac{A'}{A}=10^8
	\label{eq:relacion_Rc}
\end{equation}
%%% YOVANOVICH
Para los casos que la resistencia de contacto es presentada como conductancia de contacto según un modelo matemático \cite{experimental_Rc_SS}, según la ecuación de Cooper reducida por Yovanovich (ecuación \ref{eq:modeloYovanovich}).
\begin{subequations}
\begin{equation}
\dfrac{P}{H_c}=\left[ \dfrac{P}{c_1\left(1.62\sigma/m\right)^{c_2}} \right]^{\frac{1}{2+0.071c_2}}
\label{eq:modeloYovanovich}
\end{equation}
\begin{equation}
\dfrac{h_c\sigma}{k_sm}=1.25\left(\dfrac{P}{H_c}\right)^{0.95}
\label{eq:correlacionCooperSimplificadaYovanovich}
\end{equation}
\label{eqs:ecuacionesRcYovanovich}
\end{subequations}
%% Valores de las constantes
Donde $c_1$ es 10.6 GPa, $c_2$ es -0.40, $\sigma$ es la combinación RMS de la rugosidad de ambas superficies de los materiales con $\sigma_i$ siendo la rugosidad de la superficie $i$ (ecuación \ref{eq:mRMS}), $m$ es la combinación RMS de la media absoluta de la pendiente de la rugosidad con $m_i$ siendo la media de la pendiente absoluta de la rugosidad de la superficie $i$ (ecuación \ref{eq:sigmaRMS}), $H_c$ es la micro-dureza Vickers del material más duro y P es la presión aplicada \cite{experimental_Rc_SS}.
\begin{subequations}
\begin{equation}
m=\sqrt{m_1^2+m_2^2}
\label{eq:mRMS}
\end{equation}
\begin{equation}
\sigma=\sqrt{\sigma_1^2+\sigma_2^2}
\label{eq:sigmaRMS}
\end{equation}
\label{eqs:RMS}
\end{subequations}
Las ecuaciones son para la conductancia de contacto ($h_c$) entre dos aceros inoxidables 304(SS304), entonces para tener una mejor idea como la resistencia de contacto afecta a la conducción de calor se supone que el nano-espaciador de $SiO_2$ es liso, por ende, su $m_i$ y su $\sigma_i$ son nulas. Por lo tanto, se calcula la relación de las ecuaciones para el caso de los dos aceros y para el caso de SS-$SiO_2$.\\\\
Como se puede observar en las ecuaciones \ref{eqs:ecuacionesRcYovanovich}, el valor de $\sigma$ y $m$ se presentan siempre relacionados como $\sigma / m$, y se cumple que $\sigma /m =\sigma ' / m'$ porque al suponer que el caso de dos aceros tienen idénticas $m_i$ y $\sigma_i$, la relación $\sigma / m$ queda como $\sigma_i / m_i$, que resulta ser la misma relación para el caso de SS-$SiO_2$.\\\\
\[ \frac{\sigma}{m}=\frac{\sqrt{\sigma_{SS}^2+\sigma_{SS}^2}}{\sqrt{m_{SS}^2+m_{SS}^2}}=\frac{\sigma_{SS}\sqrt{2}}{m_{SS}\sqrt{2}} =\frac{\sigma_{SS}}{m_{SS}}\]
\[ \frac{\sigma'}{m'}=\frac{\sqrt{\sigma_{SS}^2+0^2}}{\sqrt{m_{SS}^2+0^2}}=\frac{\sigma_{SS}}{m_{SS}} \]
Por lo tanto, considerando que $c_1$ y $c_2$ no varían con el cambio de material la relación de $h_c'$ respecto a $h_c$ se obtiene relacionando la ecuación \ref{eq:correlacionCooperSimplificadaYovanovich} del modelo a simular con la realidad.
\[Cte=1.25\frac{m}{\sigma}\cdot \left(\dfrac{P}{H_c}\right)^{0.95} \]
\[h_c=k_s\cdot Cte \qquad k_s=2\cdot \frac{k_1\cdot k_2}{k_1+k_2}\]
Donde $Cte$ es una constante que es igual para el caso de los aceros y el caso SS-$SiO_2$. El valor de $k_s$ para el caso de de los aceros es $k_{SS}$, siendo $k_{SS}$ la conductividad térmica del acero inoxidable 304. 
\[ \frac{h_c'}{h_c}=\frac{k_s'}{k_s}\cdot \frac{Cte}{Cte}=\frac{2 \cdot \frac{k_{SS}\cdot k_{SiO_2}}{k_{SS}+k_{SiO_2}}}{k_{SS}}\]
\begin{equation}
\frac{h_c'}{h_c}=2\cdot \frac{k_{SiO_2}}{k_{SS}+k_{SiO_2}}
\label{eq:relacion_conductividadesTermicas}
\end{equation}
Para este trabajo los datos utilizados de las conductividades térmicas de los materiales para la relación de las conductancias de contacto son a temperatura ambiente, el valor de $k_{SiO_2}$ es $1.5 \ W/\left( m^2 K\right)$ y el valor de $k_{SS}$ es de $15 \ W/\left( m^2 K\right)$, quedando la relación de conductancias de contacto como ${h_c'}/{h_c}=0.1818$

\section{Simulación de transmisión de calor por radiación de campo cercano}
%Aquí tendrías que meter las simulaciones de campo cercano y hablar del método (está descrito en un paper que te pasé

\section{Simulación de transmisión de calor por conducción}

\section{Procedimiento de extracción de resultados de CFD}