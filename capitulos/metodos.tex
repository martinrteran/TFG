\chapter{Métodos}
En este capítulo se detalla principalmente los criterios seguidos para el diseño del espaciador, su modelado en Inventor, las simulaciones en CFD y los cálculos en MATLAB, a su vez los métodos empleados para la rectificación del correcto funcionamiento de CFD para la extracción de datos.

En segundo lugar se describe el procedimiento realizado para la extracción de datos de las simulaciones realizadas en CFD.

Por último, se describe el procedimiento seguido para la obtención de los resultados.
\section{Criterios seguidos} 

\section{Simulación en CFD}


\section{Análisis dimensional}
Dado que el programa Inventor está limitado a partir a la escala milimétrica, por lo tanto, se tiene que aplicar un escalado. Se procede a realizar un análisis dimensional de las ecuaciones para obtener los nuevos parámetros para la nueva escala del modelo.\\

Siendo $L'$ la longitud en el modelo y $L$ la longitud de la realidad, se considera que ${L'}/{L}=10^4$.
\subsection{Área}
La sección de los nano-espaciadores es un polígono regular de cuatro lados cuya fórmula del área es $A=L^2$, donde $A$ es el área y $L$ el lado del polígono.\\

Para cualquier polígono regular a fórmula del área se puede expresar como $A=Cte\cdot L^2$, siendo $Cte$ una constante distinta para cada polígono y tomando en cuenta la relación de escala.

\begin{equation}
	\dfrac{A'}{A}=\left(\dfrac{L'}{L}\right)^2=10^8
\end{equation}
\subsection{Volumen}
El volumen de un prisma de base de polígono regular se expresa como $V=A\cdot L$, donde $L$ es la altura del prisma.
\begin{equation}
	V'=A'\cdot L' = A\cdot L \cdot 10^8\cdot 10^4= V\cdot 10^12
\end{equation}

El volumen de cada nano-espaciador en el modelo será $10^12$ veces el volumen original.
\subsection{Conductancia Térmica}

Al aplicarse la escala la conductancia térmica (\textit{C}) se mantiene constante, variando la conductividad térmica (\textit{k}) del material. La nueva conductividad térmica $k'$ será:

\begin{equation}
	C=k\cdot \frac{A}{L}=k'\cdot\frac{A'}{L'}
\end{equation}
\begin{equation}	
	 k'=\dfrac{L'}{L} \cdot \dfrac{A}{A'} \cdot k=10^4\cdot 10^{-8}=k\cdot 10^{-4}
\end{equation}

Donde $k$ es la conductividad térmica del material  y $k'$ es la conductividad térmica para la escala aplicada. $k'=k\cdot10^{-4}$

\subsection{Calor Específico}

\subsection{Coeficiente de expansión térmica}

\subsection{Densidad}
La masa de cada elemento tiene que ser constante entre el modelo y la realidad.
\begin{equation}
	M=M'\Longrightarrow \rho\cdot V=\rho'\cdot V'= \rho'\cdot V\cdot {10}^{12}\Longrightarrow \rho'=\rho\cdot{10}^{-12}
\end{equation}

La densidad de cada elemento en el modelo será $10^{-12}$ veces la densidad de la realidad.

\subsection{Resistencia de Contacto}
En \cite{noauthor_parallel-plate_nodate} se descubre como la resistencia de contacto afecta a los espaciadores y se obtiene el valor de su coeficiente($\rho$) $4\cdot 10^{-6}m^2 KW^{-1} $. Para obtener cual es el coeficiente de la resistencia de contacto del modelo se realiza un análisis dimensional.

\begin{equation}
	R= \rho \cdot A =\rho ' \cdot A' \Longrightarrow \rho '=\rho \cdot \dfrac{A}{A'}=\rho \cdot 10^8
\end{equation}

La resistencia térmica por contacto

Según el módelo de XXXXXXX:
\begin{equation}
	h_c=\dfrac{k_s\cdot}{m}\cdot \left( \dfrac{P}{H_c} \right)^{0.95} \\ \dfrac{P}{H_c} = \left[ \dfrac{P}{c1\cdot \left(1.62\cdot \sigma/m \right)^{c_2}}\right]^{\dfrac{1}{1+0.0071\cdot c_2}} \\	\sigma =\sqrt{{\sigma_1}^2+{\sigma_2}^2} \\ m=\sqrt{{m_1}^2+{m_2}^2}	\\ k_s=2\cdot \dfrac{k_1\cdot k_2}{k_1+k_2}
\end{equation}

Para el dato hc=1000;
\begin{equation}
	\dfrac{\sigma}{m}=\dfrac{\sigma_1}{m1}	\\ h_{c1}=k_{s1}\cdot cte \\ h_{c2}=k_{s2}\cdot cte	\\ k_{s1}=k_1 \\ \dfrac{h_{c2}}{h_{c1}}= 2\cdot \dfrac{k_2}{k_1+k_2}=cof
\end{equation}
k1=15 y k2=1.5
\begin{equation}
	cof=2\cdot \dfrac{1.5}{15+1.5}=0.18182
	\\ h_{c2}=h_{c1}\cdot cof=181.82 \rightarrow h_{c2}=181.82 \rightarrow R_{c2}=5.5E-3
\end{equation}