\chapter{Resultados y discusión}

En este capítulo se presentan los resultados obtenidos de las simulaciones de transmisión de calor por conducción y radiación, comparando varios casos de estudio de diferentes materiales de emisor y célula, con varias distancias de separación entre el emisor y la célula. También se estudia los efectos de la porosidad del $SiO_2$ y la resistencia de contacto entre nano-espaciador y emisor.\\\\
Los casos a estudiar son los siguientes:
\begin{enumerate}
	\item Comprobación del correcto funcionamiento del método de extracción de los datos de la transmisión de calor por conducción en CFD \label{res:comprobacionCFD}.
	\item El caso más sencillo a simular para tener una primera idea del funcionamiento del sistema es el de emisor de $Si$ y célula de $Si$, representado como $Si-Si$.\label{res:caso_Si_Si}
	\item $Si_Ge$: es el siguiente caso más sencillo, donde la célula es la correspondiente a utilizar.\label{res:caso_Si_Ge}
	\item $SS_Ge$: es el caso de recuperación de calor en la industria porque en la industria se utiliza mucho el acero inoxidable, siendo los aceros inoxidables 304 y 304L muy populares y utilizados.\label{res:caso_SS_Ge}
	\item $SiC_Ge$: 
\end{enumerate}
\section{Comprobación de procedimiento de extracción de datos de CFD}


\section{Resultados}
\begin{figure}
	\begin{tikzpicture}
		\begin{axis}[]
			\addplot[] {exp(-x/10)*( cos(deg(x)) + sin(deg(x))/10 )};
		\end{axis}
	\end{tikzpicture}
\end{figure}


\section{Discusión}