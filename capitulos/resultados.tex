\chapter{Resultados y discusión}
En este capítulo se presentan las relaciones entre la cantidad de potencia transmitida por radiación (que es una potencia aprovechable para generar electricidad) y la cantidad de potencia transmitida por conducción (la cual no se puede aprovechar) para diferentes combinaciones de materiales de emisor y célula.\\\\
Para determinar estas relaciones se simula la transmisión de calor por conducción a través de un nano-espaciador, para luego extrapolar la cantidad de calor por conducción que se obtendría para un dispositivo de 1 $cm^2$ con distintas distribuciones de nano-espaciadores ($n^{\underline{\circ}}\ espaciadores/cm^2$), y se simula la transmisión de calor por radiación de campo cercano en un dispositivo de 1 $cm^2$. Las simulaciones se realizan con el emisor a una temperatura constante de 800\textdegree C y la célula a 25\textdegree C.
\begin{itemize}
	\item Primero se valida la simulación de CFD.
	\item Se presentan los resultados de las simulaciones de transmisión de calor por conducción y por radiación de campo cercano para diferentes combinaciones de materiales de emisor y célula, y la relación entre ambas simulaciones.
	\item Se estudia el número mínimo de espaciadores necesarios para soportar la carga de los emisores.
	\item Por último, se presentan los resultados de usar un nano-espaciador de $Si$ en vez de $SiO_2$ para emisores de $Si$, $SS$ y $SiC$.
\end{itemize}
%%%%%%%%%%%%%%%%%%%%%%%%%%%%%%%%%%%%%%%%%%%%%%%%%%%%%%%%%%%%%%%%%%%%%%%
%% COMPROBACIÓN DEL PROCEDIMIENTO DE EXTRACCIÓN DE RESULTADOS DE CFD
%%%%%%%%%%%%%%%%%%%%%%%%%%%%%%%%%%%%%%%%%%%%%%%%%%%%%%%%%%%%%%%%%%%%%%%
\section{Validación de las simulaciones de CFD} \label{sec:val_CFD}
Las simulaciones de CFD son validas al comprobarse que con ellas se obtienen los mismos resultados que el modelo analítico (ecuación \eqref{eq:conduccion}) cuando se imponen las mismas simplificaciones de dicho modelo, es decir, cuando se impone una conductividad térmica constante y altura del nano-espaciador de 1000 nm.
\begin{equation}
Q=k\cdot \Delta T
\label{eq:conduccion}
\end{equation}
Para simplificar suponemos que la conductividad térmica ($k$) de los materiales es constante con la variación de la temperatura y usamos su valor para una temperatura de 25 \textdegree C, la $k$ del $Si$ es 182.98 W/m\textdegree C y la del $SiO_2$ es 1.30 W/m\textdegree C. De la simulación se extrae el flujo de calor del sistema y la temperatura media máxima y mínima de las superficies de contacto del nano-espaciador con los otros componentes del sistema.\\\\
La temperatura media máxima del nano-espaciador es 792.601\textdegree C, la temperatura media mínima del nano-espaciador es 32.39\textdegree C y el flujo de calor es 8897.93 $\mu W$. Con los resultados de las temperaturas medias del nano-espaciador se obtiene un flujo de calor teórico de 8899.05 $\mu W$ (obtenido mediante la ecuación \eqref{eq:conduccion}, siendo $k$ la conductancia térmica $k=\sigma \cdot A/L$), obteniéndose un error relativo aproximado del $12.6\cdot 10^{-3}$\%, por lo tanto el procedimiento es apropiado para la obtención de los resultados de las simulaciones de transmisión de calor por conducción.
%%%%%%%%%%%%%%%%%%%%%%%%%%%%%%%%%%%%%%%%%%%%%%%%%%%%%%%%%%%%%%%%%%%%%%%%%%
%%% SIMULACIONES DE TRANSMISIÓN DE CALOR POR CONDUCCIÓN EN CFD   Si-SiO2-Si
%%%%%%%%%%%%%%%%%%%%%%%%%%%%%%%%%%%%%%%%%%%%%%%%%%%%%%%%%%%%%%%%%%%%%%%%%%%
\section{Resultados de las simulaciones para una nTPV de Si-SiO2-Si}\label{sec:res_SiSiO2Si}
A continuación se estudian los efectos de la resistencia de contacto y la porosidad sobre un sistema sencillo compuesto por un emisor de $Si$ de 1 $mm$ lado y 0.2 $mm$ de altura con la cara superior a 800 \textdegree C, un nano-espaciador de 3 $\mu m$ de lado y una célula de $Si$ de las mismas dimensiones que el emisor con la cara inferior a 25\textdegree C. El emisor y la célula son de 1 $mm$ de lado porque para 10 $mm$ de lado el factor de escala hace que sea más difícil de realizar el proceso de modelado 3D y el de simulación de transmisión de calor por conducción.
%%%%%%%%%%%%%%%%%%%%%%%%%%%%%%%%%%%%%%%%
%               RC
\subsection{Efectos de la resistencia de contacto sobre la conducción}
La resistencia de contacto ($R_c$) usada es de unos $4\cdot 10^{-6} \ m^2 K/W$ \cite{nf_TPV_Pillars_SiO2} que es aplicada a la superficie superior del nano-espaciador que entra en contacto con la superficie inferior del emisor, solo se considera en dicha superficie porque el nano-espaciador será depositado sobre la superficie de la célula en la fase de fabricación, con lo cual la interfaz de con la célula se considera perfecta.
\begin{figure}[H]
	\centering
		\includegraphics[width=.55\textwidth]{figuras/Resultados/conduccion/pdf/PrcComb2PDF_SiSiO2Si}
	\caption[Efectos de la resistencia de contacto sobre el flujo de calor por conducción]{Representación gráfica del flujo de calor por conducción frente a las diferentes alturas del nano-espaciador con y sin $R_c$. 
	%Comparación de la potencia conducida de un nano-espaciador sin $R_c$ y con $R_c$ sobre un mismo eje. Y en el recuadro la potencia conducida de un nano-espaciador con $R_c$.
	}
	\label{fig:PcondRc_SiSiO2Si}
\end{figure}
Para un área de 9$\mu m^2$ la resistencia térmica obtenida de la $R_c$ es aproximadamente de $444.44\cdot 10^3 \ K/W$, comparada con los $85.43\cdot 10^3 \ K/W$ de la mayor resistencia que presenta el nano-espaciador con 1000 nm de altura a 25\textdegree C es al menos 5 veces más grande, por lo cual, como se observa en la figura \ref{fig:PcondRc_SiSiO2Si} la $R_c$ domina sobre la resistencia del nano-espaciador dando así la forma aproximada de una recta (recuadro de la figura \ref{fig:PcondRc_SiSiO2Si}) porque la mayor caída de temperatura se dá en la superficie de contacto, evitando que aumente la temperatura del nano-espaciador variando menos su resistencia térmica (figuras \ref{fig:Pcond_SiSiO2Si_CFD} y \ref{fig:Pcond_SiSiO2Si_Rc_CFD}).
\graphicspath{ {./figuras/Resultados/conduccion/} }
\begin{figure}[H]
	\centering 		% cond sin Rc
	\begin{subfigure}[b]{0.49\textwidth}
	\centering
		\includegraphics[width=1.00\textwidth]{SiSiO2Si_1000nm_Plane2.png}
		\caption{ }
	\label{fig:Pcond_SiSiO2Si_CFD}
\end{subfigure}
\hfill 					% cond con Rc
\begin{subfigure}[b]{0.49\textwidth}
	\centering
		\includegraphics[width=1.00\textwidth]{SiSiO2Si_1000nm_Plane_Rc.png}
		\caption{ }
	\label{fig:Pcond_SiSiO2Si_Rc_CFD}
\end{subfigure}
\caption{Resultados gráficos de la simulación de CFD de la transmisión de calor por conducción a través de un nano-espaciador de 1000nm de altura sin $R_c$ (\subref{fig:Pcond_SiSiO2Si_CFD}) y con $R_c$ (\subref{fig:Pcond_SiSiO2Si_Rc_CFD}).}
	\label{fig:Pconds_SiSiO2Si_CFD}
\end{figure}
La disminución de flujo de calor por conducción es significativa para todos los casos, siendo la potencia de conducción con $R_c$ variando aproximadamente entre un 3\% y un 14\% de la potencia sin $R_c$, lo que implica una disminución de conducción de unos 97\% y 85\%.\\\\
Hay que tomar en cuenta que la $R_c$ en la realidad no es constante con la temperatura a diferencia de las simulaciones en CFD donde la $R_c$ es constante, pero sirve para tener una primera idea de su importancia en la eliminación de la transferencia de calor por conducción.
%%%%%%%%%%%%%%%%%%%%%%%%%%%%%%%%%%%%%%%%%%%%%%%%%%%%%%%%
%             POROSIDAD
\subsection{Efecto de la porosidad sobre la conducción}
%% Ahora a comentar sobre las porosidades
Para diferentes porosidades la conductividad térmica varía, disminuyendo con el aumento del grado de porosidad \cite{ThermalConductivity_SiO2_2018}, por este motivo la potencia de conducción disminuye para todas las alturas de nano-espaciador. La relación o conductividad térmica normalizada para una porosidad del 25\% y 50\% son respectivamente 0.64 y 0.25 veces la conductividad térmica del material según la referencia \cite{ThermalConductivity_SiO2_2018}.\\\\
Como se puede observar en la figura \ref{fig:relPpor_SiSiO2Si} las relaciones de potencia no se cumplen completamente porque la temperatura en todo el nano-espaciador no es la misma lo que produce que la conductividad térmica a lo largo del espaciador sea distinta. Por tal motivo, al disminuir la altura del nano-espaciador aumenta la relación porque aumenta el gradiente de temperatura.
\begin{figure}[H]
\centering
	\begin{subfigure}[b]{0.49\textwidth}
		\centering
		\includegraphics[width=1.0\textwidth]{figuras/Resultados/conduccion/pdf/Ppor_SiSiO2Si.pdf}
		\caption{ }
		\label{fig:Ppor_SiSiO2Si}
	\end{subfigure}
	\hfill
	\begin{subfigure}[b]{0.49\textwidth}
		\centering
		\includegraphics[width=1.0\textwidth]{figuras/Resultados/conduccion/pdf/relPpor_SiSiO2Si.pdf}
		\caption{ }
		\label{fig:relPpor_SiSiO2Si}
	\end{subfigure}
	\caption[Efectos de la porosidad del nano-espaciador sobre el flujo de calor por conducción]{(\subref{fig:Ppor_SiSiO2Si}) Representación gráfica de las potencias de calor transmitidas por conducción  a través de un nano-espaciador frente a la variación de la altura del nano-espaciador para diferentes grados de porosidad del 0\%, 25\% y 50\% del $SiO_2$ y su modelo analítico (ec. \eqref{eq:Pcond_d_p}). (\subref{fig:relPpor_SiSiO2Si}) Representación gráfica de las relaciones de las potencias por conducción para porosidades del $SiO_2$ de 25\% y 50\% respecto a la potencia de 0\% de porosidad frente a la variación de la altura del nano-espaciador.}
	\label{fig:PcondPor_SiSiO2Si}
\end{figure}
Utilizando la aplicación \textbf{Curve Fitting} de MATLAB se obtiene un modelo matemático que relaciona la potencia de conducción respecto a la altura del nano-espaciador ($d$) y la porosidad del material del nano-espaciador ($\rho$), como se muestra en la ecuación \eqref{eq:Pcond_d_p} donde $d$ es en nanómetros y $\rho$ en base a uno.
\begin{equation}
P(d,\rho)=- \frac{  16.47\cdot \rho-11.03 }{d-106.80\cdot \rho +74.68}
\label{eq:Pcond_d_p}
\end{equation}\vspace{2cm}
%%%%%%%%%%%%%%%%%%%%%
%% RAD Si-SiO2-Si
\subsection{Radiación de campo cercano}
Para la radiación por campo cercano se utiliza la aplicación descrita en la sección \ref{sec:calc_campo_cercano} para dos placas gruesas de $Si$ para varias separaciones entre ellas. La potencia radiada espectral (figura \ref{fig:rad_SiSi_ds}) aumenta con la disminución de la distancia de separación como lo indica el componente exponencial en la ecuación \eqref{eq:flujoEvasNF} de \cite{nfTPV_equations}.
\begin{figure}[H]
	\centering
		\includegraphics[width=0.65\textwidth]{figuras/Resultados/radiacion/SiSi_ds.pdf}
	\caption{Potencia radiada espectral ($q_w$) para dos placas gruesas planas de $Si$ en todo el rango espectral frente a diferentes distancias de separación de las placas ($d$) en metros.\sourceSpectralRadiation}
	\label{fig:rad_SiSi_ds}
\end{figure}
Integrando la potencia en el rango espectral de longitudes de onda con energías mayores a 1.1 eV, BG del $Si$, se obtiene en promedio potencias del orden de $60 \ W/m^2$ (figura \ref{fig:prad_Eg11_SiSi}) a diferencia de integrar en todo el rango espectral, hasta las $\sim$20 $\mu m$, cuyo orden es de $10^4 \ W/m^2$ (figura \ref{fig:prad_full_SiSi}), Lo cual indica que se desaprovecha una gran cantidad de energía.
\begin{figure}[H]
	\centering
		\begin{subfigure}[b]{0.49\textwidth}
	\centering
		\includegraphics[width=1.00\textwidth]{figuras/Resultados/radiacion/p_11_SiSi.pdf}
	\caption{ }
	\label{fig:prad_Eg11_SiSi}
\end{subfigure}
\hfill
\begin{subfigure}[b]{0.49\textwidth}
	\centering
		\includegraphics[width=1.00\textwidth]{figuras/Resultados/radiacion/p_full_SiSi.pdf}
	\caption{ }
	\label{fig:prad_full_SiSi}
\end{subfigure}
	\caption{Potencias por unidad de área transmitida por radiación por efecto de campo cercano para el rango espectral de energía mayor a 1.1 eV (\subref{fig:prad_Eg11_SiSi}) y para todo el rango espectral (\subref{fig:prad_full_SiSi}) frente a las diferentes distancias de separación.}
	\label{fig:prad_SiSi}
\end{figure}
Las potencias radiadas frente a las alturas de los nano-espaciadores obtenidas para el rango de longitudes de onda mayor a la BG del $Si$ son muy pequeñas, como se observa en la figura \ref{fig:prad_Eg11_SiSi}, produciendo que no sea viable este sistema porque las pérdidas por conducción son demasiado grandes. Esto se puede visualizar en la figura \ref{fig:rel_SiSi11_Rc}, donde ni siquiera asumiendo la presencia de un único nano-espaciador con Rc se consigue que la potencia radiada sea al menos un orden de magnitud más alta que la potencia transferida por conducción (no se alcanza un factor 10).
\begin{figure}[H]
	\centering
\begin{subfigure}[b]{0.49\textwidth}
	\centering
		\includegraphics[width=1.00\textwidth]{figuras/Resultados/RelacionCondRad/rel_SiSi11.pdf}
	\caption{ }
	\label{fig:rel_SiSi11}
\end{subfigure}
\hfill
\begin{subfigure}[b]{0.49\textwidth}
	\centering
		\includegraphics[width=1.00\textwidth]{figuras/Resultados/RelacionCondRad/rel_SiSi11_Rc.pdf}
	\caption{ }
	\label{fig:rel_SiSi11_Rc}
\end{subfigure}
\caption[Relación entre la potencia radiada para energías mayores e igual a 1.1 eV y la potencia transmitida por conducción para distintas densidades de nano-espaciadores (eje y) y distintas alturas de los mismos (eje x), con (\subref{fig:rel_SiSi11_Rc}) y sin (\subref{fig:rel_SiSi11}) $R_c$ térmica para una célula de 1 $cm^2$ de $Si$ y emisor de $Si$. La barra de colores lateral representa los colores asociados a cada uno de los valores de las relaciones de potencias, con los contornos de las relaciones más significativas representadas en las gráficas.]{\small  Relación entre la potencia radiada para energías mayores e igual a 1.1 eV y la potencia transmitida por conducción para distintas densidades de nano-espaciadores (eje y) y distintas alturas de los mismos (eje x), con (\subref{fig:rel_SiSi11_Rc}) y sin (\subref{fig:rel_SiSi11}) $R_c$ térmica para una célula de 1 $cm^2$ de $Si$ y emisor de $Si$. La barra de colores lateral representa los colores asociados a cada uno de los valores de las relaciones de potencias, con los contornos de las relaciones más significativas representadas en las gráficas.
%Densidad de los nano-espaciadores para diferentes relaciones de las potencias de radiación en el rango espectral de energía mayor a 1.1 eV respecto a las de conducción para cada densidad de nano-espaciadores de un sistema nTPV de 1$cm^2$ y célula de $Si$ sin $R_c$ (\subref{fig:rel_SiSi11}) y con $R_c$ (\subref{fig:rel_SiSi11_Rc}) frente a las diferentes alturas de los nano-espaciadores. La barra de colores al lateral de cada gráfica representa todas las relaciones entre ambas potencias para diferentes densidades de nano-espaciadores en el rango que se muestra en los extremos de cada barra. También se muestran los contornos con su valor para destacar las distintas relaciones existentes de una manera sencilla.
}
	\label{fig:rels_SiSi11}
\end{figure}
%% RELACION ENTRE COND Y RAD
Para tener una primera mejor idea de los valores numéricos de los resultados obtenidos de las simulaciones de transmisión de calor por conducción y radiación de campo cercano se recopilan en la tabla \ref{tab:condTerSiSiO2Si}, estando en notación científica y con los decimales necesarios para una clara diferenciación de los resultados con el cambio de la distancia de separación entre emisor y célula.
\begin{table}[H]
	\centering
		\caption[Tabla de resultados de las simulaciones de conducción y radiación de campo cercano para diferentes alturas del nano-espaciador. Flujos de calor del nTPV $Si-SiO_2-Si$ para diferentes alturas del nano-espaciador, para los casos sin $R_c$ y con $R_c$ igual a $4 \cdot 10^{-6} \ m^2 K/W$, y sin $R_c$ pero con las proporciones de las porosidades de unos 25\% y un 50\%.]{Tabla de resultados de las simulaciones de conducción y radiación de campo cercano para diferentes alturas del nano-espaciador. Flujos de calor del nTPV $Si-SiO_2-Si$ para diferentes alturas del nano-espaciador, para los casos sin $R_c$ y con $R_c$ igual a $4 \cdot 10^{-6} \ m^2 K/W$ \cite{nf_TPV_Pillars_SiO2}, y sin $R_c$ pero con las proporciones de las porosidades de \cite{ThermalConductivity_SiO2_2018} para un 25\% y un 50\%.}	
		\begin{tabular}{|c||c|c|c|c||c|c|}
		\hline
			\multirow{2}{*}{ }& \multicolumn{6}{c|}{\textbf{\large Potencias según como se transmite el calor}}\\ \cline{2-7}
		  & \multicolumn{4}{c||}{Conducción (W/nº nano-espaciadores)}& \multicolumn{2}{c|}{Radiación $(W/m^2)$}\\ \hline
			Dist. (nm)&$P_{Normal}$&$P_{R_c-Empirico}$&$P_{Porosidad25}$&$P_{Porosidad50}$&$P_{Eg>1.1eV}$&$P_{full}$\\ \hline \hline
			100&6,31E-02&1,69E-03&4,67E-02&2,30E-02&156.89&1,65E+05\\ \hline
			200&3,99E-02&1,66E-03&2,78E-02&1,26E-02&93.75&1,10E+05\\ \hline
			300&2,93E-02&1,63E-03&1,99E-02&8,69E-03&71.75&8,51E+04\\ \hline
			400&2,32E-02&1,60E-03&1,55E-02&6,63E-03&66.39&7,01E+04\\ \hline
			500&1,92E-02&1,58E-03&1,27E-02&5,36E-03&73.00&6,02E+04\\ \hline
			600&1,64E-02&1,55E-03&1,08E-02&4,50E-03&77.52&5,32E+04\\ \hline
			700&1,43E-02&1,53E-03&9,33E-03&3,88E-03&71.63&4,80E+04\\ \hline
			800&1,27E-02&1,50E-03&8,24E-03&3,41E-03&64.90&4,43E+04\\ \hline
			900&1,14E-02&1,48E-03&7,38E-03&3,04E-03&61.42&4,14E+04\\ \hline
		 1000&1,04E-02&1,46E-03&6,68E-03&2,74E-03&62.09&3,93E+04\\ \hline
		\end{tabular}
	\label{tab:condTerSiSiO2Si}
\end{table}
Cabe destacar que la potencia de conducción depende del número de nano-espaciadores y la potencia de radiación depende del área. Por lo tanto, para compararlas hay que multiplicar el número de nano-espaciadores por la potenia de conducción y el área en metros cuadrados a la potencia de radiación integrada.
\vfill \newpage
%%% AHORA EL CASO DE Si SiO2 y Ge
\section{Resultados de las simulaciones para una nTPV de Si-SiO2-Ge}\label{sec:res_SiSiO2Ge}
A continuación se procede a estudiar un caso más realista del sistema nTPV descrito en la sección anterior con una célula de $Ge$ en vez de una de $Si$, cuya BG es de 0.7 eV respecto a los 1.1 eV del $Si$.
\subsection{Simulaciones de CFD}
Se realizan las simulaciones de transmisión de calor por conducción en CFD, obteniéndose una reducción considerable de la potencia de la nTPV con $R_c$ (figura \ref{fig:Prc_SiSiO2Ge}) respecto a sin $R_c$ (figura \ref{fig:Pn_SiSiO2Ge}), como en el caso de la célula de $Si$ (figura \ref{fig:PcondRc_SiSiO2Si}).
\graphicspath{ {./figuras/Resultados/conduccion/pdf/} }
\begin{figure}[H]
	\centering
	%% Si-SiO2-Si Eg
	\begin{subfigure}[b]{0.49\textwidth}
		\centering
		\includegraphics[width=1\textwidth]{Pn_SiSiO2Ge.pdf}
		\caption{ }
		\label{fig:Pn_SiSiO2Ge}
	\end{subfigure}
	\hfill
	\begin{subfigure}[b]{0.49\textwidth}
		\centering
		\includegraphics[width=1.00\textwidth]{Prc2_SiSiO2Ge.pdf}
		\caption{ }
		\label{fig:Prc_SiSiO2Ge}
	\end{subfigure}
	\caption{Potencias transmitidas por conducción para un sistema nTPV de $Si-SiO_2-Ge$ de un nano-espaciador con emisor a 800\textdegree C y célula a 25\textdegree C sin $R_c$ (\subref{fig:Pn_SiSiO2Ge}) y con $R_c$ (\subref{fig:Prc_SiSiO2Ge}) frente a las diferentes alturas de nano-espaciador.}
	\label{fig:Pcond_SiSiO2Ge}
\end{figure}
La potencia de conducción disminuye de unos 53.80 $mW$ sin $R_c$ a unos 1.68 $mW$ con $R_c$ para una altura de nano-espaciador de 100 nm, que representa una reducción del 96.87\%. Para unos 1000 nm de altura de nano-espaciador la potencia disminuye de unos 10.20 $mW$ sin $R_c$ a unos 1.46 $mW$ con $R_c$, que representa una reducción del 85.69 \%.\\\\
Comparando los resultados obtenidos de la simulación de transmisión de calor por conducción de la nTPV de célula de $Ge$ respecto a la de $Si$, se observa una variación casi nula para cuando hay $R_c$. Siendo aproximadamente las potencias de conducción de la nTPV con célula de $Ge$ un 99\% de las potencias de la nTPV con célula de $Si$ para todas las alturas del nano-espaciador, dando a entender que la diferencia de la conductividad térmica de los materiales no produce un efecto significativo sobre la conducción cuando existe $R_c$. Para el caso de que la nTPV no tenga $R_c$, las potencias para la célula de $Ge$ son entre un $\sim$85\% y un $\sim$98\% respecto a la de la célula de $Si$.
\vfill
\subsection{Radiación de campo cercano}
De las simulaciones de radiación de campo cercano se obtienen también resultados muy parecidos a los obtenidos en el caso de la célula de $Si$ (figura \ref{fig:SiSi_vs_SiGe}), solo mostrándose hasta los $\sim$14$\mu m$ de longitud de onda, donde se observa como al disminuir la distancia de separación aumenta la potencia radiada espectral (figura \ref{fig:rad_SiGe}).
\begin{figure}[H]
\centering
\begin{subfigure}[b]{0.49\textwidth}
	\centering
		\includegraphics[width=1.00\textwidth]{figuras/Resultados/radiacion/SiGe.pdf}
	\caption{ }
	\label{fig:rad_SiGe}
\end{subfigure}
\hfill
\begin{subfigure}[b]{0.49\textwidth}
	\centering
		\includegraphics[width=1.00\textwidth]{figuras/Resultados/radiacion/SiSi_vs_SiGe.pdf}
	\caption{ }
	\label{fig:SiSi_vs_SiGe}
\end{subfigure}
\caption{Potencia radiada espectral ($q_w$) por campo cercano para un sistema nTPV $Si-SiO_2-Ge$ frente a las longitudes de onda para varias distancias de separación entre placas(\subref{fig:rad_SiGe}) y para dos materiales de célula, $Si$ y $Ge$ (\subref{fig:SiSi_vs_SiGe}).}
\label{fig:rads_SiGe}
\end{figure}
Para la obtención de las potencias de radiación se procede a realizar la integral en el rango espectral de energía mayor a los 0.7 eV, obteniéndose potencias alrededor de los $10^3 \ W/m^2$ (figura \ref{fig:p_Eg_SiGe}), y para todo el rango espectral de longitudes de onda se obtiene potencias alrededor de los $10^4 \ W/m^2$ con un máximo de $\sim$1.5$\cdot10^5 \ W/m^2$ (figura \ref{fig:p_full_SiGe} y tabla \ref{tab:SiSiO2Ge}). 
%% POTENCIAS INTEGRADAS
\begin{figure}[H]
\centering
\begin{subfigure}[b]{0.49\textwidth}
	\centering
		\includegraphics[width=1.00\textwidth]{figuras/Resultados/radiacion/p_Eg_SiGe.pdf}
	\caption{P }
	\label{fig:p_Eg_SiGe}
\end{subfigure}
\hfill
\begin{subfigure}[b]{0.49\textwidth}
	\centering
		\includegraphics[width=1.00\textwidth]{figuras/Resultados/radiacion/p_full_SiGe.pdf}
	\caption{ }
	\label{fig:p_full_SiGe}
\end{subfigure}
	\caption[Potencias por unidad de área para la radiación de campo cercano para el sistema $Si-SiO_2-Ge$ frente a las diferentes alturas del nano-espaciador]{Potencias por unidad de área para la radiación de campo cercano para el sistema $Si-SiO_2-Ge$ frente a las diferentes alturas del nano-espaciador. (\subref{fig:p_Eg_SiGe}) Potencias en el rango espectral de todas las longitudes de onda de energía mayor a 0.7 eV. (\subref{fig:p_full_SiGe}) Potencias en todo el rango espectral, hasta las $\sim$14$\mu m$.}
	\label{fig:p_SiGe}
\end{figure}
Como para el caso de la nTPV de $Si-SiO_2-Si$ la diferencia de la potencia entre integrar hasta los 0.7 eV del rango espectral e integrar en todo el rango espectral es significativa. Al cambiar la célula de $Si$ por una de menor BG de $Ge$ se aumenta la cantidad de potencia que se puede convertir en electricidad. También se observa una disminución de las potencias alrededor de los 600 nm (figura \ref{fig:p_Eg_SiGe}), a diferencia de la célula de $Si$ que aumenta debido al efecto Fabry-Perot (\ref{fig:prad_Eg11_SiSi}).
%%%   DENSIDAD DE NANO-ESPACIADORES
\subsection{Densidad de nano-espaciadores}
Para tener una mejor percepción de las ventajas que presenta el usar una célula de $Ge$ frente a una de $Si$ se calcula la densidad de nano-espaciadores frente a las alturas del nano-espaciador para las relaciones de la potencia de radiación respecto a la de conducción, observando que para las mismas alturas de nano-espaciador y densidades similares, se obtienen valores mucho mayores.\\\\
Hay que tener en cuenta que para un centímetro cuadrado de superficie de radiación la transmisión de calor por radiación se ve multiplicado por $10^{-4}$, por lo tanto, la potencia de radiación con $E_g>0.7eV$ en un centímetro cuadrado es $\sim$10 veces mayor que la potencia conducida sin resistencia de contacto para un nano-espaciador (figura \ref{fig:rel_SiSiO2Ge}), pero $\sim$100 veces mayor que la potencia conducida con resistencia de contacto (figura \ref{fig:rel_SiSiO2Ge_Rc}).
\begin{figure}[H]
	\centering
	%% Si-SiO2-Si Eg
	\begin{subfigure}[b]{0.49\textwidth}
		\centering
		\includegraphics[width=1.00\textwidth]{figuras/Resultados/RelacionCondRad/SiGe.png}
		\caption{ }
		\label{fig:rel_SiSiO2Ge}
	\end{subfigure}
	\hfill
	\begin{subfigure}[b]{0.49\textwidth}
			\centering
			\includegraphics[width=1.00\textwidth]{figuras/Resultados/RelacionCondRad/SiGe_Rc.png}
			\caption{ }
			\label{fig:rel_SiSiO2Ge_Rc}
		\end{subfigure}
	\caption[Relación entre la potencia radiada para energías mayores e igual a 0.7 eV y la potencia transmitida por conducción para distintas densidades de nano-espaciadores (eje y) y distintas alturas de los mismos (eje x), con (\subref{fig:rel_SiSiO2Ge_Rc}) y sin (\subref{fig:rel_SiSiO2Ge}) $R_c$ térmica para una célula de 1 $cm^2$ de $Ge$ y emisor de $Si$. La barra de colores lateral representa los colores asociados a cada uno de los valores de las relaciones de potencias, con los contornos de las relaciones más significativas representadas en las gráficas.]{\small Relación entre la potencia radiada para energías mayores e igual a 0.7 eV y la potencia transmitida por conducción para distintas densidades de nano-espaciadores (eje y) y distintas alturas de los mismos (eje x), con (\subref{fig:rel_SiSiO2Ge_Rc}) y sin (\subref{fig:rel_SiSiO2Ge}) $R_c$ térmica para una célula de 1 $cm^2$ de $Ge$ y emisor de $Si$. La barra de colores lateral representa los colores asociados a cada uno de los valores de las relaciones de potencias, con los contornos de las relaciones más significativas representadas en las gráficas.
	%Densidad de los nano-espaciadores para diferentes relaciones de las potencias de radiación en el rango espectral de energía mayor a 0.7 eV respecto a las de conducción para cada densidad de nano-espaciadores de un sistema nTPV de 1$cm^2$ y célula de $Ge$ sin $R_c$ (\subref{fig:rel_SiSiO2Ge}) y con $R_c$ (\subref{fig:rel_SiSiO2Ge_Rc}) frente a las diferentes alturas de los nano-espaciadores. La barra de colores al lateral de cada gráfica representa todas las relaciones entre ambas potencias para diferentes densidades de nano-espaciadores en el rango que se muestra en los extremos de cada barra. También se muestran los contornos con su valor para destacar las distintas relaciones existentes de una manera sencilla.
	}
	\label{fig:relation_SiSiO2Ge}
\end{figure}
Como se observa en las figuras \ref{fig:relation_SiSiO2Ge} \subref{fig:rel_SiSiO2Ge} y \subref{fig:rel_SiSiO2Ge_Rc}, la forma de la curva cambia su tendencia según se incluya o no la $R_c$ porque sin $R_c$ la tendencia de la curva de crecer con el aumento de la altura de los nano-espaciadores está marcada por la potencia de conducción (inversamente proporcional), y para el caso con $R_c$ la  tendencia de la curva de aumentar con la disminución de la altura de los nano-espaciadores está marcada por la potencia de radiación de campo cercano.\\\\
Los valores de las relaciones de la potencia de útil respecto a la pérdidas, es decir, la potencia de radiación divido por la potencia de conducción, son bajos para todas las densidades de nano-espaciadores del caso de la nTPV sin $R_c$, apenás consiguiendo una relación de al menos un orden de magnitud solo para una densidad de un 1 nº$esp/cm^2$ con altura mínima de 600 nm, como se puede observar en la figura \ref{fig:rel_SiSiO2Ge}. Para el caso con $R_c$ se consigue como mínimo una relación de dos ordenes de magnitud para todas las alturas de un nano-espaciador, como se puede observar en la figura \ref{fig:rel_SiSiO2Ge_Rc}.\\\\
Estas relaciones disminuyen con el aumento de la densidad de los nano-espaciadores porque aumenta la potencia de conducción, por este motivo solo serán viables aquellas configuraciones que como mínimo tengan una relación de potencias de un orden de magnitud y una configuración mínima de cuatro nano-espaciadores, brindando apoyo y una mayor estabilidad que para un mínimo de tres nano-espaciadores.\\\\
La densidad de nano-espaciadores para una relación de potencias de un orden de magnitud es como mínimo 7 veces mayor para una nTPV con $R_c$ (\ref{fig:rel_SiSiO2Ge_Rc}) respecto a sin $R_c$ (\ref{fig:rel_SiSiO2Ge}), que para dicha relación de potencias su densidad de nano-espaciadores es menor a 4 nº$esp/cm^2$. Para la nTPV con $R_c$, la densidad de nano-espaciadores es mayor que 4 para un relación de potencias de 10.\\\\
Por lo tanto, entre ambos casos el que puede ser viable es el sistema nTPV con $R_c$ porque presenta una mayor densidad de nano-espaciadores para una misma relación de potencias, lo que permite que se distribuya la carga sobre los nano-espaciadores y sea más fácil mantener la separación entre el emisor y la célula constante sobre toda la superficie.
\begin{figure}[H]
		\begin{subfigure}[b]{0.49\textwidth}
		\centering
		\includegraphics[width=1.00\textwidth]{figuras/Resultados/RelacionCondRad/SiGe_full.png}
		\caption{ }
		\label{fig:rel_SiSiO2Ge_full}
	\end{subfigure}
		\hfill
		\begin{subfigure}[b]{0.49\textwidth}
			\centering
			\includegraphics[width=1.00\textwidth]{figuras/Resultados/RelacionCondRad/SiGe_Rc_full_10.png}
			\caption{ }
			\label{fig:rel_SiSiO2Ge_Rc_full}
		\end{subfigure}
	\caption[Relación entre la potencia radiada total y la potencia transmitida por conducción para distintas densidades de nano-espaciadores (eje y) y distintas alturas de los mismos (eje x), con (\subref{fig:rel_SiSiO2Ge_Rc_full}) y sin (\subref{fig:rel_SiSiO2Ge_full}) $R_c$ térmica para una célula de 1 $cm^2$ de $Ge$ y emisor de $Si$. La barra de colores lateral representa los colores asociados a cada uno de los valores de las relaciones de potencias, con los contornos de las relaciones más significativas representadas en las gráficas.]{\small  Relación entre la potencia radiada total y la potencia transmitida por conducción para distintas densidades de nano-espaciadores (eje y) y distintas alturas de los mismos (eje x), con (\subref{fig:rel_SiSiO2Ge_Rc_full}) y sin (\subref{fig:rel_SiSiO2Ge_full}) $R_c$ térmica para una célula de 1 $cm^2$ de $Ge$ y emisor de $Si$. La barra de colores lateral representa los colores asociados a cada uno de los valores de las relaciones de potencias, con los contornos de las relaciones más significativas representadas en las gráficas.
	%Densidad de los nano-espaciadores para diferentes relaciones de las potencias de radiación en todo el rango espectral respecto a las de conducción para cada densidad de nano-espaciadores de un sistema nTPV de 1$cm^2$ y célula de $Ge$ sin $R_c$ (\subref{fig:rel_SiSiO2Ge_full}) y con $R_c$ (\subref{fig:rel_SiSiO2Ge_Rc_full}) frente a las diferentes alturas de los nano-espaciadores. La barra de colores al lateral de cada gráfica representa todas las relaciones entre ambas potencias para diferentes densidades de nano-espaciadores en el rango que se muestra en los extremos de cada barra. También se muestran los contornos con su valor para destacar las distintas relaciones existentes de una manera sencilla.
	}%
	\label{fig:rels_SiSiO2Ge_full}%
\end{figure}
Por último, se representan el caso de la densidad de nano-espaciadores frente a las alturas de los nano-espaciadores para varias relaciones de las potencias de radiación en todo el rango espectral y las potencias de conducción sin y con $R_c$ (figuras \ref{fig:rels_SiSiO2Ge_full}  \subref{fig:rel_SiSiO2Ge_full} y \subref{fig:rel_SiSiO2Ge_Rc_full}, respectivamente), observándose las mismas tendencias que para el rango espectral de energías mayores e iguales a 0.7 eV pero con mayor densidad de nano-espaciadores por el aumento de la potencia radiada de campo cercano, por lo tanto, el aumento del rango espectral de radiación útil es bastante importante para el aumento de la potencia a convertir en electricidad.\\\\
Para facilitar la revisión de los resultados obtenidos de las simulaciones y de los cálculos realizados se recopilan en la tabla \ref{tab:SiSiO2Ge}, donde se presentan en notación científica para facilitar la observación de los ordenes de magnitud de los resultados.
\begin{table}[H]
	\centering
	\caption{Tabla de las potencias de resultado de las simulaciones de transmisión de calor por radiación de campo cercano y conducción para el sistema nTPV $Si-SiO_2-Ge$.}	
		\begin{tabular}{|c||c|c||c|c|}
		\hline
\multirow{2}{*}{ }& \multicolumn{4}{c|}{\textbf{\large Potencias según transmisión del calor}}\\ \cline{2-5}
& \multicolumn{2}{c||}{Conducción (W/nº esp.)}& \multicolumn{2}{c|}{Radiación $(W/m^2)$}\\ \hline
Dist. (nm)&$P_{Normal}$&$P_{R_c-Empirico}$&$P_{Eg>0.7eV}$&$P_{full}$\\ \hline \hline
100&5,38E-02&1,68E-03&4,87E+03&1,54E+05\\ \hline 
200&3,65E-02&1,65E-03&2,94E+03&1,01E+05\\ \hline 
300&2,76E-02&1,62E-03&2,16E+03&7,71E+04\\ \hline 
400&2,22E-02&1,60E-03&1,77E+03&6,31E+04\\ \hline 
500&1,85E-02&1,57E-03&1,59E+03&5,38E+04\\ \hline 
600&1,59E-02&1,55E-03&1,55E+03&4,74E+04\\ \hline 
700&1,39E-02&1,52E-03&1,62E+03&4,27E+04\\ \hline 
800&1,24E-02&1,50E-03&1,73E+03&3,93E+04\\ \hline 
900&1,12E-02&1,48E-03&1,78E+03&3,67E+04\\ \hline 
1000&1,02E-02&1,46E-03&1,73E+03&3,49E+04\\ \hline 
		\end{tabular}
	\label{tab:SiSiO2Ge}
\end{table}
\vfill
%%% RESULTADOS DE SS
\newpage
%%%%%%%%%%%%%%%%%%%%%%%%%%%%%%%%%%%%%%%%%%%%%%%%%%%%%%%%%%%%%%%%%%%%%%%%%
%               EMISOR DE  STAINLESS STEEL
%%%%%%%%%%%%%%%%%%%%%%%%%%%%%%%%%%%%%%%%%%%%%%%%%%%%%%%%%%%%%%%%%%%%%%%%%
\section{Resultados de las simulaciones para una nTPV de SS-SiO2-Ge}\label{sec:res_SsSiO2Ge}
Una de las razones más importantes del estudio de las nTPV es la recuperación de calor residual, por lo cual se simula la transmisión por conducción y radiación para una nTPV de emisor de acero inoxidable ($SS$) porque es un material altamente utilizado en todos los sectores, principalmente en la industria, ya sea en calderas, tuberías u otros componentes o máquinas que se encuentran a altas temperaturas.
%%%%%%%%%%%%%%%%%%%%%%%%%%%%%%%%%%%%
%           CFD
\subsection{Simulaciones de CFD}
Para las simulaciones de transmisión de calor por conducción se estudian los efectos de resistencias de contacto aún mayores a la empírica estudiada hasta ahora. Las nuevas resistencias de contacto se obtienen aplicando la inversa a la \gls{hc} ($R_c=1/h_c$), la cual se obtiene mediante la relación de las $h_c$ según la ecuación \eqref{eq:relacion_conductividadesTermicas} para un emisor de $SS$ y un nano-espaciador liso de $SiO_2$. La $h_c$ utilizada es 1000 $W/(m^2 K)$, que corresponde a la $h_c$ para dos aceros inoxidables a una presión de $\sim$1 GPa para los experimentos realizados en \cite{experimental_Rc_SS}, porque a dicha presión el error respecto al modelo matemático es pequeño.\\\\
A partir de dicha $h_c$ y aplicando la ecuación \eqref{eq:relacion_conductividadesTermicas} se obtiene una $R_c$ de $5.5\cdot 10^{-3} \ m^2 K/W$ para una superficie de SS en contacto con una superficie lisa de $SiO_2$. Se toma un valor intermedio entre dicha $R_c$ y la $R_c$ empírica de $4\cdot 10^{-6} \ m^2 K/W$ \cite{nf_TPV_Pillars_SiO2} para obtener el efecto de la variación de la $R_c$ sobre la transmisión de conducción y la viabilidad del sistema, siendo el valor de dicha $R_c$ intermedia unos $2.75\cdot 10^{-3} \ m^2 K/W$.\\\\
Primero se simula la transmisión de calor por conducción para el caso sin $R_c$ (figura \ref{fig:Pn_SsSiO2Ge}) y con $R_c$ empírica de $4\cdot 10^{-6} \ m^2 K/W$ (figura \ref{fig:Prc_SsSiO2Ge_Emp}), para luego estudiar con $R_c$ calculada máxima de $5.5\cdot 10^{-3} \ m^2 K/W$ (figura \ref{fig:Prc_SsSiO2Ge_Inter}) y con $R_c$ calculada intermedia de $2.75\cdot 10^{-3} \ m^2 K/W$ (figura \ref{fig:Prc_SsSiO2Ge_Max}).
\graphicspath{ {./figuras/Resultados/conduccion/pdf/} }
\begin{figure}[H]
	\centering
	\begin{subfigure}[b]{0.49\textwidth}
		\centering
			\includegraphics[width=1.00\textwidth]{Pn_SsSiO2Ge.pdf}
		\caption{ }
		\label{fig:Pn_SsSiO2Ge}
	\end{subfigure}
	\hfill
	\begin{subfigure}[b]{0.49\textwidth}
		\centering
			\includegraphics[width=1.00\textwidth]{Prc_SsSiO2Ge_Emp.pdf}
		\caption{ }
		\label{fig:Prc_SsSiO2Ge_Emp}
	\end{subfigure}
	\caption{Potencias transmitidas por conducción para una nTPV $SS-SiO_2-Ge$ sin $R_c$ (\subref{fig:Pn_SsSiO2Ge}) y con $R_c$ empírica de $4\cdot 10^{-6} \ m^2 K/W$ (\subref{fig:Prc_SsSiO2Ge_Emp}) frente a las diferentes alturas de nano-espaciador.}
	\label{fig:Pcond1_SsSiO2Ge}
\end{figure}
La potencia máxima sin $R_c$ es menor que 0.05W (figura \ref{fig:Pn_SsSiO2Ge}) a diferencia de la nTPV de $Si-SiO_2-Ge$ que la potencia máxima es menor a unos 0.055W (figura \ref{fig:Pn_SiSiO2Ge}). Para la nTPV con $R_c$ empírica la potencia conducida se encuentra en el rango de 1.45 $mW$ a 1.7 $mW$ (figura \ref{fig:Prc_SsSiO2Ge_Emp}), igual que para el caso de la nTPV de $Si-SiO_2-Ge$, por lo tanto, el material del emisor no es significativo cuando existe una resistencia de contacto de por lo menos $4\cdot 10^{-6} \ m^2 K/W$.\\\\
Las potencias transmitidas por conducción para los casos con $R_c$ calculadas (figuras \ref{fig:Prc_SsSiO2Ge_Max} \subref{fig:Prc_SsSiO2Ge_Inter} y \subref{fig:Prc_SsSiO2Ge_Max}) son tres ordenes de magnitud menor que para las potencias del caso con $R_c$ empírica (figura \ref{fig:Prc_SsSiO2Ge_Emp}), esta reducción corresponde a que ambas $R_c$ calculadas son tres ordenes de magnitud superior a la $R_c$ empírica. Por ende, la $R_c$ es la resistencia de mayor relevancia del sistema nTPV para cualquiera de estos tres valores.
\begin{figure}[H]
	\centering
	\begin{subfigure}[b]{0.49\textwidth}
		\centering
			\includegraphics[width=1.00\textwidth]{Prc_SsSiO2Ge_Inter.pdf}
		\caption{ }
		\label{fig:Prc_SsSiO2Ge_Inter}
	\end{subfigure}
	\hfill
	\begin{subfigure}[b]{0.49\textwidth}
		\centering
			\includegraphics[width=1.00\textwidth]{Prc_SsSiO2Ge_Max.pdf}
		\caption{ }
		\label{fig:Prc_SsSiO2Ge_Max}
	\end{subfigure}
	\caption[Potencias trasmitidas por conducción para una nTPV $SS-SiO_2-Ge$ con $R_c$ calculadas, con una $R_c$ máxima de  $5.5\cdot 10^{-3} \ m^2 K/W$ (\subref{fig:Prc_SsSiO2Ge_Max}) y una $R_c$ intermedia $2.75\cdot 10^{-3} \ m^2 K/W$ (\subref{fig:Prc_SsSiO2Ge_Inter}) frente a las diferentes alturas de un nano-espaciadores.]{Potencias transmitidas por conducción para una nTPV $SS-SiO_2-Ge$ con $R_c$ calculadas a partir de las conductancias térmicas \cite{experimental_Rc_SS}, para una $R_c$ máxima de  $5.5\cdot 10^{-3} \ m^2 K/W$ (\subref{fig:Prc_SsSiO2Ge_Max}) y una $R_c$ intermedia $2.75\cdot 10^{-3} \ m^2 K/W$ (\subref{fig:Prc_SsSiO2Ge_Inter}) frente a las diferentes alturas de nano-espaciador.}
	\label{fig:Pcond2_SsSiO2Ge}
\end{figure}
Las potencias para las nTPV con $R_c$ calculadas son del orden de los mil $n W$, y como se pueden observar en las figuras \ref{fig:Pcond2_SsSiO2Ge} (\subref{fig:Prc_SsSiO2Ge_Inter}) y (\subref{fig:Prc_SsSiO2Ge_Max}) las potencias son aproximadamente rectas con muy poca variación entre cada altura consecutiva de nano-espaciador, pudiéndose considerarse constantes para unos valores de $\sim 2.534 \mu W$ para la nTPV de $R_c$ calculada intermedia y $\sim 1.268 \mu W$ para la nTPV de $R_c$ calculada máxima.\\\\
Se calcula la relación de cada una de las potencias de conducción de las nTPV con $R_c$ respecto a la potencia conducida sin $R_c$, obteniéndose una reducción media muy grande de un 99.98\% para las nTPVs con $R_c$ calculadas porque la resistencia total es mucho mayor que para la $R_c$ empírica, reducción máxima de un 96.50\%.\\\\
Dado que la resistencia de contacto en las simulaciones es constante, no se procede a calcular un modelo analítico que relacione las potencias de conducción con las alturas de los nano-espaciadores y con la resistencia de contacto, y por la alta complejidad de la resistencia de contacto en la realidad, es decir, su alta dependencia en varios parámetros como la presión, temperatura, conductividades de los materiales, entre otros.
%%%%%%%%%%%%%%%%%%%%%%%%%%%%%%%%
%%%   CAMPO CERCANO    SS
\subsection{Radiación de campo cercano}
Para la simulación de radiación de campo cercano solo se consideran los resultados en el rango espectral de energía mayor a 0.7eV o longitudes de onda menores a $\sim$1.8 $\mu m$. Los datos disponibles del índice de refracción $n$ y coeficiente de extinción $k$ del acero inoxidable llegan hasta los 1.2 $\mu m$ de longitud de onda \cite{ss_optical_2017}, se realiza una extrapolación lineal hasta los 1.8$\mu m$ para poder obtener las potencias de radiación para diferentes distancias de separación.\\\\
Se comparan las potencias de radiación integradas en el rango espectral de energía mayor a 0.7eV para un emisor de Hierro ($Fe$), cuyos valores de $n$ y $k$ son conocidos para este rango energético, y de $SS$ para verificar que no existan grandes diferencias entre las potencias para un emisor de $SS$ respecto a las del $Fe$ como consecuencia de la extrapolación de los valores de $n$ y $k$ hasta los 1.8$\mu m$. Se compara respecto al $Fe$ porque es el elemento principal que componen a los aceros.
\begin{figure}[H]
	\centering
		\includegraphics[width=0.65\textwidth]{figuras/rad_mat/FevsSs.pdf}
	\caption{Potencias por radiación integradas en el rango espectral de energía mayor a 0.7eV para un emisor de $Fe$ y un emisor de $SS$ con receptor de $Ge$ frente a las diferentes distancias de separación de las placas.}
	\label{fig:FevsSs}
\end{figure}
Se obtienen valores similares para todo el rango de distancias de separación estudiadas (figura \ref{fig:FevsSs}), de 100 nm a 1000 nm. Por lo tanto, se utiliza para las simulaciones de radiación de campo cercano el emisor de $SS$ con $n$ y $k$ extrapolados linealmente.\\\\
Se procede a simular la radiación por campo cercano para solo un emisor de $SS$ (figura \ref{fig:SsGe}), obteniéndose la potencia radiada espectral de la figura \ref{fig:rad_SiGe}. Se puede observar como esta potencia es menor que para el emisor de $Si$ (figura \ref{fig:rad_SiGe}) y también las potencias integradas en el rango espectral de energía superior a 0.7eV del emisor $SS$ (figura \ref{fig:p_Eg_SsGe}) son menores que para las del emisor de $Si$ (figura \ref{fig:p_Eg_SiGe}) para todas las distancias de separación.
\begin{figure}[H]
	\centering
	\begin{subfigure}[b]{0.49\textwidth}
	\centering
		\includegraphics[width=1.00\textwidth]{figuras/Resultados/radiacion/SsGe.pdf}
	\caption{ }
	\label{fig:SsGe}
\end{subfigure}
\begin{subfigure}[b]{0.49\textwidth}
	\centering
		\includegraphics[width=1.00\textwidth]{figuras/Resultados/radiacion/p_Eg_SsGe.pdf}
	\caption{ }
	\label{fig:p_Eg_SsGe}
\end{subfigure}
\caption{(\subref{fig:SsGe}) Potencia radiada espectral ($q_w$) por campo cercano para un emisor de $SS$ y una célula de $Ge$. (\subref{fig:p_Eg_SsGe}) Potencia radiada para un rango de longitudes de onda cuya energía es mayor a los 0.7eV ($\sim 1.8 \ \mu m$).}
	\label{fig:rad_SsGe}
\end{figure}
Las potencias integradas no son todas mayores de 1000 $Wm^{-2}$ para las distancias de separación de 500 nm, 600 nm y 700 nm, pero en promedio se pueden considerar que las potencias se encuentran en el rango de los miles de $Wm^{-2}$ desde los 300 nm a los 1000 nm, creciendo en mayor proporción a partir de los 200 nm hasta los 100 nm.
%%%   DENSIDAD DE NANO-ESPACIADORES
\subsection{Densidad de nano-espaciadores}
Se realizan los cálculos de las densidades de nano-espaciadores para las relaciones de las potencias de radiación respecto a las de conducción. Para el caso de la nTPV sin $R_c$ se gráfica solo para potencias de radiación 3 veces superior a las potencias de conducción. Para el caso con $R_c$ se gráfica para potencias de readiación de un orden de magnitud superior a las de conducción.
\begin{figure}[H]
	\centering
	\begin{subfigure}[b]{0.49\textwidth}
		\centering
			\includegraphics[width=1.00\textwidth]{figuras/Resultados/RelacionCondRad/SS.png}
		\caption{ }
		\label{fig:rel_SsSiO2Ge}
	\end{subfigure}
	\hfill
	\begin{subfigure}[b]{0.49\textwidth}
		\centering
			\includegraphics[width=1.00\textwidth]{figuras/Resultados/RelacionCondRad/SS_Rc_empirico.png}
		\caption{ }
		\label{fig:rel_SsSiO2Ge_Rc_emp}
	\end{subfigure}
	\caption[Relación entre la potencia radiada para energías mayores e igual a 0.7 eV y la potencia transmitida por conducción para distintas densidades de nano-espaciadores (eje y) y distintas alturas de los mismos (eje x), con (\subref{fig:rel_SsSiO2Ge_Rc_emp}) y sin (\subref{fig:rel_SsSiO2Ge}) $R_c$ térmica para una célula de 1 $cm^2$ y emisor de $SS$. La barra de colores lateral representa los colores asociados a cada uno de los valores de las relaciones de potencias, con los contornos de las relaciones más significativas representadas en las gráficas.]{\small Relación entre la potencia radiada para energías mayores e igual a 0.7 eV y la potencia transmitida por conducción para distintas densidades de nano-espaciadores (eje y) y distintas alturas de los mismos (eje x), con (\subref{fig:rel_SsSiO2Ge_Rc_emp}) y sin (\subref{fig:rel_SsSiO2Ge}) $R_c$ térmica para una célula de 1 $cm^2$ y emisor de $SS$. La barra de colores lateral representa los colores asociados a cada uno de los valores de las relaciones de potencias, con los contornos de las relaciones más significativas representadas en las gráficas.
%	Densidades de nano-espaciadores para todas las relaciones de las potencias de radiación respecto a las conducción para un emisor de $SS$ frente a las altura de los nano-espaciadores para un emisor y una célula de 1 $cm^2$ sin $R_c$ (\subref{fig:rel_SsSiO2Ge}) y con $R_c$ empírica (\subref{fig:rel_SsSiO2Ge_Rc_emp}). El espectro de colores para cada relación entre las potencias de cada gráfica está indicado en las barra de colores lateral en el rango de valores que se encuentran en los extremos de cada barra.
	}
	\label{fig:rels_SsSiO2Ge_PnvsRc}
\end{figure}
Las densidades de nano-espaciadores obtenidas son menores que para la nTPV de $Si-SiO_2-Ge$ para todas las relaciones de potencias, siendo para el caso sin $R_c$ entre 1 a 2 nano-espaciadores menos por $cm^2$. Para las densidades de las nTPVs con $R_c$ para una altura de nano-espaciadores de 100 nm la densidad disminuye de unos $\sim$30 $n^{\circ} esp/cm^2$ (figura \ref{fig:rel_SiSiO2Ge_Rc}) para la nTPV con emisor de $Si$ a unos $\sim$19 $n^{\circ} esp/cm^2$ (figura \ref{fig:rel_SsSiO2Ge_Rc_emp}) para la nTPV con emisor de $SS$ porque la potencia radiada es menor para el emisor de $SS$ que para el emisor de $Si$.
\begin{figure}[H]
	\centering
	\begin{subfigure}[b]{0.49\textwidth}
		\centering
			\includegraphics[width=1.00\textwidth]{figuras/Resultados/RelacionCondRad/SS_Rc_Intermedio.png}
		\caption{ }
		\label{fig:rel_SsSiO2Ge_Rc_inter}
	\end{subfigure}
	\hfill	
	\begin{subfigure}[b]{0.49\textwidth}
		\centering
			\includegraphics[width=1.00\textwidth]{figuras/Resultados/RelacionCondRad/SS_Rc.png}
		\caption{ }
		\label{fig:rel_SsSiO2Ge_Rc_max}
	\end{subfigure}
	\caption[Relación entre la potencia radiada para energías mayores e igual a 0.7 eV y la potencia transmitida por conducción para distintas densidades de nano-espaciadores (eje y) y distintas alturas de los mismos (eje x), con $R_c$ calculada intermedia (\subref{fig:rel_SsSiO2Ge_Rc_inter}) y con $R_c$ calculada máxima (\subref{fig:rel_SsSiO2Ge_Rc_max}) $R_c$ térmica para una célula de 1 $cm^2$ y un emisor de $SS$. La barra de colores lateral representa los colores asociados a cada uno de los valores de las relaciones de potencias, con los contornos de las relaciones más significativas representadas en las gráficas.]{\small Relación entre la potencia radiada para energías mayores e igual a 0.7 eV y la potencia transmitida por conducción para distintas densidades de nano-espaciadores (eje y) y distintas alturas de los mismos (eje x), con $R_c$ calculada intermedia (\subref{fig:rel_SsSiO2Ge_Rc_inter}) y con $R_c$ calculada máxima (\subref{fig:rel_SsSiO2Ge_Rc_max}) $R_c$ térmica para una célula de 1 $cm^2$ y un emisor de $SS$. La barra de colores lateral representa los colores asociados a cada uno de los valores de las relaciones de potencias, con los contornos de las relaciones más significativas representadas en las gráficas.
%Densidades de nano-espaciadores para todas las relaciones de las potencias de radiación respecto a las conducción para un emisor de $SS$ frente a las altura de los nano-espaciadores para un emisor y una célula de 1 $cm^2$ con $R_c$ calculada intermedia (\subref{fig:rel_SsSiO2Ge_Rc_inter}) y con $R_c$ calculada máxima (\subref{fig:rel_SsSiO2Ge_Rc_max}). El espectro de colores para cada relación entre las potencias de cada gráfica está indicado en las barra de colores lateral en el rango de valores que se encuentran en los extremos de cada barra.
}
	\label{fig:rels_SsSiO2Ge_Prc1vsPrc2}
\end{figure}
Para los casos con $R_c$ calculadas, las densidades de nano-espaciadores para una relación mínima de potencias de un orden de magnitud son del orden de $10^4 \ n^{\circ}esp/cm^2$ y disminuyendo al orden de los miles de $n^{\circ}esp/cm^2$ para relaciones de potencias de dos ordenes de magnitud. Para las relaciones de las potencias de dos ordenes de magnitud las densidades de nano-espaciadores a una altura de nano-espaciadores de 100 nm son $\sim$ 2500 $n^{\circ} esp/cm^2$ para la nTPV con $R_c$ intermedia y $\sim$ 1000 $n^{\circ} esp/cm^2$ para la nTPV con $R_c$ máxima (figuras \ref{fig:rels_SsSiO2Ge_Prc1vsPrc2} \subref{fig:rel_SsSiO2Ge_Rc_inter} y \subref{fig:rel_SsSiO2Ge_Rc_max}).\\\\
Se observa que la relación entre las densidades de nano-espaciadores para ambos casos de $R_c$ calculada es aproximadamente igual a la relación entre ambas $R_c$ porque son constantes y solo afectan a la conducción en las simulaciones. El conocer estas relaciones es de utilidad para poder determinar en una primera instancia que valor de $R_c$ sería necesario como mínimo para tener la relación de potencias deseada con una cierta cantidad de nano-espaciadores de cierta altura.\\\\
También es importante tener en cuenta la variación de la densidad de nano-espaciadores o de la potencia de radiación entre su valor máximo y mínimo para una relación de potencias de un orden de magnitud porque las superficies tanto del emisor como de la célula presentan curvaturas que provocan un aumento o disminución de las distancias de separación entre ambos componentes, por ende, un aumento o disminución de las potencias de radiación útiles para su conversión en electricidad.\\\\
Para estudiar dicha variación, se calcula la reducción máxima de la densidad de nano-espaciadores para una relación de potencias de un orden de magnitud, suponiendo que todos los nano-espaciadores son de 100 nm de altura. Para el mejor caso se considera que la distancia de separación entre el emisor y la célula es igual a la altura de los nano-espaciadores, 100 nm, y para el peor caso es de 600 nm, por presentar la menor potencia de radiación.
Se obtiene que para todos los casos con $R_c$, la reducción de las densidades entre el mejor y peor caso supuestos es de un 70.95\%. Las nTPV con $R_c$ calculadas no se ven significativamente afectas por la reducción de la densidad de nano-espaciadores, a diferencia de la $R_c$ empírica, porque aún así la densidad es lo suficientemente alta para poder mantener una separación más estable entre le emisor y la célula, y mantener una relación de potencias más estable.\\\\
Por último, se recopilan los resultados obtenidos de las simulaciones en la tabla \ref{tab:SsSiO2Ge} para tener una mejor idea de los valores numéricos y de la magnitud de los resultados.
\begin{table}[H]
	\centering
	\caption{Tabla de recopilación de los resultados de las simulaciones de transmisión de calor por conducción y radiación de campo cercano para una nTPV de emisor de $SS$.}
		\begin{tabular}{|c||c|c|c|c||c|}
		\hline
		\multirow{2}{*}{ }& \multicolumn{5}{c|}{\textbf{\large Potencias según transmisión del calor}}\\ \cline{2-6}
& \multicolumn{4}{c||}{Conducción (W/nº esp.)}& Radiación $(W/m^2)$\\ \hline
Dist. (nm)&$P_{Normal}$&$P_{R_c-Cal.Max}$&$P_{R_c-Cal.Inter}$&$P_{R_c-Empirico}$&$P_{Eg>0.7eV}$\\ \hline \hline
100&4,80E-02&1,26815E-06&2,53438E-06&1,68E-03&3,13E+03\\ \hline 
200&3,42E-02&1,26813E-06&2,53431E-06&1,65E-03&1,73E+03\\ \hline 
300&2,64E-02&1,26811E-06&2,53424E-06&1,62E-03&1,24E+03\\ \hline 
400&2,15E-02&1,26809E-06&2,53417E-06&1,60E-03&1,01E+03\\ \hline 
500&1,81E-02&1,26808E-06&2,53410E-06&1,57E-03&9,11E+02\\ \hline 
600&1,56E-02&1,26806E-06&2,53403E-06&1,54E-03&9,10E+02\\ \hline 
700&1,37E-02&1,26804E-06&2,53396E-06&1,52E-03&9,93E+02\\ \hline 
800&1,22E-02&1,26802E-06&2,53388E-06&1,50E-03&1,10E+03\\ \hline 
900&1,11E-02&1,26800E-06&2,53381E-06&1,48E-03&1,14E+03\\ \hline 
1000&1,01E-02&1,26799E-06&2,53374E-06&1,45E-03&1,08E+03\\ \hline 
		\end{tabular}
	\label{tab:SsSiO2Ge}
\end{table}
\vfill \newpage
%%%%%%%%%%%%%%%%%%%%%%%%%%%%%%%%%%%
%%%%%%%%%%%    SiC-SiO2-Ge
\section{Resultados de las simulaciones para una nTPV de SiC-SiO2-Ge}\label{sec:res_SiCSiO2Ge}
Se procede a estudiar el caso de una nTPV con un emisor de Carburo de Silicio ($SiC$) por ser una material con mayor conductividad térmica que el $Si$ y el $SS$ a 800\textdegree C, que rondan unos 60 $W/(m^2 K)$ para el emisor de $SiC$, 26 $W/(m^2 K)$ para el $Si$ y 30 $W/(m^2 K)$ para el $SS$. Además el $SiC$ tiene un punto de fusión mayor que el $Si$ y el $SS$, siendo una característica del material de gran utilidad para su uso en baterías para almacenar la energía en forma de calor.\\\\
Otra razón de estudiar la nTPV con emisor de $SiC$ es por su utilización en \cite{doi:Near_field_ThinFilm} en una capa fina depositada sobre el emisor de una nTPV para aumentar el flujo de calor por radiación de campo cercano simulado entre el emisor y el receptor de $SiC$ grueso.
%%%%%    CFD
\subsection{Simulaciones de CFD}
Se realizan las simulaciones de transmisión de calor por conducción, obteniéndose resultados similares a los casos anteriores porque la mayor cantidad de caída de temperatura se produce en el nano-espaciador por tener una baja conductividad térmica respecto al emisor y la célula, como se observa en las figuras \ref{fig:relPrc_SiCSiO2Ge} \subref{fig:Prc_SiCSiO2Ge} y \subref{fig:relPrc_SiCSiO2Ge}.
\graphicspath{ {./figuras/Resultados/conduccion/pdf/} }
\begin{figure}[H]
	\centering
	\begin{subfigure}[b]{0.49\textwidth}
		\centering
			\includegraphics[width=1.00\textwidth]{Pn_SiCSiO2Ge.pdf}
		\caption{ }
		\label{fig:Prc_SiCSiO2Ge}
	\end{subfigure}
	\hfill
	\begin{subfigure}[b]{0.49\textwidth}
		\centering
			\includegraphics[width=1.00\textwidth]{Prc_SiCSiO2Ge.pdf}
		\caption{ }
		\label{fig:relPrc_SiCSiO2Ge}
	\end{subfigure}
	\caption{Potencias transmitidas por conducción para una nTPV $SiC-SiO_2-Ge$ sin $R_c$ (\subref{fig:Prc_SiCSiO2Ge}) y con $R_c$ empírica de $4\cdot 10^{-6} \ m^2 K/W$ (\subref{fig:relPrc_SiCSiO2Ge}) frente a las diferentes alturas de nano-espaciador.}
	\label{fig:PrcCond_SiCSiO2Ge}
\end{figure}
Algo que si se nota es que la potencia máxima es un 15.75\% mayor que para el emisor de $Si$, pero la reducción de las potencias con $R_c$ respecto a sin $R_c$ es menor, no superando el 86.14\% a los 1000 nm. Siendo esta mayor que para el caso de la nTPV de emisor de $Si$, con una reducción de 85.69\% a los 1000 nm.\\\\
\subsection{Radiación de campo cercano}
Se realizan las simulaciones de transmisión de calor por radiación de campo cercano para el emisor de $SiC$, obteniéndose una potencia radiada espectral similar al del emisor de $Si$ para longitudes de onda menores a 2$\mu m$ (figura \ref{fig:rad_SiCGe}) y también tiene valores similares de la potencia radiada integrada en el rango espectral de energía mayor a 0.7eV frente a las distancias de separación respecto al emisor de $Si$ (figura \ref{fig:Prad_SiCGe}).
%%%           GRAFICAS
\graphicspath{ {./figuras/Resultados/radiacion} }
\begin{figure}[H]
	\centering
	\begin{subfigure}[b]{0.49\textwidth}
		\centering
		\includegraphics[width=1.00\textwidth]{SiCvsSi.pdf}
		\caption{ }
		\label{fig:Prad_SiCGe}
	\end{subfigure}
	\hfill
	\begin{subfigure}[b]{0.49\textwidth}
		\centering
		\includegraphics[width=1.00\textwidth]{SiCvsSi_rad.pdf}
		\caption{ }
		\label{fig:rad_SiCGe}
	\end{subfigure}
	\caption{(\subref{fig:Prad_SiCGe}) Potencias integradas en el rango espectral de energía mayor a 0.7eV para un emisor de $SiC$ y un emisor de $Si$ frente a las distancias de separación entre emisor y receptor. (\subref{fig:rad_SiCGe}) Potencia radiada espectral ($q_w$) para un emisor de $Si$ y $SiC$ con un receptor de $Ge$ para una distancia de separación de 100nm frente longitudes de onda.}
	\label{fig:rads_SiCGe}
\end{figure}
Aunque la potencia radiada espectral difiere al aumentar la longitud de onda ($\lambda$), no supone una gran diferencia entre ambos emisores porque en el rango energético de interés ($E>0.7eV$) las curvas son bastante similares, provocando que las potencias integradas en dicho rango energético sean similares también. Este rango se encuentra en la parte inferior izquierda de la figura \ref{fig:SiCSiC}, que va desde el origen hasta aproximadamente 1.8 $\mu m$.
\begin{figure}[H]
	\centering
		\includegraphics[width=0.65\textwidth]{figuras/Resultados/radiacion/SiCSiC.pdf}
	\caption{Potencias radiadas espectrales ($q_w$) frente a las longitudes de onda para un sistema de $SiC-SiC$ frente a diferentes distancias de separación entre placas.}
	\label{fig:SiCSiC}
\end{figure}
También se simula un sistema con receptor de $SiC$ y emisor de $SiC$ para observar los efectos de la \gls{wres} sobre las potencias de radiación frente a las longitudes de onda. Se observa en la figura \ref{fig:SiCSiC} que la \gls{wres} es aproximadamente $1.57\cdot 10^{14} \ rad/s$ ($\lambda = 12\ \mu m$) porque existe un pico de potencia espectral máxima que disminuye al aumentar la distancia de separación entre emisor y receptor, pudiéndose aprovechar dicha potencia para la conversión en electricidad si mediante una combinación de materiales se consigue que la frecuencia se encuentre dentro del rango espectral útil.
%%%    DENSIDAD
\subsection{Densidad de nano-espaciadores}
Se procede a calcular las densidades de nano-espaciadores por centímetro cuadrado para la nTPV con emisor de $SiC$ para los casos de sin $R_c$ (figura \ref{fig:rel_SiCSiO2Ge}) y con $R_c$ (figura \ref{fig:rel_SiCSiO2Ge_Rc}) de $4\cdot 10^{-6} \ m^2 K/W$ \cite{nf_TPV_Pillars_SiO2}. Obteniéndose unos resultados muy parecidos a los de la nTPV de emisor de $Si$ porque los resultados obtenidos de las simulaciones también son bastante cercanos.
\graphicspath{ {./figuras/Resultados/RelacionCondRad} }
\begin{figure}[H]
	\centering
	%% Si-SiO2-Si Eg
	\begin{subfigure}[b]{0.49\textwidth}
		\centering
		\includegraphics[width=1.00\textwidth]{SiC_Ge.png}
		\caption{ }
		\label{fig:rel_SiCSiO2Ge}
	\end{subfigure}
	\hfill
	\begin{subfigure}[b]{0.49\textwidth}
		\centering
		\includegraphics[width=1.00\textwidth]{SiC_Rc.png}
		\caption{ }
		\label{fig:rel_SiCSiO2Ge_Rc}
	\end{subfigure}
	\caption[Relación entre la potencia radiada para energías mayores e igual a 0.7 eV y la potencia transmitida por conducción para distintas densidades de nano-espaciadores (eje y) y distintas alturas de los mismos (eje x), con (\subref{fig:rel_SiCSiO2Ge_Rc}) y sin (\subref{fig:rel_SiCSiO2Ge}) $R_c$ térmica para una célula de 1 $cm^2$ y emisor de $SiC$. La barra de colores lateral representa todas las relaciones entre ambas potencias en el rango de los valores extremos que se muestran en la misma.]{\small
	Relación entre la potencia radiada para energías mayores e igual a 0.7 eV y la potencia transmitida por conducción para distintas densidades de nano-espaciadores (eje y) y distintas alturas de los mismos (eje x), con (\subref{fig:rel_SiCSiO2Ge_Rc}) y sin (\subref{fig:rel_SiCSiO2Ge}) $R_c$ térmica para una célula de 1 $cm^2$ y emisor de $SiC$. La barra de colores lateral representa todas las relaciones entre ambas potencias en el rango de los valores extremos que se muestran en la misma.
	%Densidades de nano-espaciadores ($n^{\circ}esp/cm^2$) frente a las alturas de los nano-espaciadores para todas las relaciones de la potencia de radiación en el rango espectral de energías mayores e igual a 0.7 eV respecto a las potencias de conducción (dependientes en la densidad de nano-espaciadores) mayores a 3 para el caso sin $R_c$ (\subref{fig:rel_SiCSiO2Ge}) y mayores a 10 para el caso con $R_c$ (\subref{fig:rel_SiCSiO2Ge_Rc}). Donde las barras de colores laterales de cada gráfica representan los colores asociados a cada uno de los valores de las relaciones de las potencias, con los contornos de las relaciones más significativas representadas en las gráficas.
	}
	\label{fig:relation_SiCSiO2Ge}
\end{figure}
La principal diferencia es en las densidades de nano-espaciadores para el caso de la nTPV sin $R_c$ (figura \ref{fig:rel_SiCSiO2Ge}), siendo esta menor que la del emisor de $Si$ para todo el rango de alturas de los nano-espaciadores porque la conductividad térmica del $SiC$ a 800\textdegree C es mayor que para el $Si$, por lo tanto, la resistencia térmica del sistema disminuye, teniendo así unas menores densidades. Para el caso de la nTPV con $R_c$, las diferencias no son tan notables como en el caso sin $R_c$ porque la componente con mayor relevancia es la resistencia de contacto.\\\\
Por último, se recopilan los resultados obtenidos de las simulaciones de transferencia de calor por conducción y radiación de campo cercano en la tabla \ref{tab:SiCSiO2Ge}, los resultados se muestran en notación científica para simplificar el análisis de los mismos.
\begin{table}[h]
	\centering
	\caption{Tabla de recopilación de los resultados de las simulaciones de transmisión de calor por conducción y radiación de campo cercano para una nTPV de emisor de $SiC$.}
		\begin{tabular}{|c||c|c||c|c|}
		\hline
\multirow{2}{*}{ }& \multicolumn{4}{c|}{\textbf{\large Potencias según transmisión del calor}}\\ \cline{2-5}
& \multicolumn{2}{c||}{Conducción (W/nº esp.)}& \multicolumn{2}{c|}{Radiación $(W/m^2)$}\\ \hline
Dist. (nm)&$P_{Normal}$&$P_{R_c-Empirico}$&$P_{Eg>0.7eV}$&$P_{full}$\\ \hline \hline
100&6,23E-02&1,69E-03&4,90E+03&1,50E+05\\ \hline 
200&4,09E-02&1,66E-03&3,00E+03&1,01E+05\\ \hline 
300&3,03E-02&1,63E-03&2,21E+03&7,80E+04\\ \hline 
400&2,39E-02&1,61E-03&1,82E+03&6,43E+04\\ \hline 
500&1,98E-02&1,58E-03&1,64E+03&5,52E+04\\ \hline 
600&1,68E-02&1,55E-03&1,60E+03&4,87E+04\\ \hline 
700&1,47E-02&1,53E-03&1,67E+03&4,40E+04\\ \hline 
800&1,30E-02&1,51E-03&1,76E+03&4,06E+04\\ \hline 
900&1,17E-02&1,49E-03&1,81E+03&3,80E+04\\ \hline 
1000&1,06E-02&1,46E-03&1,75E+03&3,61E+04\\ \hline 
		\end{tabular}
	\label{tab:SiCSiO2Ge}
\end{table}

\vfill \newpage
%%%%%%%%%%%%%%%%%%%%%%%%%%%%%%%%%%%%%%%%%%%%
%%%     NUMERO DE NANO-ESPACIADORES
%%%%%%%%%%%%%%%%%%%%%%%%%%%%%%%%%%%%%%%%%%%%
\section{Densidad de nano-espaciadores para soportar la carga}\label{sec:densidad_carga}
Dado que los nano-espaciadores tendrán que soportar el peso del emisor o la célula se estudia la cantidad mínima de nano-espaciadores necesarios en un centímetro cuadrados para soportar la carga. También se estudia la densidad de nano-espaciadores para el caso de que se aplique una presión de una atmósfera a una de las caras para asegurar que las distancias de separación sean lo más uniformes posibles.\\\\
La resistencia a compresión del $SiO_2$ es de unos $1.16 \ GPa$\footnote{El valor se ha obtenido de la base de datos de \hyperref[sec:grantaEduPack]{Granta EduPack 2021 R2} para el $SiO_2$ de nombre \textbf{Silica (quartz fused)}, que presenta una pureza del 99.9\%} y la densidad mayor entre los materiales de emisor y células es el acero inoxidable 304, con una densidad aproximada de $8 \ g/cm^3$ o $8\cdot 10^3 kg/m^3$. Para un emisor $SS$ de área $1 \ cm^2$ y espesor de $0.5 \ cm$ la masa total es de unos 4 $gr$ o un peso de unos $\sim 0.04 \ N$, obteniéndose una densidad mínima de nano-espaciadores para solo soportar la carga del emisor de $4 \ n^{\circ}esp/cm^2$, un valor bastante pequeño para disminuir las curvaturas por flexión o imperfecciones del acabado de la superficie del emisor, pero teniendo una mayor relación entre las potencias de radiación respecto a las de conducción, es decir, mayor transferencia de potencia útil respecto a las pérdidas por conducción, disminuyendo así el aumento indeseado de la temperatura de la célula.\\\\
Al aplicarse una presión de una atmósfera sobre la superficie superior del emisor de la nTPV, la fuerza que soportan los nano-espaciadores aumenta hasta los $10.17 \ N$ generando que se necesiten una densidad mínima de nano-espaciadores de $975 \ n^{\circ}esp/cm^2$ para poder soportar la carga. Está densidad mínima es bastante alta respecto a las obtenidas en todos los casos de nTPVs de célula de germanio estudiados con resistencia de contacto empírica de $4\cdot 10^{-6} \ m^2 K/W$, pero no es tan elevada para el caso de la nTPV $SS-SiO_2-Ge$ con las $R_c$ calculadas para una relación de potencias mayor de un orden de magnitud (figuras \ref{fig:rels_SsSiO2Ge_Prc1vsPrc2} \subref{fig:rel_SsSiO2Ge_Rc_inter} y \subref{fig:rel_SsSiO2Ge_Rc_max}) porque sus valores de $R_c$ son mayores que la $R_c$ empírica y por ende, disminuyen las pérdidas que provoca un aumento de las densidades.
\vfill \newpage
%%%%%%%%%%%%%%%%%%%%%%%%%%%%%%%%%%%%%%%%%%%%%%%%%%%%%%%%%%%
%%    COMPARACION MATERIAL NANO-ESPACIADORES  Ctrl+Q to comment and Ctrl+W to uncomment
%%%%%%%%%%%%%%%%%%%%%%%%%%%%%%%%%%%%%%%%%%%%%%%%%%%%%%%%%%%%%
\section{Resultados de las simulaciones para unas nTPVs de nano-espaciadores de Si}\label{sec:res_XxSiGe}
\graphicspath{ {./figuras/Resultados/DiffMatEsp} }
Como último caso de simulación se simula la transmisión de calor en CFD por conducción para una nTPV de nano-espaciador de $Si$ y un emisor de $Si$, $SS$ y $SiC$ con resistencias de contacto empíricas de $4\cdot 10^{-6} \ m^2 K/W$, por si durante el proceso de deposición del nano-espaciador no se llegara a conseguir depositar $SiO_2$ sino un material con características semejantes a las del $Si$, esto ya ha sucedido en unos ensayos para la fabricación de los nano-espaciadores en la IES. 
%Esto sucede porque durante unas pruebas de fabricación del $SiO_2$ en la IES, no siempre se consiguía $SiO_2$ de buena calidad y que en el peor de los casos el $SiO_2$ fabricado podría tener unas características más similares al $Si$ que al $SiO_2$ ideal.
% comparar el efecto de cambiar el material del nano-espaciador
\begin{table}[H]
	\centering
	\caption{Tabla de recopilación de los resultados de las simulaciones de transferencia de calor por conducción para las nTPVs con nano-espaciador de Si y emisores de $Si$, $SS$ y $SiC$ respecto a cada altura del nano-espaciador.}
		\begin{tabular}{|c||c|c|c|}
		\hline
		 \multicolumn{4}{|c|}{\textbf{Potencias de Conducción ($mW$)}}\\	\hline
		Dist. (nm)&$P_{R_c-Si}$&$P_{R_c-SS}$&$P_{R_cSiC}$\\ \hline \hline 
		100&$1.7136$&$1.7113$&$1.7261$\\ \hline 
		200&$1.7135$&$1.7112$&$1.7260$\\ \hline 
		300&$1.7134$&$1.7110$&$1.7258$\\ \hline 
		400&$1.7132$&$1.7109$&$1.7257$\\ \hline 
		500&$1.7130$&$1.7107$&$1.7255$\\ \hline 
		600&$1.7128$&$1.7105$&$1.7252$\\ \hline 
		700&$1.7126$&$1.7103$&$1.7250$\\ \hline 
		800&$1.7125$&$1.7102$&$1.7249$\\ \hline 
		900&$1.7122$&$1.7099$&$1.7246$\\ \hline 
		1000&$1.7121$&$1.7098$&$1.7244$\\ \hline
		\end{tabular}
	\label{tab:nanoEspaciadorDeSi}
\end{table}
En primer lugar, de los resultados de las simulaciones de CFD, recopiladas en la tabla \ref{tab:nanoEspaciadorDeSi}, se observa como las pérdidas de conducción son mayores para los nano-espaciadores de $Si$ (figura \ref{fig:prc_xxSi}) que para los nano-espaciador de $SiO_2$ (figura \ref{fig:prc_xxSiO2}) porque la conductividad del $Si$ es mayor que la del $SiO_2$. Por lo tanto, la resistencia térmica del nano-espaciador es menos significativa y la resistencia de contacto tiene una mayor importancia.
\begin{figure} [H]%
	\centering
	\begin{subfigure}[b]{0.48\textwidth}%
			\includegraphics[width=\columnwidth]{Prc_XxSiGe}%
			\caption{}%
			\label{fig:prc_xxSi}%
	\end{subfigure}
	\hfill
	\begin{subfigure}[b]{0.48\textwidth}%
			\includegraphics[width=\columnwidth]{Prc_XxSiO2Ge}%
			\caption{}%
			\label{fig:prc_xxSiO2}%
	\end{subfigure}
	\caption{Potencias transmitidas por conducción de unas nTPVs con emisores de $Si$, $SS$ y $SiC$ frente a las alturas de un solo nano-espaciador en nanometros para un nano-espaciador de $Si$ (\subref{fig:prc_xxSi}) y para un nano-espaciador de $SiO_2$ (\subref{fig:prc_xxSiO2}).}%
	\label{fig:prc_xxXX}%
\end{figure}
En segundo lugar, se observa como las curvas de las potencias de conducción no varían tanto entre su valor máximo y valor mínimo, siendo para el caso de las nTPVs con nano-espaciadores de $Si$ el que presenta la menor variación de los dos, siendo esta menor de $0.002 \ mW$ (figura \ref{fig:prc_xxSi}), a diferencia de los nano-espaciadores de $SiO_2$ cuya variación es menor de $0.25 \ mW$ (figura \ref{fig:prc_xxSiO2}). También se observa con mayor detalle como la diferencias de las resistencias térmicas de los materiales afecta al flujo de calor, aumentando con la disminución de la resistencia térmica o aumento de la conductividad térmica, cuyos valores se mencionan en la sección \ref{sec:res_SiCSiO2Ge}.\\\\
En tercer lugar, las potencias de conducción son muy similares para ambos materiales de nano-espaciador con la altura de 100 nm porque la resistencia térmica de los nano-espaciadores disminuye considerablemente con la disminución de la distancia, por ser proporcional a ella. Esto produce que la $R_c$ sea la principal resistencia térmica del sistema, por tal motivo, los valores de las potencias de conducción se asemejan a dicha altura del nano-espaciador.\\\\
Por último, se calculan las densidades de nano-espaciadores frente a las alturas de los nano-espaciadores para la nTPV de emisor de $SiC$ y nano-espaciadores de $Si$ (figura \ref{fig:prc_SiCSi}), y se compara con las densidades de la nTPV de emisor de $SiC$ pero nano-espaciadores de $SiO_2$ (figura \ref{fig:prc_SiCSiO2}). Se observa que existe una disminución de las densidades para cada relación de potencias y todas las alturas de los nano-espaciadores, siendo la más notable a los 1000 nm de altura y relación de potencias de un orden de magnitud, pasando la densidad de ser $12 \ n^{\circ}esp/cm^2$ (figura \ref{fig:prc_SiCSiO2}) a ser $10 \ n^{\circ}esp/cm^2$ (figura \ref{fig:prc_SiCSi}). Esta poca variación para dos materiales distintos de nano-esapciadores se dá porque para ambos casos la nTPV tiene una $R_c$ que es la resistencia térmica más significativa del sistema.
\begin{figure} [h]%
	\centering
	\begin{subfigure}[b]{0.48\textwidth}%
			\includegraphics[width=\columnwidth]{rel_SiC}%
			\caption{}%
			\label{fig:prc_SiCSi}%
	\end{subfigure}
	\hfill
	\begin{subfigure}[b]{0.48\textwidth}%
			\includegraphics[width=\columnwidth]{SiC_Rc}%
			\caption{}%
			\label{fig:prc_SiCSiO2}%
	\end{subfigure}
	\caption[Relación entre la potencia radiada para energías mayores e igual a 0.7 eV y la potencia transmitida por conducción con $R_c$ para distintas densidades de nano-espaciadores (eje y) y distintas alturas de los mismos (eje x), para nTPV de 1 $cm^2$ con nano-espaciadores de $Si$ (\subref{fig:prc_SiCSi}) y de $SiO_2$ (\subref{fig:prc_SiCSiO2}). 
	La barra de colores lateral representa los colores asociados a cada uno de los valores de las relaciones de potencias, con los contornos de las relaciones más significativas representadas en las gráficas.]{\small
	Relación entre la potencia radiada para energías mayores e igual a 0.7 eV y la potencia transmitida por conducción con $R_c$ para distintas densidades de nano-espaciadores (eje y) y distintas alturas de los mismos (eje x), para nTPV de 1 $cm^2$ con nano-espaciadores de $Si$ (\subref{fig:prc_SiCSi}) y de $SiO_2$ (\subref{fig:prc_SiCSiO2}). 
	La barra de colores lateral representa los colores asociados a cada uno de los valores de las relaciones de potencias, con los contornos de las relaciones más significativas representadas en las gráficas.
	%Densidades de nano-espaciadores ($n^{\circ}esp/cm^2$) frente a las alturas de los nano-espaciadores para todas las relaciones de la potencia de radiación en el rango espectral de energías mayores e igual a 0.7 eV respecto a las potencias de conducción con $R_c$ (dependientes en la densidad de nano-espaciadores) mayores a 10 para nTPVs de nano-espaciadores de $Si$ (\subref{fig:prc_SiCSi}) y $SiO_2$(\subref{fig:prc_SiCSiO2}) . Donde las barras de colores laterales de cada gráfica representan los colores asociados a cada uno de los valores de las relaciones de las potencias, con los contornos de las relaciones más significativas representadas en las gráficas.
	}%
	\label{fig:prc_SiCXX}%
\end{figure}