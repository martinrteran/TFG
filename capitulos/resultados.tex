\chapter{Resultados y discusión}

En este capítulo se presentan los resultados obtenidos de las simulaciones de transmisión de calor por conducción y radiación, comparando varios casos de estudio de diferentes materiales de emisor y célula, con varias distancias de separación entre el emisor y la célula. También se estudia los efectos de la porosidad del $SiO_2$ y la resistencia de contacto entre nano-espaciador y emisor.\\\\
Los casos a estudiar son los siguientes:
\begin{enumerate}
	\item Comprobación del correcto funcionamiento del método de extracción de los datos de la transmisión de calor por conducción en CFD \label{res:comprobacionCFD}.
	\item El caso más sencillo a simular para tener una primera idea del funcionamiento del sistema es el de emisor de $Si$ y célula de $Si$, representado como $Si-Si$.\label{res:caso_Si_Si}
	\item $Si_Ge$: es el siguiente caso más sencillo, donde la célula es la correspondiente a utilizar.\label{res:caso_Si_Ge}
	\item $SS_Ge$: es el caso de recuperación de calor en la industria porque en la industria se utiliza mucho el acero inoxidable, siendo los aceros inoxidables 304 y 304L muy populares y utilizados.\label{res:caso_SS_Ge}
	\item $SiC_Ge$: 
\end{enumerate}
\section{Comprobación de procedimiento de extracción de datos de CFD}
A partir del caso de emisor de $Si$, nano-espaciador de $SiO_2$ y célula de $Si$ se realiza una simulación con todos los valores de conductividades térmicas constantes a 25\textdegree C y altura de espaciador de 1000nm, comparándose así con los resultados teóricos. La conductividad térmica del $Si$ es 182.977 W/m\textdegree C y la del $SiO_2$ es 1.30067 W/m\textdegree C, que para una diferencia de temperatura de 760.2107\textdegree C el flujo de calor por conducción es 0.00889882 W que respecto al empírico es 0.00889793, habiendo un error de apenas el 0.01\% respecto al teórico.\\\\
Al ser el error relativo menos del 5\% el procedimiento de extracción de los datos de la simulación de transmisión de calor por conducción es correcta y se procede a usarse para el resto de casos.\\

\section{Caso 1: Sistema TPV de $Si-SiO_2-Si$}
El sistema TPV a estudiado está compuesto de un emisor de $Si$ a 800\textdegree C, una nano-espaciador de $SiO_2$ y una célula de $Si$ a 25\textdegree C. Se estudia el efecto de la porosidad del $SiO_2$ en la potencia de conducción total que fluye por un nano-espaciador de diferentes alturas, el efecto de la resistencia de contacto sobre la conductividad del sistema y la relación entre la potencia radiada y conducida para cada caso en el rango de longitudes de onda que trabaja la célula.\\\\
De las simulaciones de transmisión de calor de CFD, se obtienen los siguientes resultados:
\begin{table}[H]
	\centering
		\begin{tabular}{|c||c|c|c|c|}
		\hline
			Distancia (nm)&$P_{normal}$ (W)&$P_{R_c-Empirico}$&$P_{Porosidad25}$&$P_{Porosidad50}$\\ \hline \hline
			100&6,31E-02&1,69E-03&4,67E-02&2,30E-02\\ \hline
			200&3,99E-02&1,66E-03&2,78E-02&1,26E-02\\ \hline
			300&2,93E-02&1,63E-03&1,99E-02&8,69E-03\\ \hline
			400&2,32E-02&1,60E-03&1,55E-02&6,63E-03\\ \hline
			500&1,92E-02&1,58E-03&1,27E-02&5,36E-03\\ \hline
			600&1,64E-02&1,55E-03&1,08E-02&4,50E-03\\ \hline
			700&1,43E-02&1,53E-03&9,33E-03&3,88E-03\\ \hline
			800&1,27E-02&1,50E-03&8,24E-03&3,41E-03\\ \hline
			900&1,14E-02&1,48E-03&7,38E-03&3,04E-03\\ \hline
			1000&1,04E-02&1,46E-03&6,68E-03&2,74E-03\\ \hline
		\end{tabular}
	\caption{Flujos de calor del TPV $Si-SiO_2-Si$ para diferentes alturas del nano-espaciador y para los casos sin $R_c$, con $R_c$ de \cite{nf_TPV_Pillars_SiO2}, sin $R_c$ pero con las proporciones de las porosidades de \cite{ThermalConductivity_SiO2_2018} para un 25\% y un 50\%.}
	\label{tab:condTerSiSiO2Si}
\end{table}
\subsection{Efectos de la resistencia de contacto}


\subsection{Efectos de la porosidad}

\subsection{Relación entre conducción y radiación}

\section{Caso 2: Sistema TPV de $Si-SiO_2-Ge$}

\section{Caso 3: Sistema TPV de $SS-SiO_2-Ge$}

\section{Caso 4: Sistema TPV de $SiC-SiO_2-Ge$}

\section{Resultados}


\section{Discusión}