\chapter{Agradecimientos}

Agradezco a mi tutor Pablo por darme la oportunidad de realizar un trabajo dentro de las líneas de investigación de la termo-fotovoltaica y contribuir a esta misma, permitirme explorar por mi cuenta y experimentar como la investigación llega a ser.\\\\
A mi co-tutora Esther por brindarme tanto apoyo durante la realización del trabajo, explicándome conceptos en mayor detalle y permitirme ir más allá con el trabajo. También agradezco a Alejandro Datas por brindarme apoyo para no salirme de los objetivos de este trabajo.\\\\
Gracias a los tres por permitirme participar en la elaboración del póster presentado en la treceava conferencia mundial en la generación de electricidad termo-fotovoltaica (\href{https://www.tpv-13.org/}{TPV-13}) de Abril del 2022, compartida con la dieciochoava conferencia internacional de sistemas fotovoltaicos de concentración (\href{https://www.cpv-18.org/}{CPV-18}).\\\\  
A mi padre por siempre apoyarme durante la carrera, y a mi hermana Andrea por brindarme apoyo cuando más lo necesitaba y por siempre estar ahí encargándote de muchas cosas.\\\\
A mis amigos, Aiwen, ZhanYao y Teresa con los cuales he pasado estos duros cuatro años juntos estudiando, trasnochando realizando los trabajos de las asignaturas, viviendo nuevas experiencias y aprendiendo bastante cosas más allá de la ingeniería.\\\\
A Yvonne por empujarme a salir de mi zona de confort, hacer lo que hay que hacer cuando se debía y aprender a como seguir con la vida.\\\\
A Noelia por estar ahí gran parte del tiempo de realización de este trabajo, apoyándome en cada fase y recordándome de tomar descansos.\\\\
A la Asociación del Cubo por permitirme conocer a gente tan bonita como Rafael, que siempre ha brindado su apoyo y llegando a ser de gran ayuda para la realización de este trabajo. Al CREA por permitirme conocer a más gente que le encanta la electrónica y pasar unas buenos ratos con los proyectos.\\\\