\chapter{Conclusiones}
En este capítulo se presentan las conclusiones de las simulaciones realizadas y la viabilidad de cada caso estudiado, también se presentan los posibles desarrollos futuros en base a los resultados obtenidos en este trabajo.

\section{Conclusiones del trabajo}
En primer lugar, se concluye para el caso sencillo de una nTPV de $Si-SiO_2Si$ (sección \ref{sec:res_SiSiO2Si}) que el aumento de la porosidad del material del nano-espaciador disminuye las pérdidas de conducción por la disminución de la conductividad térmica, obteniéndose un modelo analítico que relaciona las pérdidas por conducción con la porosidad y la altura del nano-espaciador. También se concluye que la existencia de una resistencia de contacto es de gran importancia para la disminución de las pérdidas, pero no lo suficiente para que sea viable usar una célula de Silicio porque no se obtienen buenas relaciones entre potencia de radiación útil y pérdidas.\\\\
En segundo lugar, para el caso de una nTPV de $Si-SiO_2-Ge$ (sección \ref{sec:res_SiSiO2Ge}) las radiaciones frente al rango espectral son muy parecidos que para el caso de la nTPV de $Si-SiO_2-Si$, por lo tanto, que al disminuir la banda energética de corte aumenta la potencia total útil, concluyéndose que podría llegar a ser viable el sistema solo con la resistencia de contacto sí se consigue disminuir las imperfecciones en las superficies y las curvaturas.\\\\
Para el tercer caso, una nTPV de $SS-SiO_2-Ge$ (sección \ref{sec:res_SsSiO2Ge}) fue estudiada para un sistema de recuperación de calor residual, se concluye que a pesar de tener menor potencia integrada útil de radiación que el caso anterior de nTPV de célula de germanio se pueden conseguir unas altas densidades de nano-espaciadores y grandes relaciones de potencias entre la potencia útil y las pérdidas (calor transmitido por conducción) con el aumento significativo de la resistencia de contacto, siendo esta de gran importancia para la disminución de las pérdidas a cambio de disminuir la potencia total útil y siendo importante para determinar la viabilidad del sistema.\\\\
Para el cuarto caso, de una nTPV de $SiC-SiO_2-Ge$ (sección \ref{sec:res_SiCSiO2Ge}) se concluye que teniendo potencias de radiación y conducción muy parecidas al caso de la nTPV de emisor de silicio pero menor densidad de nano-espaciadores por cada relación de potencias, se puede llegar a aprovechar la frecuencia de resonancia sí se es capaz de desplazarla dicha frecuencia al rango de potencia útil de radiación en un sistema multi-capa con carburo de silicio en el emisor y el receptor.\\\\
Para la densidad mínima de nano-espaciadores (sección \ref{sec:densidad_carga}) se concluye que solo es viable para las nTPVs con resistencia de contacto, solo siendo viable con la $R_c$ empírica para cuando solo se considera el peso del emisor de acero inoxidable como la carga que soportan los nano-espaciadores y una densidad mínima de $3 \ n^{\circ}esp/cm^2$, también es viable para las nTPV con $R_c$ calculadas, que son tres ordenes de magnitud respecto a la empírica, con la carga del emisor y una presión de una atmósfera con una densidad mínima de $673 \ n^{\circ}esp/cm^2$.\\\\
El uso de diferentes materiales para los nano-espaciadores no es tan significativo para las pérdidas por conducción y tampoco lo es el material del emisor, siempre y cuando exista una resistencia de contacto de magnitud lo suficientemente grande para que sea la principal fuente de disminución de el flujo de calor por conducción. Esto se ha demostrado en las simulaciones de la sección \ref{sec:res_XxSiGe}, donde se corrobora que existe una diferencia no tan significativa de las pérdidas por conducción a pesar de un poco mayores, afectando en poca proporción a las densidades de nano-espaciadores para todas las relaciones de potencias para cada altura de nano-espaciador.\\\\
Por último, la resistencia de contacto es de gran importancia como se ha destacado hasta ahora en la disminución de pérdidas por conducción y el aumento de las densidades de nano-espaciadores para las relaciones de potencias, pero también genera la disminución de la expansión térmica de los nano-espaciadores produciendo así que no aumente excesivamente la presión entre emisor y nano-espaciadores que evita una mayor disminución de la propia resistencia de contacto, como se da en \cite{experimental_Rc_SS}, porque la mayor caída de temperatura se da en la resistencia de contacto.
\section{Desarrollos futuros}

Un posible desarrollo...