\chapter{Conclusiones}
En este capítulo se presentan las conclusiones de las simulaciones realizadas y la viabilidad de cada caso estudiado, también se presentan los posibles desarrollos futuros en base a los resultados obtenidos en este trabajo.

\section{Conclusiones del trabajo}
En primer lugar, se concluye para el caso sencillo de una nTPV de $Si-SiO_2Si$ que el aumento de la porosidad del material del nano-espaciador disminuye las pérdidas de conducción por la disminución de la conductividad térmica, obteniéndose un modelo analítico que relaciona las pérdidas por conducción con la porosidad y la altura del nano-espaciador. También se concluye que la existencia de una resistencia de contacto es de gran importancia para la disminución de las pérdidas, pero no lo suficiente para que con una célula de Silicio se obtengan buenas relaciones entre potencia de radiación útil y pérdidas.\\\\


\section{Desarrollos futuros}

Un posible desarrollo...