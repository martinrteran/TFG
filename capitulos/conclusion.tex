\chapter{Conclusiones}
En este capítulo se presentan las conclusiones de las simulaciones realizadas y la viabilidad de cada caso estudiado, también se presentan los posibles desarrollos futuros en base a los resultados obtenidos en este trabajo.

\section{Conclusiones del trabajo}
En primer lugar, se concluye para el caso sencillo de una nTPV de $Si-SiO_2Si$ (sección \ref{sec:res_SiSiO2Si}) que el aumento de la porosidad del material del nano-espaciador disminuye las pérdidas de conducción por la disminución de la conductividad térmica, obteniéndose un modelo analítico que relaciona las pérdidas por conducción con la porosidad y la altura del nano-espaciador. También se concluye que la existencia de una $R_c$ es de gran importancia para la disminución de las pérdidas, pero no lo suficiente para que sea viable usar una célula de Silicio porque no se obtienen buenas relaciones entre potencia de radiación útil y pérdidas.\\\\
En segundo lugar, para el caso de una nTPV de $Si-SiO_2-Ge$ (sección \ref{sec:res_SiSiO2Ge}) la potencia radiada espectral son muy parecidos que para el caso de la nTPV de $Si-SiO_2-Si$, por lo tanto, que al disminuir la banda energética de corte aumenta la potencia total útil, concluyéndose que podría llegar a ser viable el sistema solo con la $R_c$ sí se consigue disminuir las imperfecciones en las superficies y las curvaturas.\\\\
Para el tercer caso, una nTPV de $SS-SiO_2-Ge$ (sección \ref{sec:res_SsSiO2Ge}) fue estudiada para un sistema de recuperación de calor residual, se concluye que a pesar de tener menor potencia integrada útil de radiación que el caso anterior de nTPV de célula de $Ge$ se pueden conseguir unas altas densidades de nano-espaciadores y grandes relaciones de potencias entre la potencia útil y las pérdidas (calor transmitido por conducción) con el aumento significativo de la $R_c$, siendo esta de gran importancia para la disminución de las pérdidas a cambio de disminuir la potencia total útil y siendo importante para determinar la viabilidad del sistema.\\\\
Para el cuarto caso, de una nTPV de $SiC-SiO_2-Ge$ (sección \ref{sec:res_SiCSiO2Ge}) se concluye que teniendo potencias de radiación y conducción muy parecidas al caso de la nTPV de emisor de silicio pero menor densidad de nano-espaciadores por cada relación de potencias, se puede llegar a aprovechar la frecuencia de resonancia sí se es capaz de desplazarla dicha frecuencia al rango de potencia útil de radiación en un sistema multi-capa con carburo de silicio en el emisor y el receptor.\\\\
Para la densidad mínima de nano-espaciadores (sección \ref{sec:densidad_carga}) se concluye que solo es viable para las nTPVs con $R_c$. Para las nTPV con $R_c$ empírica solo son viables cuando se considera exclusivamente el peso del emisor de SS como la carga que soportan los nano-espaciadores, con una densidad mínima de $4 \ n^{\circ}esp/cm^2$ para poder soportar la carga. Y para las nTPV con $R_c$ calculadas son viables para ambos casos, cuando la carga que soportan los nano-espaciadores es el emisor y cuando se aplica una presión de una atmósfera, siendo para este último caso necesario una densidad mínima de $975 \ n^{\circ}esp/cm^2$ para poder soportar la carga.\\\\
%%%%%%%    Espaciador de Silicio
Por último, el uso de diferentes materiales para los nano-espaciadores no es tan significativo para las pérdidas por conducción y tampoco lo es el material del emisor, siempre y cuando exista una $R_c$ de magnitud lo suficientemente grande para que sea la principal fuente de disminución de el flujo de calor por conducción. Esto se ha demostrado en las simulaciones de la sección \ref{sec:res_XxSiGe}, donde se corrobora que existe una diferencia no tan significativa de las pérdidas por conducción a pesar de un poco mayores, afectando en poca proporción a las densidades de nano-espaciadores para todas las relaciones de potencias para cada altura de nano-espaciador.
%%%%%%%%%%%%%%%%%%%%%%%%%%%%%%%%%%%%%%%%%%%
%%           Desarrollos futuros
%%%%%%%%%%%%%%%%%%%%%%%%%%%%%%%%%%%%%%%%%%%
\section{Desarrollos futuros}
A continuación se detallan los posibles desarrollos a futuro a partir de los resultados y conclusiones obtenidos durante la realización del trabajo. Estos son principalmente nuevos trabajos que van más en detalle de cada una de las propiedades más importante del sistema nTPV.
\begin{itemize}
	\item Diseñar y desarrollar una aplicación para el cálculo de la transmisión de calor por radiación de campo cercano para sistemas de varias capas de diferentes materiales y temperaturas según las ecuaciones \eqref{eq:flujoPropNF} y \eqref{eq:flujoEvasNF} de \cite{nfTPV_fullEquations}.
	\item Estudiar que materiales, principalmente cerámicas por sus propiedades térmicas y altas fuerzas de compresión, se pueden utilizar para la fabricación de los nano-espaciadores con altas transmitancias para el rango espectral de potencias de radiación de campo cercano útiles de un sistema nTPV por permitir el paso de la mayoría de la radiación útil.
	\item Estudiar que procesos se pueden utilizar para aumentar la $R_c$ entre el emisor y los nano-espaciadores mediante el tratamiento de sus superficies de contacto.
	\item Estudiar como la expansión térmica y presión por el aumento de la temperatura se ve afectada por la resistencia de contacto, y como esta varía para un sistema nTPV de nano-espaciador de $SiO_2$ y célula de $Ge$.	
	%%%%%%%%%%%%%%%
	%%% EXTRA
	\item Estudiar que procesos de tratamiento de superficies se pueden utilizar para conseguir menos curvaturas en la célula y en el emisor, y para conseguir una mayor $R_c$.
	\item Estudiar el uso de de células termo-fotovoltaicas de unión múltiple para el aumento del rango espectral de potencia de radiación de campo cercano útil y el uso de filtros o superficies traseras de reflexión.
	\item Diseñar y fabricar un sistema nTPV con una BSR o un filtro sobre la superficie superior de la célula conectada a un regulador mppt, que sigue el punto de máxima potencia de la célula, para corroborar la validez de las ecuaciones de \cite{nfTPV_fullEquations} y obtener una relación analítica de las $R_c$ respecto a diferentes rangos de temperatura.
\end{itemize}