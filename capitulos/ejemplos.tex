\chapter{Cómo escribir en Latex}

\section{Citas}

%las referencias a artículos se ponen con \cite, 
%las referencias a imágenes \ref, 
%y las referencias a ecuaciones \eqref

Esto es un ejemplo de cita de un artículo %\cite{Brunete:2013}.


\section{Listas}

%itemize es una lista. Cada término lleva delante un \item
Ejemplo de lista de puntos:
\begin{itemize}
\item Ejemplo1.
\item Ejemplo2.
\end{itemize} 

Y lista numerada:
\begin{enumerate}
\item Elemento 1
\item Elemento 2
\end{enumerate}

\section{Tablas}

Ejemplo de tabla. Como se aprecia en la tabla \ref{tab:table_example}...
\begin{table}[tb]
\caption{Ejemplo de tabla}
\label{tab:table_example}
\begin{center}
\begin{tabular}{|c||c|c|}
\hline
One & Two & Three\\
\hline
F1A & F1B & F1C\\
F2A & F2B & F2C\\
\hline
\end{tabular}
\end{center}
\end{table}

\section{Referencia a una sección}
\label{sec:refsec}

Ejemplo de referencia a la sección \ref{sec:refsec}

\section{Texto}

Testo en \textbf{negrita} y \textit{cursiva}.

\section{Figuras}

Ejemplo de referencia a figura (figura \ref{fig:logo_upm}). Es importante que todas las figuras que aparezcan estén referenciadas, así como las tablas. En general las figuras se colocarán al principio o al final de cada página ([tb] en latex), a no ser que por alguna necesidad se deban colocar en una posición exacta ([h]).

%caption es el pie de foto, y label es el nombre que se da a la imagen para referenciarla después. label no puede llevar acentos y no se muestra de cara al documento final (es sólo interno).
\begin{figure}[tb]
\centering
\includegraphics[width=0.45\textwidth]{figuras/Logo_UPM.jpg}   
\caption{Logotipo de la UPM}
\label{fig:logo_upm}
\end{figure}