\chapter{Estado del arte}


En este capítulo se exponen los principales temas de interés para la realización de este trabajo, así como los artículos más representativos usados durante la realización de este trabajo.
\section{Termo-fotovoltaica}
La termo-fotovoltaica (TPV) se basa en convertir la radiación electromagnética proveniente de una fuente de calor en electricidad \cite{ThermophotovolticEnergyConversion_DATAS2021285}, teniendo la ventaja sobre la fotovoltaica solar de que el emisor no está en constante movimiento \cite{ThermophotovolticEnergyConversion_DATAS2021285}.\\

La primera célula Thero
\subsection{Célula fotovoltaica}
Dispositivo eléctrico que convierte la radiación solar en electricidad mediante el efecto fotovoltaico, generando una corriente proporcional a la intensidad de iluminación. En 

Hoy en día una célula fotovoltaica se compone en su forma más sencilla de dos capas de semiconductores dopados, uno con un exceso de huecos y otro con exceso de electrones libres, llamados respectivamente semiconductor tipo p y tipo n, que forman una unión p-n produciéndose una región despoblada de electrones libres y huecos que generan una diferencia de potencial entre cada capa \cite{PhotovoltaicCell_FullDescription_EstadoDelArte_KHALIGH2018725}.\\

Estas células se agrupan en un conjunto de módulos para aumentar la potencia de salida, los cuales se pueden conectar en serie para aumentar la tensión o en paralelo para la corriente, que en conjunto con un sistema de seguimiento del punto de máxima potencia se es capaz de extraer la mayor cantidad de energía de estos dispositivos como se describe en \cite{PhotovoltaicCell_FullDescription_EstadoDelArte_KHALIGH2018725}.
\subsection{Efecto fotovoltaico}
El efecto fotovoltaico fue primeramente reportado por Edmun Bequerel en 1839 cuando observaba el efecto de la luz en un electrodo de platino recubierto de plata inmerso en un electrólito produciéndose una corriente eléctrica \cite{PhotovoltaicEffect_History_SUDHAKAR2018117}, siendo usada en la primera aplicación industrial en 1954 \cite{PhotovoltaicEffect_History_IndustrialUse_DINCER2018707} .% con una es la excitación de un electrón a otro estado energético mayor cuando un la radiación electromagnética es absorbido por el material y se produce una diferencia de potencial.

\section{Transmisión de calor}
El calor es una forma de energía que se propaga entre distintos medios de tres formas distintas, por convección, radiación y conducción.
\subsection{Convección}
La transmisión de calor por convección se produce por la conducción de la energía cuando el fluido entra en contacto con el sólido y luego el transporte de la energía mediante el movimiento del fluido.
\subsection{Radiación}

\subsubsection{Plank}
\subsubsection{Campo Cercano}
\subsection{Conducción}
La transmisión de calor por conducción se dá a través de uno o varios cuerpos, producido por la diferencia de temperatura entre las caras opuestas del conjunto. Para una dimensión la conducción térmica se modela como $P_{cond}={\bigtriangleup T}/{R} $, siendo $R$ la resistencia térmica del sistema.
Para un solo material, la resistencia térmica se modela como $R = l/{\left(A\cdot h\right)}$, donde $l$ es la longitud del material, $A$ es la superficie y $h$ es la conductividad térmica del material.\\

Para varios materiales colocados en serie, es decir, el flujo de calor que los atraviesa es el mismo para todos, la resistencia de conducción se define como la sumatoria de todas las resistencias de cada material($R=\sum R_i$).

Para el caso de transmisión de calor en serie existe un conjunto de resistencias que se producen por las imperfecciones en las interfaces de contacto entre los materiales, a dicha resistencia se le llama resistencia de contacto.


\subsubsection{Resistencia de contacto}
La resistencia de contacto en la interfaz entre dos conductores produce una caída de temperatura significante, como se observa en el \cite{noauthor_parallel-plate_nodate}, la cual es dependiente de muchos parámetros, tales como la temperatura, la presión, la rugosidad, etc.\\

Esta gran cantidad de dependencias hace que sea difícil parametrizar su valor, por lo tanto, se utilizan valores empíricos como el obtenido en \cite{noauthor_parallel-plate_nodate} de unos $4E{-6} \ m^2K/W$, pero también se puede llegar a parametrizar para ciertos casos, como en \textbf{Otra citación}, donde modelan el cálculo de la resistencia de contacto entre dos aceros 304L a temperatura ambiente como $Modelo de resistencia de contacto$.