\chapter{Estado del arte}


En este capítulo se exponen los principales temas de interés para la realización de este trabajo, así como los artículos más representativos usados durante la realización de este trabajo.
\section{Termo-fotovoltaica}
Un generador termo-fotovoltaico(TPV) se basa en la conversión de energía calorífica en energía eléctrica mediante el efecto fotovoltaico a través una célula termo-fotovoltaica sin requerir ninguna parte móvil, conocido este tipo de sistemas de generación como motores pasivos de calor. El emisor se encuentra a una alta temperatura lo cual produce que se transmite el calor en forma de radiación que al llegar a la célula es reflejada, transmitida o absorbida, la porción de radiación absorbida excita a los electrones produciendo un par electrón-hueco sí solo sí la energía del fotón absorbido es mayor que la energía de la banda energética de la célula, al conectar los terminales de la célula a una carga se produce una corriente que alimenta a la carga proporcional a la intensidad lumínica, aquellos fotones con energía menor a la banda energética son suprimidos o reflejados para disminuir el flujo de calor\cite{Present_Efficiencies_and_Future_Opportunities_in_Thermophotovoltaics}.\\

La termofotovoltaica ha tenido mucho interés en la milicia, conduciendo a varias investigaciones para la búsqueda de un generador silencioso y portátil \cite{military_TPV}, en 2004 se cumplieron 40 años de investigación todavía sin conseguirse potencias por encima de 500W \cite{military_TPV_40Years}. También tiene mucho interés en aplicaciones espaciales por presentar beneficios en rendimiento para misiones cercanas al Sol y misiones en el espacio profundo porque los componentes más sensibles se encuentran resguardados de la dura radiación, siendo posible también el guardar energía en gravedad cero \cite{TPV_space_applications}. Otra área donde también existe mucho interés es en la recuperación de calor residual siendo menor del 40\% en las plantas de generación de energía de combustibles fósiles convencionales \cite{wasteHeat_TPV}.\\

Todas estas áreas de interés ha provocado un aumento de las investigaciones en los sistemas de generación termo-fotovoltaicos, estudiándose la utilización de células multi-estados \cite{MultiEstados_Capas_TPVs}, diferentes materiales de emisor para aumentar la potencia radiada, aplicación de capas finas en el emisor para el aumento de la potencia radiada\cite{doi:Near_field_ThinFilm}, aplicación de filtros \cite{multiLayerFilters} y capas reflectantes para la recuperación de fotones y disminución de calentamiento\cite{Thermoionic_nTPV_DATAS201910}, disminución del espacio entre emisor y célula para aprovechar los efectos de la radiación de campo cercano\cite{thermoionic_TPV_NF,modelEfficiency_NF_TPV,doi:NearField_200nm,nf_TPV_Pillars_SiO2}, y la combinación con un TEC para aumentar la densidad de potencia y eficiencia total del sistema generador\cite{thermoionic_TPV_NF,progress_Thermoionic_TPV,Thermoionic_nTPV_DATAS201910}.\\

La termo-fotovoltaica (TPV) se basa en convertir la radiación electromagnética proveniente de una fuente de calor en electricidad \cite{ThermophotovolticEnergyConversion_DATAS2021285}, teniendo la ventaja sobre la fotovoltaica solar de que el emisor no está en constante movimiento \cite{ThermophotovolticEnergyConversion_DATAS2021285}.\\



\section{Transmisión de calor}
El calor es una forma de energía que se propaga entre distintos medios de tres formas distintas, por convección, radiación y conducción.
\subsection{Convección}
La transmisión de calor por convección se produce por la conducción de la energía cuando el fluido entra en contacto con el sólido y luego el transporte de la energía mediante el movimiento del fluido.
\subsection{Radiación}

\subsubsection{Plank}
\subsubsection{Campo Cercano}
\subsection{Conducción}
La transmisión de calor por conducción se dá a través de uno o varios cuerpos, producido por la diferencia de temperatura entre las caras opuestas del conjunto. Para una dimensión la conducción térmica se modela como $P_{cond}={\bigtriangleup T}/{R} $, siendo $R$ la resistencia térmica del sistema.
Para un solo material, la resistencia térmica se modela como $R = l/{\left(A\cdot h\right)}$, donde $l$ es la longitud del material, $A$ es la superficie y $h$ es la conductividad térmica del material.\\

Para varios materiales colocados en serie, es decir, el flujo de calor que los atraviesa es el mismo para todos, la resistencia de conducción se define como la sumatoria de todas las resistencias de cada material($R=\sum R_i$).

Para el caso de transmisión de calor en serie existe un conjunto de resistencias que se producen por las imperfecciones en las interfaces de contacto entre los materiales, a dicha resistencia se le llama resistencia de contacto.


\subsubsection{Resistencia de contacto}
La resistencia de contacto en la interfaz entre dos conductores produce una caída de temperatura significante, como se observa en el \cite{noauthor_parallel-plate_nodate}, la cual es dependiente de muchos parámetros, tales como la temperatura, la presión, la rugosidad, etc.\\

Esta gran cantidad de dependencias hace que sea difícil parametrizar su valor, por lo tanto, se utilizan valores empíricos como el obtenido en \cite{noauthor_parallel-plate_nodate} de unos $4E{-6} \ m^2K/W$, pero también se puede llegar a parametrizar para ciertos casos, como en \textbf{Otra citación}, donde modelan el cálculo de la resistencia de contacto entre dos aceros 304L a temperatura ambiente como $Modelo de resistencia de contacto$.