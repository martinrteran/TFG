\chapter{Estado del arte}


En este capítulo...
\cite{noauthor_parallel-plate_nodate}
\cite{doi:MicroGapTPV}
\cite{doi:Thermoionic_Campbell}
\cite{doi:Near-field_ThinFilm}

\section{Introducción al capítulo}
En este capítulo se exponen los principales temas de interés para la realización de este trabajo, así como los artículos más representativos usados durante la realización de este trabajo.
%\section{Sector Energético}
%\section{Fotovoltaica}
\section{Termo-fotovoltaica}

\section{Transmisión de calor}
El calor es una forma de energía que se representa como parte de la energía cinética que presentan los electrones de los átomos. Esta energía se puede transmitir a otros objetos de dos maneras, por radiación o por conducción.
\subsection{Transmisión por Radiación}

\subsubsection{Plank}
\subsubsection{Campo Cercano}
\subsection{Conducción}
La transmisión de calor por conducción se dá a través de uno o varios cuerpos, producido por la diferencia de temperatura entre las caras opuestas del conjunto. Para una dimensión la conducción térmica se modela como $P_{cond}={\bigtriangleup T}/{R} $, siendo $R$ la resistencia térmica del sistema.
Para un solo material, la resistencia térmica se modela como $R = l/{\left(A\cdot h\right)}$, donde $l$ es la longitud del material, $A$ es la superficie y $h$ es la conductividad térmica del material.\\

Para varios materiales colocados en serie, es decir, el flujo de calor que los atraviesa es el mismo para todos, la resistencia de conducción se define como la sumatoria de todas las resistencias de cada material($R=\sum R_i$).

Para el caso de transmisión de calor en serie existe un conjunto de resistencias que se producen por las imperfecciones en las interfaces de contacto entre los materiales, a dicha resistencia se le llama resistencia de contacto.


\subsubsection{Resistencia de contacto}
La resistencia de contacto en la interfaz entre dos conductores produce una caída de temperatura significante, como se observa en el \textbf{aquí iría la cita}, la cual es dependiente de muchos parámetros, tales como la temperatura, la presión, la rugosidad, etc.

Esta gran cantidad de dependencias hace que sea difícil parametrizar su valor, por lo tanto, se utilizan valores empíricos como el obtenido en \textbf{Otra cita} de unos $4E{-6} \ m^2K/W$, pero también se puede llegar a parametrizar para ciertos casos, como en \textbf{Otra citación}, donde modelan el cálculo de la resistencia de contacto entre dos aceros 304L a temperatura ambiente como $Modelo de resistencia de contacto$.

\section{Investigaciones más importantes}

