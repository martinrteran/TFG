%chapter introduce un nuevo capítulo
\chapter{Resumen}

La termofotovoltaica (TPV) es un campo de interés por sus aplicaciones en sistemas de recuperación de calor y su uso en sistemas de almacenamiento de calor latente. Un área de interés es el aumento de su potencia de salida, que se consigue al disminuir la distancia entre el emisor y la célula a unos cientos de nanómetros porque aumenta considerablemente la transmisión de calor por radiación, obteniéndose potencias superiores a las proporcionadas por la ley de Planck, este efecto se le conoce como ``efecto de campo cercano''. Los sistemas TPV que aprovechan el efecto de la radiación de campo cercano, conocidos como nTPV, presentan el problema de mantener una distancia de separación constante, por lo que se utilizan nano-espaciadores.\\\\
%%% SEGUNDO
En este trabajo se simulan la transmisión de calor por conducción y por radiación de campo cercano para varios sistemas nTPV de emisores de silicio, acero inoxidable y carburo de silicio a 800\textdegree C con un nano-espaciador de dióxido de silicio y unas células de germanio a 25\textdegree C para un rango de alturas de nano-espaciador de 100 nm a 1000 nm, con y sin resistencia de contacto entre el emisor y el nano-espaciador. Se obtienen para cada caso la relación entre la potencia transmitida por radiación y potencia transmitida por conducción en un centímetro cuadrado para distintas densidades de nano-espaciadores y distintas alturas de estos, para poder determinar la viabilidad de cada sistema.\\\\
También se estudia la viabilidad de un sistema nTPV con célula fotovoltaica de silicio y emisor de silicio para diferentes grados de porosidad del dióxido de silicio del nano-espaciador y para el caso de la nTPV con resistencia de contacto entre emisor y nano-espaciadores.\\\\
Por último, se calculan las densidades de nano-espaciadores en un centímetro cuadrado de emisor y célula para soportar la carga del emisor y una presión de una atmósfera. %Y se simula un caso de una nTPV con nano-espaciador de silicio y resistencia de contacto, obteniendo la importancia de la elección de los materiales para los nano-espaciadores.

%Se simulan las transmisión de calor por conducción para los casos de nTPVs de nano-espaciadores de silicio y células de germanio para los diferentes materiales de emisor ya mencionados, con y sin resistencia de contacto, obteniendo la importancia de la elección de los materiales para los nano-espaciadores.

\paragraph{Palabras clave:} Termofotovoltaica, radiación de campo cercano, recuperación de calor.
\begin{otherlanguage}{british}
\chapter{Abstract}

%% TRADUCCION GOOGLE, REVISADO MANUAL Y CON GRAMMARLY
Thermophotovoltaics (TPV) is a field of interest due to its applications in heat recovery systems and its use in latent heat storage systems. An area of interest is the increase of its output power, which is achieved by reducing the distance between the emitter and the cell to a few hundred nanometers because it vastly increases the heat transfer by radiation, obtaining powers higher than those provided by Planck's law, this effect is known as ``near field effect''. The TPV systems that take advantage of the effect of near-field radiation, known as nTPV, have the problem of keeping a constant separation distance, which is why nano-spacers are used.\\\\
%
In this work, the heat transfer by conduction and near-field radiation is simulated for several nTPVs systems with emitters made of silicon, stainless steel and silicon carbide at 800\textdegree C with a silicon dioxide nano-spacer and germanium cells at 25\textdegree C, with and without contact resistances between the emitter and the nano-spacers for a range of the nano-spacers heights from 100 nm to 1000 nm. For each case, the ratio between the power transferred by radiation and the one transferred by conduction in a square centimeter is obtained for different densities of nano-spacers and different heights of these, in order to determine the feasibility of each system.\\\\
%
The feasibility of an nTPV system with silicon photovoltaic cell and silicon emitter is also studied for different degrees of porosity of silicon dioxide of the nano-spacer and the case of an nTPV with contact resistance between emitter and nano-spacers.\\\\
%
Finally, the densities of nano-spacers in a square centimetre of emitter and cell to support the load of the emitter and the pressure of one atmosphere are calculated. %And a case of an nTPV with a silicon nano-spacer and contact resistance is simulated, obtaining the importance of the selection of the materials for the nano-spacers.

\paragraph{Keywords:} Thermophotovoltaic, near-field radiation, heat recovery.
\end{otherlanguage}