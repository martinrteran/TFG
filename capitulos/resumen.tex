%chapter introduce un nuevo capítulo
\chapter{Resumen}

La termo-fotovoltaica (TPV) es un campo de interés por sus aplicaciones en sistemas de recuperación de calor y su uso en sistemas de almacenamiento de energía en forma de calor latente, consiguiéndose aumentar la potencia de salida al disminuir la distancia de separación entre el emisor y la célula a unos cientos de nanómetros por el aumento de la transmisión de calor por radiación, siendo las potencias superiores a las de un cuerpo negro según la ley de Planck, este efecto se le conoce como ``efecto de campo cercano''. Estos sistemas TPV que aprovechan el efecto de la radiación de campo cercano se les conocen como nTPV y su mayor problema es el mantenimiento constante de la distancia de separación, por lo que se utilizan nano-espaciadores que producen pérdidas por conducción.\\\\
%%% SEGUNDO
En este trabajo se simulan la transmisión de calor por conducción y por radiación de campo cercano para varios sistemas nTPVs de emisores de silicio, acero inoxidable y carburo de silicio a 800\textdegree C con un nano-espaciador de dióxido de silicio y unas células principalmente de germanio a 25\textdegree C, con y sin resistencias de contacto entre el emisor y los nano-espaciadores para un rango de alturas de nano-espaciadores de 100nm a 1000nm. Se obtienen las densidades de nano-espaciadores en un centímetro cuadrado de nTPV para las relaciones de las potencias mayores a un orden de magnitud de cada caso de cada sistema para determinar su viabilidad.\\\\
También se estudia la viabilidad de un sistema nTPV con célula fotovoltaica de silicio y emisor de silicio para diferentes grados de porosidad del dióxido de silicio en el nano-espaciador y para el caso de la nTPV con resistencia de contacto entre emisor y nano-espaciadores.\\\\
Por último, se calculan las densidades de nano-espaciadores en un centímetro cuadrado de emisor y célula para soportar la carga del emisor y una presión de una atmósfera. Se simulan las transmisión de calor por conducción para los casos de nTPVs de nano-espaciadores de silicio y células de germanio para los diferentes materiales de emisor ya mencionados, con y sin resistencia de contacto, obteniendo la importancia de la elección de los materiales para los nano-espaciadores.

\paragraph{Palabras clave:} Termo-fotovoltaica, radiación de campo cercano, recuperación de calor.
\begin{otherlanguage}{british}
\chapter{Abstract}

%% TRADUCCION GOOGLE, REVISADO MANUAL Y CON GRAMMARLY
Thermophotovoltaics (TPV) is a field of interest due to its applications in heat recovery systems and its use in energy storage systems in the form of latent heat, achieving an increase in output power by reducing the separation distance between the emitter and the cell at a few hundred nanometres due to the increase in heat transmission by radiation, being the powers higher than those of a black body according to Planck's law, this effect is known as ``near field effect''. These TPV systems that take advantage of the near field radiation effect are known as nTPV and their biggest problem is the constant maintenance of the separation distance, which is why nano-spacers are used that produce conduction losses.\\\\
In this work, the heat transmission by conduction and near-field radiation is simulated for several nTPVs systems with emitters made of silicon, stainless steel and silicon carbide at 800\textdegree C with a silicon dioxide nano-spacer and cells mainly made of germanium at 25\textdegree C, with and without contact resistances between the emitter and the nano-spacers for a range of the nano-spacers heights from 100nm to 1000nm. Nano-spacer densities in one square centimetre of nTPV are obtained for power ratios greater than one order of magnitude for each case of each system to determine their feasibility.\\\\
The feasibility of an nTPV system with silicon photovoltaic cell and silicon emitter is also studied for different degrees of porosity of silicon dioxide of the nano-spacer and the case of an nTPV with contact resistance between emitter and nano-spacers.\\\\
Finally, the densities of nano-spacers in a square centimetre of emitter and cell to support the load of the emitter and the pressure of one atmosphere are calculated. Heat transmission by conduction is simulated for the cases of nTPVs of silicon nano-spacers and germanium cells for the same emitter materials already mentioned, with and without contact resistance, obtaining the importance of the selection of the materials for the nano-spacers.

\paragraph{Keywords:} Thermophotovoltaic, near-field radiation, heat recovery.
\end{otherlanguage}