\chapter{Introducción}

En este capítulo no deben faltar los siguientes apartados:

\section{Motivación del proyecto}

\subsection{Crisis Energética}

\subsection{Energías Renovables}

\subsection{Crisis de suministro de baterías}

\subsection{AMADEUS Project}


\section{Objetivos}
El objetivo de este estudio es diseñar y simular pilares de dimensiones nanométricas para el aprovechamiento del efecto de campo cercano en un convertidor termo-fotovoltaico, así como determinar la viabilidad de este sistema. 
\begin{itemize}
	\item Obtención de la escala del modelo.
	\item Diseño del modelo y obtención de las propiedades de los materiales.
	\item Análisis de la influencia de la altura de los nano-espaciadores en la potencia de transmisión.
	\item Análisis de la influencia de la porosidad y resistencia de contacto en el flujo de calor por conducción.
	\item Determinar para que casos el sistema es viable. 
	\item Comparación de los casos...
\end{itemize}


\section{Estructura del documento}

A continuación y para facilitar la lectura del documento, se detalla el contenido de cada capítulo.

\begin{itemize}
\item En el \textbf{capítulo 1} se realiza una introducción del trabajo con la respectiva motivación y objetivos.
\item En el \textbf{capítulo 2} se desarrolla el estado de arte, definiendo los apartados más importantes y resaltando las investigaciones con mayor relevancia.
\item En el \textbf{capítulo 3} se exponen las herramientas y materiales utilizados.
\item En el \textbf{capítulo 4} se mencionan los métodos seguidos y los cálculos realizados para el desarrollo del trabajo.
\item En el \textbf{capítulo 5} se exponen los resultados obtenidos de las simulaciones.
\item En el \textbf{capítulo 6} se desarrolla la conclusión y se realiza planteamientos para futuros trabajos.
\end{itemize}
