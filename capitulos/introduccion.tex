\chapter{Introducción}

En este capítulo no deben faltar los siguientes apartados:

\section{Motivación del proyecto}

En los últimos años el consumo de energía primaria en España ha aumentado significativamente de unos 88455 ktep($\sim1.03E12 \ kWh$) en 1990 a unos 126107 ktep($\sim 1.43E12 \ kWh$) en 2019, que equivale a un aumento del 42\% respecto a su valor en 1990, llegando a ser su valor máximo unos 146891 ktep($\sim 1.71E12 \ kWh$) en 2007.\\\\
De 2008 a 2014, como consecuencia de la crisis económica el consumo energético disminuyó hasta unos 117824 ktep ($\sim 1.37E12 \ kWh$) en 2014, recuperándose a partir del 2015, como se puede observar en la figura \ref{fig:demandaenergiaprimariaproporcion1990}. \\
\begin{figure}[H]
	\centering
	\includegraphics[width=10cm, height=7cm]{figuras/DemandaEnergiaPrimariaProporcion1990}
	\caption{Representación gráfica de la relación entre la demanda total de energía primaria anual de 1990 hasta 2019 en España. \textit{Fuente de los datos utilizados: Ministerio para la transición ecológica y el reto demográfico de España.} }
	\label{fig:demandaenergiaprimariaproporcion1990}
\end{figure}
Dicha energía se puede dividir en diferentes categorías según el sector que la consume o la fuente de dicha energía. En el 2019, un 44.5\% de la energía primaria fue consumida de productos petrolíferos (figura ) y la industria consumió un 23.6\% de la energía final consumida en España (figura ), representando casi una cuarta parte del consumo total de energía final.
\begin{minipage}[left]{0.4\linewidth}
	
\end{minipage}
\begin{minipage}[right]{0.4\linewidth}
	
\end{minipage}

  año en el cual la ONU (Organización de las Naciones Unidas) aprobó la \textit{Agenda 2030 sobre el Desarrollo Sostenible}, la cual cuenta con 17 objetivos de desarrollo sostenible (ODS)

\textbf{De energías por fuente->energías por sector-> Artículo de internet -> TPVs->Motivación del proyecto}

\begin{figure}[H]
	\centering
	\includegraphics[width=10cm, height=7cm]{figuras/ProdcVSdemAnual}
	\caption[Producción vs demanda de energía eléctrica anual]{Comparación de la producción vs la demanda de energía eléctrica anual de España entre los años 2011 y 2021. \textit{Fuente: red eléctrica de España.} }
	\label{fig:prodcvsdemanual}
\end{figure}

\begin{figure}[h]

\end{figure}

Objetivos del desarrollo sostenible

{Aprovechamiento de energía}
% Aquí irá el gran cantidad de energía que se desperdicia en las fábricas, refinerías, etc.
% Así como se puede aprovechar mediante la termo-fotovoltaíca.



\section{Objetivos}
El objetivo de este estudio es diseñar y simular pilares de dimensiones nanométricas para el aprovechamiento del efecto de campo cercano en un convertidor termo-fotovoltaico, así como determinar la viabilidad de este sistema. 
\begin{itemize}
	\item Modelar un nano-espaciador dentro de los rangos permitidos de los parámetros de los programas de simulación y modelado 3D.
	\item Modelar el emisor y la célula del sistema TPV para que los gradientes térmicos por conducción no lleguen a los bordes.
	\item Simular la transferencia de calor por conducción a través de un nano-espaciador de $SiO_2$ de un sistema TPV para diferentes alturas del nano-espaciador, diferentes materiales de emisor y diferentes resistencias de contacto entre emisor y el nano-espaciador.
	\item Simular la transferencia de calor radiada entre el emisor y la célula para diferentes materiales del emisor y para varias distancias de separación, teniendo en cuenta los efectos de campo cercano en la radiación.
	\item Determinar entre los casos estudiados cuales pueden dar lugar a sistemas de TPV de campo cercano viables.
\end{itemize}


\section{Estructura del documento}

A continuación y para facilitar la lectura del documento, se detalla el contenido de cada capítulo.

\begin{itemize}
\item En el \textbf{capítulo 1} se realiza una introducción del trabajo con la respectiva motivación y objetivos.
\item En el \textbf{capítulo 2} se desarrolla el estado de arte, definiendo los apartados más importantes y resaltando las investigaciones con mayor relevancia.
\item En el \textbf{capítulo 3} se exponen las herramientas y materiales utilizados, así como los criterios de selección.
\item En el \textbf{capítulo 4} se mencionan los métodos seguidos y los cálculos realizados para el desarrollo del trabajo.
\item En el \textbf{capítulo 5} se exponen los resultados obtenidos de las simulaciones.
\item En el \textbf{capítulo 6} se desarrolla la conclusión y se realiza planteamientos para futuros trabajos.
\end{itemize}
