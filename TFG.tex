% Plantilla realizada por Alberto Brunete (UPM).

%parametros de tipo libro
\documentclass[10pt,a4paper]{book}

%idioma español y acentos
\usepackage[british,spanish, es-tabla]{babel}
%\usepackage[latin1]{inputenc}
\usepackage[utf8]{inputenc}
\usepackage{float}
%algunos sÌmbolos matem·ticos y paquete para usar subim·genes
\usepackage{amsmath}
\usepackage{amsfonts}
\usepackage{amssymb}
\usepackage{graphicx}
\usepackage{pgfplots}
\usepackage{caption}
\usepackage{subcaption}
%M·rgenes
\usepackage[left=3cm,top=3cm,right=3cm,bottom=3cm]{geometry}

%
\usepackage{multicol}
\usepackage{cite}
\usepackage{epstopdf}
\usepackage{array}
\usepackage{multirow}
\usepackage[acronym,toc,section=chapter,
						nonumberlist,nopostdot,
						nomain,symbols,
						nogroupskip]{glossaries}
\usepackage{siunitx}
\usepackage{tabu}
%para generar índice con hipervínculos
\usepackage{hyperref}

%parametros del documento (sus propiedades)
\hypersetup{
    pdftitle={Martin Augusto Reigadas Teran - TFG - 2022},
    pdfsubject={TFG - 2022},
    pdfauthor={Martin Augusto Reigadas Teran},
    pdfkeywords={palabraclave1} {palabraclave2} {palabraclave3},
    colorlinks,
    citecolor=black,
    filecolor=black,
    linkcolor=black,
    urlcolor=black,
}

%%%%%%%%%%%%%%%%%%%%%%%%%%%%%%%%%%%%
% Redefine package options         %
%%%%%%%%%%%%%%%%%%%%%%%%%%%%%%%%%%%%
\renewcommand{\glossarysection}[2][]{}
%\renewcommand*{\glspostdescription}{\dotfill}

%%%%%%%%%%%%%%%%%%%%%%%%%%%%%%%%%%%%
% Own styles                       %
%%%%%%%%%%%%%%%%%%%%%%%%%%%%%%%%%%%%

% -----------------
% Acronym-styles
% -----------------

\newglossary[slg]{symbolslist}{syi}{syg}{Symbolslist} % create add. symbolslist
\glsaddkey{unit}{\glsentrytext{\glslabel}}{\glsentryunit}{\GLsentryunit}{\glsunit}{\Glsunit}{\GLSunit}
%---------------------------
%  Symbols with Units Style
%----------------------------

\newglossarystyle{symbunitlong}{%
\setglossarystyle{long3col}% base this style on the list style
\renewenvironment{theglossary}{% Change the table type --> 3 columns
  \begin{longtable}{lp{0.6\glsdescwidth}>{\centering\arraybackslash}p{2cm}}}%
  {\end{longtable}}%
%
\renewcommand*{\glossaryheader}{%  Change the table header
  \bfseries S\'{i}mbolo & \bfseries Descripci\'{o}n & \bfseries Unidades \\
  \hline
  \endhead}
\renewcommand*{\glossentry}[2]{%  Change the displayed items
\glstarget{##1}{\glossentryname{##1}} %
& \glossentrydesc{##1}% Description
& \glsunit{##1}  \tabularnewline
}
}
% -----------------
% Symbols-styles
% -----------------

\newglossarystyle{mysymbolstyle}{%
  %\renewcommand{\glossarysection}[2][]{}% no title
  \renewcommand*{\glsclearpage}{}% avoid page break before glossary
  \renewenvironment{theglossary}%
    {\begin{longtabu} to \linewidth {clXc}}%
    {\end{longtabu}}%
  % Header line
  \renewcommand*{\glossaryheader}{%
    % Requires booktabs
    %\toprule%
    \textbf{Symbol} & \textbf{Name} & \textbf{Description} & \textbf{Unit}%
    \tabularnewline%
    \tabularnewline%
    %\midrule%
    \endhead%
    %\bottomrule%
    \endfoot%
  }%
  % indicate what to do at the start of each logical group
  \renewcommand*{\glsgroupskip}{0pt}% What to do between groups
  \renewcommand*{\glossentry}[1]{%
    \glsentryitem{##1}% Entry number if required
    \glstarget{##1}{\glossentrysymbol{##1}} &
    %\glossentrysymbol{##1} & % Symbol
    \glossentryname{##1}    & % Name
    \glossentrydesc{##1}    & % Description
    \glsentryuseri{##1}%      % Unit in User1-Variable
    \tabularnewline%
  }%
}



\sloppy
\epstopdfsetup{outdir=./pdf}
%\graphicspath{ {./graficasCasos_images/} }

%%%%%%%%%%%%%%%%%%%%%%%%%%%%%%%%%%%%
%         Symbols                  %
%%%%%%%%%%%%%%%%%%%%%%%%%%%%%%%%%%%%

%------------------
% Category 1
%------------------

%%%%%%%%%%%%%%%%%%%%%%%%%%%%%
%           METODOS         %
%%%%%%%%%%%%%%%%%%%%%%%%%%%%%
\newglossaryentry{hrad}{
name= $h_{rad}$,
unit=$W/m^2K$,
description={Coeficiente de transferencia de calor por radiaci\'{o}n},
	type=symbolslist
}
\newglossaryentry{wres}{
name=$\omega_{res}$,
unit= \si{rad/s},
description={Frecuencia de resonancia},
type=symbolslist
}
\newglossaryentry{hc}{
name=$h_c$,
description={Conductancia t\'{e}rmica de contacto},
unit=$W/m^2K$,
type=symbolslist
}
\newglossaryentry{densidad}{
name=$\rho$,
description={Densidad},
unit=$kg/m^3$,
type=symbolslist
}
\newglossaryentry{masa}{
name=\textit{M},
description={Masa},
unit=$kg$,
type=symbolslist
}
\newglossaryentry{condTerm}{
name=$k_i$,
description={Conductividad t\'{e}rmica del material i},
unit=$W/m\cdot k$,
type=symbolslist
}
\newglossaryentry{flujoCalor}{
name=\textit{Q},
description={Potencia calor\'{i}fica transmitida por conducci\'{o}n},
unit=$W$,
type=symbolslist
}
\newglossaryentry{deltaT}{
name=$\Delta T$,
description={Diferencia de temperatura},
unit=$K$,
type=symbolslist
}
\newglossaryentry{qw}{
name=$q_w$,
description={Potencia radiada espectral},
unit=$Wm^{-2}(rad/s)^{-1}$,
type=symbolslist
}
%%%%%%%%%%%%%%%%%%%%%%%%%%%%%%%%%%%%%%%%%%%%%
%         S�mbolos de los materiales        %
%%%%%%%%%%%%%%%%%%%%%%%%%%%%%%%%%%%%%%%%%%%%%
\newglossaryentry{sio2}{
name=$SiO_2$,
description={Di\'{o}xido de silicio},
unit= ,
type=	symbolslist}

\newglossaryentry{sic}{
name=\textit{SiC},
description={Carburo de silicio},
unit= ,
type=symbolslist}

\newglossaryentry{si}{
name=\textit{Si},
description={Silicio},
unit= ,
type=symbolslist}

\newglossaryentry{ge}{
name=$Ge$,
description={Germanio},
unit= ,
type=symbolslist} 


\newglossaryentry{beo}{
name=$BeO$,
unit= ,
description={\'{O}xido de berilio},
type=symbolslist} 

%%%%%%%%%%%%%%%%%%%%%%%%%%%%%%%%%%
%             �ptica             %
%%%%%%%%%%%%%%%%%%%%%%%%%%%%%%%%%%

\newglossaryentry{n}{
name=\textit{n},
description={\'{I}ndice de refracci\'{o}n},
unit= ,
type=symbolslist} 

\newglossaryentry{k}{
name=\textit{k},
description={Coeficiente de extinsi\'{o}n},
unit= ,
type=symbolslist} 

%%%%%%%%%%%%%%%%%%%%%%%%%%%%%%%%%%%%%
%         R_c Calculada             %
%%%%%%%%%%%%%%%%%%%%%%%%%%%%%%%%%%%%%

\newglossaryentry{sigma}{
name=$\sigma$,
description={Combinaci\'{o}n RMS de la rugosidad $\sigma=\sqrt{\sigma_1^2+\sigma_2^2}$},
unit= falta,
type=symbolslist} 

\newglossaryentry{p}{
name=\textit{P},
description={Presi\'{o}n de contacto},
unit= Pa,
type=symbolslist} 

\newglossaryentry{ks}{
name=$K_s$,
description={Media arm\'{o}nica de la conductividad t\'{e}rmica $k_s=2\cdot {k_1\cdot k_2}/{\left(k_1+k_2\right)}$},
unit= falta,
type=symbolslist}

\newglossaryentry{mrugosidad}{
name=\textit{m},
description={Combinaci\'{o}n RMS de la media absoluta de la pendiente de la rugosidad $m=\sqrt{m_1^2+m_2^2}$},
unit= falta,
type=symbolslist}

\newglossaryentry{presion}{
name=\textit{P},
description={Presi\'{o}n},
unit=$Pa$,
type=symbolslist
}
\newglossaryentry{Hc}{
name=$H_c$,
description={Dureza Vickers del material m\'{a}s duro},
unit=$VH$,
type=symbolslist
}
%\makenoidxglossaries 



%%%%%%%%%%%%%%%%%%%%%%%%%%%%%%%%%%%%%%%%%%
%                ACRONYMS
%%%%%%%%%%%%%%%%%%%%%%%%%%%%%%%%%%%%%%%%%%
\newacronym{teg}{TEG}{Generador Termoel\'{e}ctrico}
\newacronym{tic}{TIC}{Termoi\'{o}nico}
\newacronym{pv}{PV}{fotovoltaico}
\newacronym{tpv}{TPV}{Termo-fotovoltaico}
\newacronym{ntpv}{nTPV}{Termo-fotovoltaico de campo cercano}
\newacronym{nftpv}{NFTPV}{Termo-fotovoltaico de campo cercano}
\newacronym{nf-tpv}{NF-TPV}{Termo-fotovoltaico de campo cercano}
\newacronym{ictpv}{ICTPV}{C\'{e}lulas termofotovoltaicas interbandas en cascada}
\newacronym{nitpv}{nTiPV}{Dispositivo termoi\'{o}nico-fotovoltaico de campo cercano}
\newacronym{voc}{$V_{OC}$}{tensi\'{o}n de circuito abierto}
\newacronym{bsr}{BSR}{Capa trasera reflectante}
\newacronym{rc}{$R_c$}{Resistencia de contacto}
\newacronym{rcemp}{$R_{cEmp}$}{Resistencia de contacto emp\'{i}rica}
\newacronym{rcCal}{$R_{cCal}$}{Resistencia de contacto calculada}
\newacronym{rcCalMax}{$R_{cCalMax}$}{Resistencia de contacto calculada m\'{a}xima}
\newacronym{rccalInt}{$R_{cCalInt}$}{Resistencia de contacto calculada intermedia}
\newacronym{ss}{SS}{Acero inoxidable}
\newacronym{bg}{BG}{ancho de banda}
\newacronym{ies}{IES}{Instituto de Energ\'{i}a Solar}
\newacronym{tfg}{TFG}{Trabajo de Fin de Grado}
%\newacronym{hrad}{$h_{rad}$}{Coefficiente de transferencia de calor por radiaci\'{o}n}
%\newacronym{wres}{$\omega_{res}$}{frecuencia de resonancia}
%\makenoidxglossaries
%empieza el documento
\makeglossaries 
%%%%%%%%%%%%%%%%%%%%%%%%%%%%%%%%%%%%%%%%%
%       Frases PreConstruidas          
%%%%%%%%%%%%%%%%%%%%%%%%%%%%%%%%%%%%%%%%
\newcommand\sourceSpectralRadiation{%
	\itshape Fuente de la imagen: Calculadora de campo cercano, descrita en la secci\'{o}n \ref{sec:calc_campo_cercano}
}

\begin{document}  

%elementos antes del trabajo en sÌ se meten dentro de frontmatter
\frontmatter

%cada incluye referencia a un archivo de tipo .tex
\begin{titlepage}
\begin{center}
%forma de introducir imágenes. el \\[0.5 cm] de final de línea introduce un salto de ese tamaño.
%width=1\textwidth indica el tamaño de la imágen (valores entre 0-1). 
\includegraphics[width=1\textwidth]{figuras/cabecera.png}  \\[0.5 cm]

\LARGE UNIVERSIDAD POLITÉCNICA DE MADRID \\ [1 cm]

\LARGE ESCUELA TÉCNICA SUPERIOR DE INGENIERÍA Y DISEÑO INDUSTRIAL \\ [1 cm]

\LARGE Grado en Ingeniería Electrónica y Automática Industrial\\ [1 cm]

\LARGE \textbf{TRABAJO FIN DE GRADO}\\[0.75 cm]

\Huge \textsc{Modelado y simulación de espaciadores nanométricos para su aplicación en dispositivos TPVs de campo cercano}\\[1 cm]

\LARGE Martin Augusto Reigadas Teran \\[0.5 cm]
%flushleft alinea a la izquierda el texto
\begin{flushleft}
\Large
\emph{Tutor:} {Pablo García-Linares Fontes}\\
Departamento de Ingeniería Eléctrica, Electrónica, Automática y Física Aplicada\\
\emph{Cotutora:} Esther López Estrada\\
Instituto de Energía Solar\\
\end{flushleft}

%rellena de blanco el resto de la página para escribir abajo del todo
\vfill

% Bottom of the page
{\large Madrid, Septiembre, 2022}
%SE ponen al final firmas.tex
%\end{center}
%\end{titlepage}
\cleardoublepage 
\end{center}
\end{titlepage}


%Licencia opcional

\cleardoublepage

\begin{flushleft} \large
\textbf{TÌtulo:} Modelado de espaciadores nanométricos para un convertidor termo-fotovoltaico \\
\textbf{Autor:} Martin Augusto Reigadas Teran\\
\textbf{Tutor:} Pablo García-Linares Fontes \\ 
\textbf{Cotutor:} Esther López Estrada y Alejandro Datas\\ [1 cm]

\end{flushleft} 

\begin{center} \LARGE
EL TRIBUNAL \\ [1 cm]
\end{center}

\begin{flushleft} \LARGE
Presidente: \\ [1 cm]
Vocal: \\ [1 cm]
Secretario: \\ [1.5 cm]
\end{flushleft}

\large
Realizado el acto de defensa y lectura del Trabajo Fin de Grado el día ... de ....................   de ... en .........., en la Escuela Técnica Superior de Ingeniería y Diseño Industrial de la Universidad Politécnica de Madrid, acuerda otorgarle la CALIFICACIÓN de: \\ [2 cm]

\begin{center}
 \large VOCAL \\ [2.2 cm]
\end{center}

\begin{minipage}{0.5\textwidth}
 \begin{flushleft}
 \large SECRETARIO
\end{flushleft}
\end{minipage}
\begin{minipage}{0.5\textwidth}
\begin{flushright}
 \large PRESIDENTE
\end{flushright} 
\end{minipage}

\chapter{Agradecimientos}

Agradezco a mi tutor Pablo por darme la oportunidad de realizar un trabajo dentro de las líneas de investigación de la termo-fotovoltaica y contribuir a esta misma, permitirme explorar por mi cuenta y experimentar como la investigación llega a ser.\\\\
A mi cotutora Esther por brindarme tanto apoyo durante la realización del trabajo, explicándome conceptos en mayor detalle y permitirme ir más allá con el trabajo. También agradezco a Alejandro Datas por brindarme apoyo para no salirme de los objetivos de este trabajo.\\\\
Gracias a los tres por permitirme participar en la elaboración del póster presentado en la treceava conferencia mundial en la generación de electricidad termo-fotovoltaica (\href{https://www.tpv-13.org/}{TPV-13}) de Abril del 2022, compartida con la dieciochoava conferencia internacional de sistemas fotovoltaicos de concentración (\href{https://www.cpv-18.org/}{CPV-18}).\\\\  
A mi padre por siempre apoyarme durante la carrera, y a mi hermana Andrea por brindarme apoyo cuando más lo necesitaba y por siempre estar ahí encargándote de muchas cosas.\\\\
A mis amigos, Andrea, ZhanYao y Tesa con los cuales he pasado estos duros cuatro años juntos estudiando, trasnochando realizando los trabajos de las asignaturas, viviendo nuevas experiencias y aprendiendo bastante cosas más allá de la ingeniería.\\\\
A Yvonne por empujarme a salir de mi zona de confort, hacer lo que hay que hacer cuando se debía y aprender a como seguir con la vida.\\\\
A Noelia por estar ahí gran parte del tiempo de realización de este trabajo, apoyándome en cada fase y recordándome de tomar descansos.\\\\
A la Asociación del Cubo por permitirme conocer a gente tan bonita como Rafael, que siempre ha brindado su apoyo y llegando a ser de gran ayuda para la realización de este trabajo. Al CREA por permitirme conocer a más gente que le encanta la electrónica y pasar unas buenos ratos con los proyectos.\\\\

%chapter introduce un nuevo capítulo
\chapter{Resumen}

La termo-fotovoltaica (TPV) es un campo de interés por sus aplicaciones en sistemas de recuperación de calor y su uso en sistemas de almacenamiento de energía en forma de calor latente, consiguiéndose aumentar la potencia de salida al disminuir la distancia de separación entre el emisor y la célula a unos cientos de nanómetros por el aumento de la transmisión de calor por radiación, siendo las potencias superiores a las de un cuerpo negro según la ley de Planck, este efecto se le conoce como ``efecto de campo cercano''. Estos sistemas TPV que aprovechan el efecto de la radiación de campo cercano se les conocen como nTPV y su mayor problema es el mantenimiento constante de la distancia de separación, por lo que se utilizan nano-espaciadores que producen pérdidas por conducción.\\\\
%%% SEGUNDO
En este trabajo se simulan la transmisión de calor por conducción y por radiación de campo cercano para varios sistemas nTPVs de emisores de silicio, acero inoxidable y carburo de silicio a 800\textdegree C con un nano-espaciador de dióxido de silicio y unas células principalmente de germanio a 25\textdegree C, con y sin resistencias de contacto entre el emisor y los nano-espaciadores para un rango de alturas de nano-espaciadores de 100nm a 1000nm. Se obtienen las densidades de nano-espaciadores en un centímetro cuadrado de nTPV para las relaciones de las potencias mayores a un orden de magnitud de cada caso de cada sistema para determinar su viabilidad.\\\\
También se estudia la viabilidad de un sistema nTPV con célula fotovoltaica de silicio y emisor de silicio para diferentes grados de porosidad del dióxido de silicio en el nano-espaciador y para el caso de la nTPV con resistencia de contacto entre emisor y nano-espaciadores.\\\\
Por último, se calculan las densidades de nano-espaciadores en un centímetro cuadrado de emisor y célula para soportar la carga del emisor y una presión de una atmósfera. Se simulan las transmisión de calor por conducción para los casos de nTPVs de nano-espaciadores de silicio y células de germanio para los diferentes materiales de emisor ya mencionados, con y sin resistencia de contacto, obteniendo la importancia de la elección de los materiales para los nano-espaciadores.

\paragraph{Palabras clave:} Termo-fotovoltaica, radiación de campo cercano, recuperación de calor.
\begin{otherlanguage}{british}
\chapter{Abstract}

%% TRADUCCION GOOGLE, REVISADO MANUAL Y CON GRAMMARLY
Thermophotovoltaics (TPV) is a field of interest due to its applications in heat recovery systems and its use in energy storage systems in the form of latent heat, achieving an increase in output power by reducing the separation distance between the emitter and the cell at a few hundred nanometres due to the increase in heat transmission by radiation, being the powers higher than those of a black body according to Planck's law, this effect is known as ``near field effect''. These TPV systems that take advantage of the near field radiation effect are known as nTPV and their biggest problem is the constant maintenance of the separation distance, which is why nano-spacers are used that produce conduction losses.\\\\
In this work, the heat transmission by conduction and near-field radiation is simulated for several nTPVs systems with emitters made of silicon, stainless steel and silicon carbide at 800\textdegree C with a silicon dioxide nano-spacer and cells mainly made of germanium at 25\textdegree C, with and without contact resistances between the emitter and the nano-spacers for a range of the nano-spacers heights from 100nm to 1000nm. Nano-spacer densities in one square centimetre of nTPV are obtained for power ratios greater than one order of magnitude for each case of each system to determine their feasibility.\\\\
The feasibility of an nTPV system with silicon photovoltaic cell and silicon emitter is also studied for different degrees of porosity of silicon dioxide of the nano-spacer and the case of an nTPV with contact resistance between emitter and nano-spacers.\\\\
Finally, the densities of nano-spacers in a square centimetre of emitter and cell to support the load of the emitter and the pressure of one atmosphere are calculated. Heat transmission by conduction is simulated for the cases of nTPVs of silicon nano-spacers and germanium cells for the same emitter materials already mentioned, with and without contact resistance, obtaining the importance of the selection of the materials for the nano-spacers.

\paragraph{Keywords:} Thermophotovoltaic, near-field radiation, heat recovery.
\end{otherlanguage}

%genera índice
\tableofcontents
\addcontentsline{toc}{chapter}{Índice}

%Índice de figuras.
\listoffigures

%Índice de tablas.
\listoftables

%Empieza la parte descriptiva del trabajo
\mainmatter

\chapter{Introducción}


\section{Motivación del proyecto}

En los últimos años el consumo de energía primaria en España ha aumentado significativamente de unos 88455 ktep ($\sim1.03E12 \ kWh$) en 1990 a unos 126107 ktep ($\sim 1.43E12 \ kWh$) en 2019, que equivale a un aumento del 42\% respecto a su valor en 1990, llegando a ser su valor máximo unos 146891 ktep ($\sim 1.71E12 \ kWh$) en 2007 \cite{libroEnergiaEnEspana2019}.\\\\
De 2008 a 2014, como consecuencia de la crisis económica el consumo energético disminuyó hasta unos 117824 ktep ($\sim 1.37E12 \ kWh$) en 2014, recuperándose a partir del 2015, como se puede observar en la figura \ref{fig:demandaenergiaprimariaproporcion1990}. \\


\begin{figure}[H]
	\centering
	\includegraphics[width=10cm, height=7cm]{figuras/DemandaEnergiaPrimariaProporcion1990}
	\caption[Relación entre demanda total de energía primaria anual]{Representación gráfica de la relación entre la demanda total de energía primaria anual de 1990 hasta 2019 en España. \textit{Fuente obtención de datos: Ministerio para la transición ecológica y el reto demográfico de España.} }
	\label{fig:demandaenergiaprimariaproporcion1990}
\end{figure}
Dicha energía se puede dividir en diferentes categorías según el sector que la consume o la fuente de la misma. En el 2019, un 44.5\% de la energía primaria fue consumida de productos petrolíferos y apenas un 14.5\% de renovables (figura \ref{fig:fuentesenergiasprimarias}), y un 23.6\% de la energía final fue consumida por el sector de la industria (figura \ref{fig:consumoenergiafinalporsectores2019}), representando casi una cuarta parte del consumo total de la energía final en España. Desafortunadamente en todos los procesos energéticos, ya sean de producción, transporte o consumo, una parte de la energía siempre es desaprovechada en forma de calor (conocido como calor residual). Por ejemplo, en 2015 la industria en España consumió aproximadamente 220 TWh de energía, desperdiciándose unos 22 TWh según los cálculos de las estimaciones realizadas en \cite{wasteEnergyindustryEstimate}.

\begin{figure}[H]
	\centering
	\begin{subfigure}[b]{0.48\textwidth}
		\includegraphics[width=\textwidth]{figuras/FuentesEnergíasPrimarias}
		\caption{
		%Consumo de energía por fuente
		}
		\label{fig:fuentesenergiasprimarias}
	\end{subfigure}
	\hfill
	\begin{subfigure}[b]{0.48\textwidth}
		\centering
		\includegraphics[width=\textwidth]{figuras/consumoEnergiaFinalPorSectores_2019}
		\caption{
		%Consumo de energía por sectores.
		}
		\label{fig:consumoenergiafinalporsectores2019}
	\end{subfigure}
\caption[Desglose del consumo de energía primaria en España 20]{Desglose del consumo de energía primaria en España 2019 por fuente de energía (\subref{fig:fuentesenergiasprimarias}). Consumo de energía final por sectores en el año 2019 en España (\subref{fig:consumoenergiafinalporsectores2019}). \textit{Fuente:  Ministerio para la transición ecológica y el reto demográfico de España.} }
 \label{fig:consumosEnergiasCategorias}
\end{figure}

Para recuperar esta energía perdida se han desarrollado instalaciones de recuperación de calor residual, no solo para disminuir los costes sino también para disminuir las emisiones de contaminantes, cumpliendo así parte de los objetivos de desarrollo sostenible. El calor residual se puede aprovechar para obtener electricidad o para calentar un fluido para mejorar la eficiencia del mismo u otro proceso, disminuyendo el consumo de combustible o energía.\\

El aprovechamiento del calor residual para el calentamiento de otro fluido se utiliza en la industria. Algunos ejemplos de esto son: los economizadores que se usan en calderas para pre-calentar los fluidos, mejorando el rendimiento térmico, y en el sector residencial se aprovecha el calor obtenido del sistema de enfriamiento de algunos sistemas fotovoltaicos para obtener agua caliente.\\ 

La obtención de electricidad mediante el aprovechamiento del calor residual puede ser por vía de trabajo mecánico o por conversión directa. Por vía de trabajo mecánico se produce mediante ciclos termodinámicos, principalmente el ciclo de Rankine, el cual mediante un intercambiador de calor se genera vapor de agua que luego hace mover unas turbinas que están conectadas a un generador. La conversión directa se basa en la utilización de dispositivos termoeléctricos (\acrshort{teg}), termoiónicos (\acrshort{tic}) y termofotovoltaicos (\acrshort{tpv}).\\

La conversión térmica por vías mecánicas es la más conocida y es altamente utilizada, por ejemplo, en las centrales nucleares, los sistemas de ciclos combinados, los sistemas de tratamientos de residuos que necesitan ser incinerados (figura \ref{fig:esquemaslomasvalorizacion}), entre otros. También se puede aprovechar el calor residual de la generación eléctrica, como es el caso de los cogeneradores.

%% IMAGENES DE SISTEMAS DE APROVECHAMIENTOS MECÁNICOS
\begin{figure}[H]
	\centering
	\includegraphics[height=5cm]{figuras/esquemasLomasValorizacion}
	\caption{Planta de valorización energética del Centro Las Lomas del Parque Tecnológico de Valdemingómez, donde se aprovecha los gases a alta temperatura para producir vapor de agua en la caldera para mover unas turbinas conectadas a generadores para producir electricidad. \textit{Fuente: Ayuntamiento de Madrid}}
	\label{fig:esquemaslomasvalorizacion}
\end{figure}
Las principales desventajas que presentan estos sistemas son: el uso de partes móviles que tienen que estar en constante mantenimiento por desgaste, el uso de fluidos que complica el control del sistema y que suelen requerir un tamaño mínimo a partir del cual pasan a ser rentables, con lo cual se reduce el número de aplicaciones donde pueden ser utilizados. Sin embargo, los dispositivos de estado sólido (\acrshort{teg}, \acrshort{tic} y \acrshort{tpv}) son escalables, pudiendo encontrar más aplicaciones donde ser utilizados (a nivel residencial, en aplicaciones espaciales, automoción, etc.).\\

% Termo Electrico
Los \acrshort{teg} son dispositivos que convierten la energía térmica en electricidad y viceversa, que están compuestos por dos materiales con coeficientes Seebeck opuestos unidos en sus extremos \cite{tegDef1}. Al aplicarse una diferencia de temperatura entre ambas uniones se genera una diferencia de potencial, conocido esto como efecto Seebeck. Los materiales que se utilizan para estos dispositivos tienen unas conductividades térmicas bajas, para disminuir las pérdidas por conducción de calor. Para aumentar la potencia de los \acrshort{teg} se aumenta la diferencia de los coeficientes Seebeck porque cuanto mayor sea esta diferencia, mayor será la tensión y la potencia de salida, como se pueden observar en las ecuaciones de \cite{tegDef1}.\\

La eficiencia de los \acrshort{teg} es alrededor del 5\% \cite{TEG5efficiency}, aunque hasta el momento la eficiencia ha llegado a un máximo de aproximadamente un 15\% \cite{TEG15efficiency}, lo cual sigue siendo demasiado baja. Aún con bajas eficiencias los \acrshort{teg} se utilizan para aplicaciones espaciales, recuperación de calor en el transporte, entre otros.\\


% Termoiónico
Otro sistema de obtención de electricidad de una fuente de calor es la generación termoiónica, que mediante la emisión termoiónica se produce un flujo de electrones entre el emisor metálico a alta temperatura y un receptor a menor temperatura, separados por el vacío. Al aumentar la temperatura del emisor, los electrones libres se excitan hasta tal punto que la energía es lo suficientemente grande como para que se desprendan del material, la densidad de corriente está determinada por la ley de Richardson, donde $J=A\cdot T^2\cdot e^{\frac{-W}{k\cdot T}}$, siendo \textit{J} la densidad de corriente, \textit{A} la constante de Richardson, \textit{W} la función de trabajo del metal, \textit{k} la constante de Boltzman y \textit{T} la temperatura en Kelvin.\\

% Termofotovoltaico
Un generador que se puede combinar con el \acrshort{tic}, es un generador \acrshort{tpv} que transforman en electricidad la radiación electromagnética producida por el emisor a alta temperatura para ser capturada por el receptor, que es una célula fotovoltaica (\acrshort{pv}). Los emisores de las \acrshort{tpv} no tienen que estar a unas temperaturas tan altas como los \acrshort{tic}, ya que no necesita que se desprendan los electrones. Ambos dispositivos presentan el mismo problema de pérdidas de electrones o fotones por los bordes del convertidor y para disminuir esto se disminuye la separación entre el emisor y el receptor. Cuando las distancias se vuelven nanométricas aumenta la transferencias de fotones por la transferencia de radiación evanescente y/o electrones por la eliminación del efecto de carga de espacio, generando una mayor potencia eléctrica porque la potencia radiada aumenta con respecto a la proporcionada por la ley de Planck, conocido como ``efecto de campo cercano''.\\

La eficiencia de las células \acrshort{tpv} han aumentado con el tiempo hasta llegar aproximadamente a un 40\% con una célula de dos uniones con materiales de ancho de banda (\acrshort{bg}) entre 1 y 1.4 eV, utilizando un reflector trasero que devuelve la radiación de menor energía al \acrshort{bg} directa al emisor, pero solo llegando a obtener una densidad de potencia de 2.39 $W{cm}^{-2}$ \cite{ThermalConductivity_SiO2_2018}. Se pueden obtener potencias eléctricas mayores mediante el uso del efecto de campo cercano.\\

Los dispositivos que aprovechan los efectos de campo cercano presentan la dificultad de mantener de manera constante las distancias de separación nanométricas a lo largo de grandes superficies (relativo a la distancia de separación). Los dispositivos de campo cercano estudiados hasta la fecha presentan áreas muy pequeñas $<0.3 \ mm^2$ \cite{inoue_one-chip_2019_0_3mm2}. Para poder fabricar dispositivos TPV que aprovechan los efectos de campo cercano con un área considerable de $\sim 1 \ cm^2$, se requiere el uso nano-espaciadores. Por ejemplo, en \cite{doi:MicroGapTPV} usan espaciadores de dióxido de silicio (\gls{sio2}) y se observó que al disminuir la distancia de separación la corriente de cortocircuito aumenta.


\section{Objetivos}
El objetivo de este estudio es diseñar y simular pilares de dimensiones nanométricas para el aprovechamiento del efecto de campo cercano en un convertidor \acrshort{tpv}, así como determinar la viabilidad de este sistema. 
\begin{itemize}
	\item Modelar un nano-espaciador dentro de los rangos permitidos de los parámetros de los programas de simulación y modelado 3D.
	\item Ensamblar los sistemas TPV para diferentes alturas del nano-espaciador para las simulaciones.
	\item Simular la transferencia de calor por conducción a través de un nano-espaciador de $SiO_2$ de un sistema TPV para diferentes alturas del nano-espaciador, diferentes materiales de emisor y diferentes resistencias de contacto entre emisor y el nano-espaciador.
	\item Simular la transferencia de calor radiada entre el emisor y la célula para diferentes materiales del emisor y para varias distancias de separación, teniendo en cuenta los efectos de campo cercano en la radiación.
	\item Determinar entre los casos estudiados cuales pueden dar lugar a sistemas de TPV de campo cercano viables.
\end{itemize}

%%% NECESARIO?
\section{Estructura del documento}

A continuación y para facilitar la lectura del documento, se detalla el contenido de cada capítulo.

\begin{itemize}
\item En el \textbf{capítulo 1} se realiza una introducción del trabajo con la respectiva motivación y objetivos.
\item En el \textbf{capítulo 2} se desarrolla el estado de arte, definiendo los apartados más importantes y resaltando las investigaciones con mayor relevancia.
\item En el \textbf{capítulo 3} se exponen las herramientas y materiales utilizados, así como los criterios de selección.
\item En el \textbf{capítulo 4} se mencionan los métodos seguidos y los cálculos realizados para el desarrollo del trabajo.
\item En el \textbf{capítulo 5} se exponen los resultados obtenidos de las simulaciones.
\item En el \textbf{capítulo 6} se desarrolla la conclusión y se realiza planteamientos para futuros trabajos.
\end{itemize}

  
  %partes finales del trabajo: conclusiones, bibliografia y anexos
  
\chapter{Estado del arte}


En este capítulo...
\cite{noauthor_parallel-plate_nodate}
\cite{doi:MicroGapTPV}
\cite{doi:Thermoionic_Campbell}
\cite{doi:Near-field_ThinFilm}

\section{Introducción al capítulo}
%\section{Sector Energético}
%\section{Fotovoltaica}
\section{Termo-fotovoltaica}

\section{Efecto Campo Cercano}
\section{Resistencia de contacto}
\section{Investigaciones más importantes}



\chapter{Materiales y Herramientas}
%\vspace{-1cm}
En este capítulo se exponen los materiales utilizados para cada componente del sistema, las herramientas utilizadas para el modelado 3D del nano-espaciador, simulación de la transmisión de calor por conducción y simulación de transmisión de calor por radiación por campo cercano, y los motivos de uso de dichos materiales y herramientas.
\section{Materiales}
Se listan y desarrolla los materiales utilizados para simular cada uno de los componentes del sistema y los criterios seguidos para su eleccións.
\subsection{Nano-espaciador}
Suponemos que los nano-espaciadores son de Dióxido de Silicio ($SiO_2$) \cite{doi:10.1063/1.1141498}, un material que ha sido utilizado en varias ocasiones en sistemas de campo cercano de TPVs o TEC por ser un material cerámico con una baja conductividad térmica, buena resistencia ante choques de calor y alta fuerza de compresión. La conductividad térmica es menor a 10 W/m\textdegree C en el rango de temperatura de trabajo, entre 25 a 800\textdegree C.\\\\
A su vez se considera el $SiO_2$ con tres porosidades distintas, la primera es con una porosidad del 0\%, segundo con una porosidad del 25\% y tercero con una porosidad del 50\%, calculando la nueva conductividad térmica para los casos de porosidad distinta al 0\% según el modelo CP en \cite{ThermalConductivity_SiO2_2018}.

\subsection{Célula}
Suponemos inicialmente que las células son de Silicio (Si) para llevar a cabo unos cálculos preliminares utilizando un material ampliamente conocido y utilizado en la fabricación de células fotovoltaicas. En el resto de simulaciones se utiliza Germanio (Ge), que es un semiconductor como el Si pero que es capaz de absorber una mayor cantidad de radiación por tener una banda energética (Band Gap) de 0.66eV a 300K, a diferencia del Si que es unos 1.11eV a 300K. La capacidad de absorber más radiación es importante en la termofotovolaica porque la mayor cantidad de radiación es infrarroja.\\

También se utiliza el Ge por ser el material de las células termofotovoltaicas a disposición en el Instituto de Energía Solar (IES).
\subsection{Emisor}
Para simular el emisor supondremos varios materiales: Silicio, Acero inoxidable 304L (SS) y Carburo de Silicio (SiC). Se utilizá Si para las pruebas de simulación iniciales para tener un punto de partida, para luego pasar al \textit{SS} por ser un material que se utiliza mucho en la industria, incluyendo al transporte, y por último el SiC que es una cerámica que ha se ha utilizado para el aumento de potencia de radiación en campo cercano \cite{doi:Near_field_ThinFilm}.
\begin{figure}[H]
	\centering
		\includegraphics[width=0.6\textwidth]{figuras/rad_mat/EgRad.png}
	\caption{Potencias por unidad de área para una célula de Ge y diferentes materiales de emisor para un rango de longitudes de onda desde aproximadamente 0.2 $\mu m$ hasta 1.8 $\mu m$ (0.7 eV) con distancias de separación de 100nm y 1000nm según las ecuaciones de \cite{nfTPV_equations} calculadas a partir de las ecuaciones \eqref{eq:flujoPropNF} y \eqref{eq:flujoEvasNF}.}
	\label{fig:EgRad}
\end{figure}
También se analizaron otros materiales adicionales que finalmente fueron descartados por ser muy caros, o proporcionar una baja potencia radiada (figura \ref{fig:EgRad}), etc. %, similar conductividad térmica al SiC, Si o SS, punto de fusión menor o cercano a los 800\textdegree C, o por ser demasiado blandos. 
Dichos materiales son los siguientes:

\begin{itemize}
	\item \textbf{Antimoniuro de Galio (GaSb)}: Semiconductor con punto de fusión de 710\textdegree C que es menor a los 800\textdegree C que se encuentra el emisor, por lo cual no se puede utilizar porque empezará a pasar a estado líquido antes de llegar a la temperatura deseada.
	\item \textbf{Grafito}: es un material relativamente barato con una buena conductividad térmica, alto punto de fusión, pero es muy blando respecto al $SiO_2$, tiene como dureza máxima unos 50 HV (dureza Vickers) en comparación a la dureza del $SiO_2$ de unos 500 HV, por lo cual cuando se aplique presión a los dispositivos el nano-espaciador se hará paso por el emisor.
	\item \textbf{Tungsteno (W)}: El coste del material es medio, como máximo unos 56.4€/kg, su dureza Vickers es de unos 350 HV que es adecuada por no ser tan inferior a la del $SiO_2$, su conductividad térmica es más alta que la del nano-espaciador pero la potencia radiativa en campo cercano es baja, más cercana a la del acero.
	\item \textbf{Alumina ($Al_2O_3$)}: Es un material muy caro, aproximadamente 16000€/kg o más, disminuyendo considerablemente la viabilidad del sistema.
\end{itemize}

\section{Herramientas}

Los herramientas o programas utilizados durante el desarrollo del trabajo son variados y por diferentes motivos, a continuación se da una breve descripción y se explican los motivos de la utilización de cada programa.\\\\
Para la obtención de las propiedades ópticas y térmicas de los materiales se utilizan Granta EduPack 2021 R2, la página web Refractive Index y el resto de datos de los materiales que no se encuentren en dichas herramientas se obtienen de artículos de investigación. Para el modelado 3D se utiliza Autodesk Inventor 2021, para las simulaciones de la transmisión de calor por conducción se utiliza Autodesk CFD 2021, para la transmisión de calor por radiación se utiliza un script de Matlab que resuelve las ecuaciones \eqref{eq:flujoPropNF} y \eqref{eq:flujoEvasNF} a partir de las propiedades ópticas (n-k datos) de los materiales implementado en la calculadora de campo cercano, y para el análisis de datos se utiliza MATLAB.\\\\
Unas herramientas a destacar son TexStudio y TexnicCenter utilizadas para la redacción de este trabajo por la facilidad que presenta el escribir ecuaciones, actualizaciones de figuras y la menor necesidad de recursos que Word. También hay que destacar el uso de Git para el control de versiones del documento.\\
%%%  GRANTA EDUPACK 2021 R2
\subsection{Granta EduPack 2021 R2}
Programa desarrollado por la empresa Ansys que contiene varias bases de datos que recopila información sobre distintos materiales, teniendo un fácil acceso a la información básica de los mismos a través de una interfaz gráfica (GUI) y estando organizadas de manera específica para cada base de datos. Este programa está disponible en la versión de Windows 10 del escritorio remoto de la UPM.\\\\
La base de datos utilizada es la \textit{Level 3} de las avanzadas porque recopila los datos de todos los materiales necesarios, en la mayoría de los casos incluye la variación de algunas propiedades respecto a la temperatura, siendo principalmente necesaria la conductividad térmica del material respecto a la temperatura.\\\\
Dado la gran cantidad de datos a disposición y la facilidad de obtención de los datos de los materiales, especialmente los datos que están disponibles con la variación de la temperatura, se utiliza este programa.
%%%   REFRACTIVE INDEX
\subsection{Refractive Index}
Esta página web (https://refractiveindex.info) recopila de artículos científicos los índices de refracción (n) e índices de extinción (k) de diferentes materiales en diferentes rangos de longitudes de onda, teniendo a disposición hipervínculos a los artículos de donde se extrajo los datos y la opción de descargar dichos datos.\\\\
El motivos de su utilización son la facilidad para la obtención de los índices de refracción y extinción, necesarios para los cálculos del flujo térmico de radiación por campo cercano, y la disponibilidad de los artículos utilizados para contrastar los valores presentados.

%\subsection{Modelado 3D}
%Para el modelado 3D de cada componente del sistema (emisor, célula y nano-espaciador) se utiliza Autodesk Inventor 2021 por la facilidades que presenta autoDesk de utilizar los archivos de un programa de la misma compañía en otro programa para continuar con el estudio del sistema.ç
%%%  Autodesk INVENTOR 2021
\subsection{Autodesk Inventor Professional 2021}
Es un software de Diseño 3D Asistido por Ordenador (CAD 3D) que ofrece herramientas para el diseño de piezas, ensamblajes, mecanismos, documentación y simulación de los sistemas. También permite el utilizar herramientas de entorno añadidas para simulaciones más complejas, el fácil acoplamiento para el uso de otros programas de Autodesk y el uso de parámetros en el diseño y modelado de las piezas.
\begin{figure}[H]
	\centering
		\includegraphics[width=3cm]{figuras/inventorpro.png}
	\caption{Icono de Autodesk Inventor Professional 2021}
	\label{fig:inventorpro}
\end{figure}
Se utiliza el programa para el modelado 3D de los nano-espaciadores, el emisor, la célula y el sistema TPV completo. Los motivos de utilización del programa son:

\begin{itemize}
	\item Licencia estudiantil facilitada por la Universidad Politécnica de Madrid (UPM).
	\item Menor consumo de recursos que otros programas de modelado 3D.
	\item No necesita una conexión constante a internet para su uso.
	\item Acoplamiento con Autodesk CFD, permitiendo el lanzar modelos de Inventor a CFD. 
\end{itemize}
Una desventaja de este programa es la limitación que presenta en las unidades mínimas de trabajo ($mm$), no pudiendo realizar modelos en escala nanométrica por lo que se utilizan distancias geométricas mayores, teniendo que adaptar los valores de las propiedades térmicas de los materiales para tener ese cambio de escala en cuenta.
%\subsection{Simulaciones de transmisión de calor por conducción}
%Para la simulación de transmisión de calor se usa el programa Autodesk Computational Fluid Dynamics (CFD), que aún ser usado principalmente para simulaciones con fluidos también permite simular solo transmisión de calor por conducción en sólidos y permite usar los diseños de Autodesk Inventor para simular, siendo fácilmente creado el provecto de CFD desde Inventor en la pestaña nueva de simulación, que se crea automáticamente con la instalación de CFD, y haciendo click sobre \textit{Active Model Assessment Tool}. Usándose exclusivamente para las simulaciones de conducción porque no puede simular la transmisión de calor por campo cercano.
\subsection{Autodesk CFD}
Es un software desarrollado por Autodesk que se usa para crear simulaciones computacional de la dinámica de fluidos para la predicción del comportamiento complejo de fluidos (líquidos y gases), el programa también simula la transmisión de calor por conducción en sólidos, convección en fluidos y de radiación. \\
\begin{figure}[H]
	\centering
		\includegraphics[width=3cm]{figuras/CFD.png}
	\caption{Icono de Autodesk CFD}
	\label{fig:CFD}
\end{figure}
Se utiliza el programa para las simulaciones de transmisión de calor por conducción en vez de el entorno Nastran de Inventor por las siguientes razones:
\begin{itemize}
	\item Utilización de escalas pequeñas de los parámetros de los materiales.
	\item Fácil control del mallado.
	\item Control de los pasos de la simulación, la frecuencia de guardado de datos y el número de iteraciones.
	\item Fácil obtención de los datos de los flujos de calor y las temperaturas del espaciador en formato CSV.
\end{itemize}

\subsection{MATLAB}
MATLAB es un programa o plataforma de programación de \textit{MathWorks} que permite desarrollar cálculos numéricos, desarrollar algoritmos, analizar datos y crear modelos matemáticos de sistemas para cálculo o simulaciones utilizando su propio lenguaje de programación. A su vez, MATLAB proporciona una tienda de paquetes donde los usuarios puede subir aplicaciones desarrolladas en MATLAB que permiten aprovechar aún más los recursos del sistema y facilitar el trabajo, ya sea analizar datos, obtener modelos matemáticos, entre otros.\\
\begin{figure}[H]
	\centering
		\includegraphics[width=3cm]{figuras/MatlabIcon.png}
	\caption{Icono de MATLAB}
	\label{fig:MatlabIcon}
\end{figure}
Se utiliza MATLAB para el análisis de los resultados obtenidos de las simulaciones de transmisión de calor por conducción y radiación, para realizar gráficas, para realizar cálculos y para desarrollar la aplicación de la calculadora de campo cercano. El principal motivo de usar MATLAB es para calcular la transmisión de calor en condiciones de campo cercano.

\subsection{Calculadora de campo cercano} \label{sec:calc_campo_cercano}
Para las simulaciones de transmisión de calor por radiación de campo cercano entre dos placas paralelas de área infinita se utiliza un código escrito en MATLAB que ya estaba implementado y que cumple con las ecuaciones de campo cercano para dos placas planas gruesas (ecuaciones \eqref{eq:flujoPropNF} y \eqref{eq:flujoEvasNF}). A partir de dicho código se desarrolla una aplicación en el App Designer de MATLAB que facilita el proceso, porque mejora la interfaz de usuario y disminuye los tiempos de simulación al realizar los cálculos con hilos (Parallel Processing Toolbox), permitiendo así realizar combinaciones de materiales y distancias de manera sencilla.\\\\
Los motivos que impulsaron para desarrollar la aplicación son el simplificar el proceso de simular la radiación por campo cercano para diferentes combinaciones de materiales y distancia de separación, y la obtención de la potencia total para el rango de longitudes de onda deseadas para todas las combinaciones.\\

La aplicación se divide en tres pestañas \textit{Potencia Radiada}, \textit{Potencia} y \textit{Materiales}. En cuanto se abre la aplicación se presenta abierta la pestaña de \textit{Potencia Radiada} (figura \ref{fig:pestana_PotenciaRadiada}), donde se seleccionan los materiales del emisor y el receptor, las distancias de separación de los cuerpos y la temperatura del emisor en grados Kelvin, ya que la del receptor es fija a 300K. También muestra la gráfica de los resultados obtenidos de la simulación para cualquier combinación excepto cuando se eligen varios materiales en un rango de distancias porque todavía no está implementado para este caso de dos rangos.\\ 
\begin{figure}[H]
	\centering
		\includegraphics[width=0.80\textwidth]{figuras/pestana_PotenciaRadiada.png}
	\caption{Pestaña \textit{Potencia Radiada} de la calculadora de campo cercano}
	\label{fig:pestana_PotenciaRadiada}
\end{figure}
Para la selección del rango de los materiales se abre una nueva ventana de una GUI donde se utiliza dos árboles de casillas de selección para los materiales del emisor y el receptor del sistema de transmisión de calor por radiación de campo cercano (figura \ref{fig:ventana_mat}). Para la selección del rango de distancias se abre una nueva ventana de una GUI donde se selecciona en múltiplos de 100 el valor máximo y el valor mínimo del rango de distancias a simular, las distancias máxima y mínima son 1000nm y 100nm respectivamente (figura \ref{fig:ventana_dist}).Antes de realizar los cálculos se realiza una interpolación lineal de los valores de n y k de los materiales para que coincidan los puntos para cada longitud de onda.\\\\
\begin{figure}[H]%
\centering
\begin{subfigure}[b]{0.48\textwidth}
\centering
	\includegraphics[width=0.75\textwidth]{figuras/pestana_Distancias.png}
	\caption{Ventana de selección de rango de distancias}%
	\label{fig:ventana_dist}%
\end{subfigure}
\hfill
\begin{subfigure}[b]{0.48\textwidth}
\centering
	\includegraphics[width=0.75\textwidth]{figuras/pestana_Elegirmateriales.png}
	\caption{Ventana de selección de materiales}%
	\label{fig:ventana_mat}%
\end{subfigure}
\caption{(\subref{fig:ventana_dist}) Ventana para la selección del rango de las distancias de separación entre el emisor y el receptor. (\subref{fig:ventana_mat}) Ventana para la selección de los materiales para el emisor y el receptor.}%
\label{fig:ventanas_d_mat}%
\end{figure}
La pestaña de \textit{Potencia} (figura \ref{fig:pestana_Potencia})contiene una calculadora para pasar de electrón-voltios a longitud de onda y viceversa, unos campos para controlar el rango de longitudes de onda para realizar la integral de la potencia radiada monocromática y una gráfica para mostrar los resultados obtenidos de integrar. La integral se realiza por el método del trapezoide. Los resultados obtenidos de la potencia integrada (figura \ref{fig:pestana_Potencia}), los de la radiación espectral o ambos se pueden guardar en un archivo excel.
\begin{figure}[H]
	\centering
		\includegraphics[width=0.70\textwidth]{figuras/pestana_Potencia.png}
	\caption{Pestaña de la Potencia}
	\label{fig:pestana_Potencia}
\end{figure}
La pestaña de \textit{Materiales} se usa para observar en una tabla o gráfica, según la pestaña interna seleccionada, los valores de n y k del material seleccionado en la lista de la izquierda, como se puede observar en la figura \ref{fig:pestana_materiales}.
\begin{figure}[H]
	\centering
		\includegraphics[width=0.70\textwidth]{figuras/pestana_materiales.png}
	\caption{Pestaña de los datos de n y k  de los materiales disponibles}
	\label{fig:pestana_materiales}
\end{figure}
Para obtener una mejor fidelidad de los resultados de la simulaciones se limita el rango de los datos dentro de las longitudes de onda que comparten todos los materiales del conjunto que se va a simular.

\chapter{Métodos}
En este capítulo se detalla principalmente los criterios seguidos para el diseño del espaciador, su modelado en Inventor, las simulaciones en CFD y los cálculos en MATLAB, a su vez los métodos empleados para la rectificación del correcto funcionamiento de CFD para la extracción de datos.

En segundo lugar se describe el procedimiento realizado para la extracción de datos de las simulaciones realizadas en CFD.

Por último, se describe el procedimiento seguido para la obtención de los resultados.
\section{Criterios seguidos} 

\section{Simulación en CFD}


\section{Análisis dimensional}
Dado que el programa Inventor está limitado a partir a la escala milimétrica, por lo tanto, se tiene que aplicar un escalado. Se procede a realizar un análisis dimensional de las ecuaciones para obtener los nuevos parámetros para la nueva escala del modelo.\\

Siendo $L'$ la longitud en el modelo y $L$ la longitud de la realidad, se considera que ${L'}/{L}=10^4$.
\subsection{Área}
La sección de los nano-espaciadores es un polígono regular de cuatro lados cuya fórmula del área es $A=L^2$, donde $A$ es el área y $L$ el lado del polígono.\\

Para cualquier polígono regular a fórmula del área se puede expresar como $A=Cte\cdot L^2$, siendo $Cte$ una constante distinta para cada polígono y tomando en cuenta la relación de escala.

\begin{equation}
	\dfrac{A'}{A}=\left(\dfrac{L'}{L}\right)^2=10^8
\end{equation}
\subsection{Volumen}
El volumen de un prisma de base de polígono regular se expresa como $V=A\cdot L$, donde $L$ es la altura del prisma.
\begin{equation}
	V'=A'\cdot L' = A\cdot L \cdot 10^8\cdot 10^4= V\cdot 10^12
\end{equation}

El volumen de cada nano-espaciador en el modelo será $10^12$ veces el volumen original.
\subsection{Conductancia Térmica}

Al aplicarse la escala la conductancia térmica (\textit{C}) se mantiene constante, variando la conductividad térmica (\textit{k}) del material. La nueva conductividad térmica $k'$ será:

\begin{equation}
	C=k\cdot \frac{A}{L}=k'\cdot\frac{A'}{L'}
\end{equation}
\begin{equation}	
	 k'=\dfrac{L'}{L} \cdot \dfrac{A}{A'} \cdot k=10^4\cdot 10^{-8}=k\cdot 10^{-4}
\end{equation}

Donde $k$ es la conductividad térmica del material  y $k'$ es la conductividad térmica para la escala aplicada. $k'=k\cdot10^{-4}$

\subsection{Calor Específico}

\subsection{Coeficiente de expansión térmica}

\subsection{Densidad}
La masa de cada elemento tiene que ser constante entre el modelo y la realidad.
\begin{equation}
	M=M'\Longrightarrow \rho\cdot V=\rho'\cdot V'= \rho'\cdot V\cdot {10}^{12}\Longrightarrow \rho'=\rho\cdot{10}^{-12}
\end{equation}

La densidad de cada elemento en el modelo será $10^{-12}$ veces la densidad de la realidad.

\subsection{Resistencia de Contacto}
En \cite{noauthor_parallel-plate_nodate} se descubre como la resistencia de contacto afecta a los espaciadores y se obtiene el valor de su coeficiente($\rho$) $4\cdot 10^{-6}m^2 KW^{-1} $. Para obtener cual es el coeficiente de la resistencia de contacto del modelo se realiza un análisis dimensional.

\begin{equation}
	R= \rho \cdot A =\rho ' \cdot A' \Longrightarrow \rho '=\rho \cdot \dfrac{A}{A'}=\rho \cdot 10^8
\end{equation}

La resistencia térmica por contacto

Según el módelo de XXXXXXX:
\begin{equation}
	h_c=\dfrac{k_s\cdot}{m}\cdot \left( \dfrac{P}{H_c} \right)^{0.95} \\ \dfrac{P}{H_c} = \left[ \dfrac{P}{c1\cdot \left(1.62\cdot \sigma/m \right)^{c_2}}\right]^{\dfrac{1}{1+0.0071\cdot c_2}} \\	\sigma =\sqrt{{\sigma_1}^2+{\sigma_2}^2} \\ m=\sqrt{{m_1}^2+{m_2}^2}	\\ k_s=2\cdot \dfrac{k_1\cdot k_2}{k_1+k_2}
\end{equation}

Para el dato hc=1000;
\begin{equation}
	\dfrac{\sigma}{m}=\dfrac{\sigma_1}{m1}	\\ h_{c1}=k_{s1}\cdot cte \\ h_{c2}=k_{s2}\cdot cte	\\ k_{s1}=k_1 \\ \dfrac{h_{c2}}{h_{c1}}= 2\cdot \dfrac{k_2}{k_1+k_2}=cof
\end{equation}
k1=15 y k2=1.5
\begin{equation}
	cof=2\cdot \dfrac{1.5}{15+1.5}=0.18182
	\\ h_{c2}=h_{c1}\cdot cof=181.82 \rightarrow h_{c2}=181.82 \rightarrow R_{c2}=5.5E-3
\end{equation}

%\chapter{Cómo escribir en Latex}

\section{Citas}

%las referencias a artículos se ponen con \cite, 
%las referencias a imágenes \ref, 
%y las referencias a ecuaciones \eqref

Esto es un ejemplo de cita de un artículo %\cite{Brunete:2013}.


\section{Listas}

%itemize es una lista. Cada término lleva delante un \item
Ejemplo de lista de puntos:
\begin{itemize}
\item Ejemplo1.
\item Ejemplo2.
\end{itemize} 

Y lista numerada:
\begin{enumerate}
\item Elemento 1
\item Elemento 2
\end{enumerate}

\section{Tablas}

Ejemplo de tabla. Como se aprecia en la tabla \ref{tab:table_example}...
\begin{table}[tb]
\caption{Ejemplo de tabla}
\label{tab:table_example}
\begin{center}
\begin{tabular}{|c||c|c|}
\hline
One & Two & Three\\
\hline
F1A & F1B & F1C\\
F2A & F2B & F2C\\
\hline
\end{tabular}
\end{center}
\end{table}

\section{Referencia a una sección}
\label{sec:refsec}

Ejemplo de referencia a la sección \ref{sec:refsec}

\section{Texto}

Testo en \textbf{negrita} y \textit{cursiva}.

\section{Figuras}

Ejemplo de referencia a figura (figura \ref{fig:logo_upm}). Es importante que todas las figuras que aparezcan estén referenciadas, así como las tablas. En general las figuras se colocarán al principio o al final de cada página ([tb] en latex), a no ser que por alguna necesidad se deban colocar en una posición exacta ([h]).

%caption es el pie de foto, y label es el nombre que se da a la imagen para referenciarla después. label no puede llevar acentos y no se muestra de cara al documento final (es sólo interno).
\begin{figure}[tb]
\centering
\includegraphics[width=0.45\textwidth]{figuras/Logo_UPM.jpg}   
\caption{Logotipo de la UPM}
\label{fig:logo_upm}
\end{figure}

\chapter{Resultados y discusión}

En este capítulo se presentan los resultados obtenidos de las simulaciones de transmisión de calor por conducción y radiación, comparando varios casos de estudio de diferentes materiales de emisor y célula, como varias distancias de separación entre el emisor y la célula.


\section{Resultados}
\begin{figure}
	\begin{tikzpicture}
		\begin{axis}[]
			\addplot[] {exp(-x/10)*( cos(deg(x)) + sin(deg(x))/10 )};
		\end{axis}
	\end{tikzpicture}
\end{figure}


\section{Discusión}
%%\chapter{Gestión del proyecto}

En este capítulo se describe la gestión del proyecto: ciclo de vida, planificación, presupuesto, etc.

\section{Ciclo de vida}

Explicación de las fases del proyecto: definición, análisis, diseño, construcción, pruebas, implementación, validación, documentación. Ejemplo: diagrama de Pert.

\section{Planificación}

Se puede indicar mediante un diagram de Gantt.

\subsection{Planificación inicial}

\subsection{Planificación final}


\section{Presupuesto}

\subsection{Personal}

\subsection{Material}

\subsection{Resumen de costes}

\chapter{Conclusiones}
En este capítulo se presentan las conclusiones de las simulaciones realizadas y la viabilidad de cada caso estudiado, también se presentan los posibles desarrollos futuros en base a los resultados obtenidos en este trabajo.

\section{Conclusiones del trabajo}
En primer lugar, se concluye para el caso sencillo de una nTPV de $Si-SiO_2Si$ que el aumento de la porosidad del material del nano-espaciador disminuye las pérdidas de conducción por la disminución de la conductividad térmica, obteniéndose un modelo analítico que relaciona las pérdidas por conducción con la porosidad y la altura del nano-espaciador. También se concluye que la existencia de una resistencia de contacto es de gran importancia para la disminución de las pérdidas, pero no lo suficiente para que con una célula de Silicio se obtengan buenas relaciones entre potencia de radiación útil y pérdidas.\\\\


\section{Desarrollos futuros}

Un posible desarrollo...
\appendix

\chapter[Procedimientos simulaciones radiación]{Procedimientos seguidos para la realización de las simulaciones de radiación de campo cercano}\label{ch:procedimientosSimRad}

En este apéndice se detallan los pasos genéricos seguidos para la realización de las simulaciones de radiación de campo cercano y la obtención de los resultados de las potencias integradas en un respectivo rango espectral utilizando la aplicación \textbf{calculadora de campo cercano}, descrita en la sección \ref{sec:calc_campo_cercano}.

\section{Procedimientos genéricos}
Existen solo cuatro posibles casos de simulaciones de transmisión de calor por radiación de campo cercano utilizando la aplicación, estos casos son los siguientes:
\begin{itemize}
	\item Una distancia de separación y un par de materiales, uno para el emisor y otro para el receptor.
	\item Rango de distancias de separación y un par de materiales.
	\item Una distancia de separación y varias combinaciones de materiales.
	\item Rango de distancias de separación y varias combinaciones de materiales.
\end{itemize}
Los procedimientos para la realización de las simulaciones de transmisión de calor por radiación de campo cercano siguen el mismo orden de ejecución genérico. Estos procedimientos son:
\begin{itemize}
	\item Selección de la temperatura del emisor, que para este trabajo se mantiene a 1073K (800\textdegree C).
	\item Selección de las combinaciones de materiales a simular.
	\item Selección del rango de distancias a simular.
	\item Hacer clic sobre el botón \textbf{Calculate}.
\end{itemize}
%%%%%%%%%%%%%%%%%%%%%%%%%%%%%%%%%%%%%%%%%%%%%%%%%%%%%%%%%%5
%%%%%%%%%%%%%%%%%%%%%%%%%%%%%%%%%%%%%%%%%%%%%%%%%%%%%%%%%%
\section{Distancia fija y un par de materiales}
Se siguen los siguientes pasos:
\begin{enumerate}
	\item Seleccionar el material de la placa superior en \textbf{Superior plate} (figura \ref{unParMat}). 
	\item Seleccionar el material de la placa inferior en \textbf{Inferior plate}.
	\item Seleccionar la distancia (figura \ref{unaDistancia}). 
%%%%%
\begin{figure}[H]
\centering
\begin{subfigure}[b]{.48\textwidth}
	\centering
		\includegraphics{figuras/unaDistancia.PNG}
	\caption{ }
	\label{unaDistancia}
\end{subfigure} \hfill
\begin{subfigure}[b]{.48\textwidth}
	\centering
		\includegraphics{figuras/unParMat.PNG}
		\caption{ }
	\label{unParMat}
\end{subfigure}
\caption{Seleccionador de una distancia de separación en nanometros (\subref{unaDistancia}) y seleccionar de los materiales de emisor (Superior plate) y receptor (Inferior plate) (\subref{unParMat}) de la calculadora de campo cercano.}
\label{fig:sencillo}
\end{figure}	
	\item Hacer click sobre el botón \textbf{Calculate}.
	\item Esperar que el indicador de estado pase de \textit{Running...}, color rojo del indicador (figura \ref{fig:indicador_Running2}), a \textit{StdBy}, color verde del indicador (figura \ref{fig:indicador_StdBy2}), es decir, esperar que termine la simulación.
	\end{enumerate}
		%% figuras de estados
	\begin{figure}[H]
	\centering
	%% Figura 1
	\begin{subfigure}[b]{0.3\textwidth}
	\centering
	\includegraphics[width=\textwidth]{figuras/Procedimiento_Simulaciones/Radiacion/estado_changed}%
	\caption{Changed}%
	\label{fig:indicador_Changed2}%
	\end{subfigure}
	\hfill
	%% Figura 2
	\begin{subfigure}[b]{0.3\textwidth}
	\centering
	\includegraphics[width=\textwidth]{figuras/Procedimiento_Simulaciones/Radiacion/estado_running}%
	\caption{Running}%
	\label{fig:indicador_Running2}%
	\end{subfigure}
	\hfill
	%% Figura 3
	\begin{subfigure}[b]{0.3\textwidth}
	\centering
	\includegraphics[width=0.9\textwidth]{figuras/Procedimiento_Simulaciones/Radiacion/estado_stdby}%
	\caption{StdBy}%
	\label{fig:indicador_StdBy2}%
	\end{subfigure}
	\hfill
	\caption{Indicadores del estado actual del sistema. (\subref{fig:indicador_Changed2}) Indicador del estado \textbf{Changed} o estado de cambio, se activa cuando se produce algún cambio en los datos seleccionados para simular. (\subref{fig:indicador_Running2}) Indicador del estado \textbf{Running} o corriendo, se activa cuando estando en el estado \textbf{Changed} se hace clic al botón \textbf{Calculate} y corre la simulación. (\subref{fig:indicador_StdBy2}) Indicador del estado \textbf{StdBy}, se activa cuando termina la simulación, avisando que está a la espera de algún cambio.}
	\label{fig:indicadorLED2}
	\end{figure}
	%%%%%%%%%%%%%%%%%%%%%%%%%%%%%%%%%%%%%%%%%%%%%%%%%%%%%%%%%%%%%
%%%%%%%%%%%%%%%%%%%%%%%%%%%%%%%%%%%%%%%%%%%%%%%%%%%%%%%%%%%%%%%
\section{Rango de distancias y un par de materiales}
\begin{enumerate}
	\item Seleccionar el material de la placa superior en \textbf{Superior plate} (figura \ref{unParMat}).
	\item Seleccionar el material de la placa inferior en \textbf{Inferior plate}.
	\item Hacer click sobre el checkbox \textbf{Distance Range} (figura \ref{fig:check_distances2}).
	\item Hacer click sobre el botón \textbf{set} del \textbf{Distance Range}, que produce que aparezca la ventana de la figura \ref{fig:set_distances2}.
	\item Seleccionar el rango de distancias deseado o el checkbox \textbf{Full Range}, según lo que se desee. Para este trabajo se selecciona el checkbox \textbf{Full Range}.
	\item Hacer click en \textbf{Accept} de la nueva ventana.
	\item Hacer click sobre el botón \textbf{Calculate} y esperar a que termine de ejecutarse la simulación.
\end{enumerate}
	\begin{figure}[H]%
	\begin{subfigure}[b]{0.48\textwidth}
		\centering
			\includegraphics[width=0.6\textwidth]{figuras/Procedimiento_Simulaciones/Radiacion/check_distances2.png}
		\caption{Checkbox de distancias}
		\label{fig:check_distances2}
	\end{subfigure}
	\hfill
	\begin{subfigure}[b]{0.48\textwidth}
		\centering
			\includegraphics[width=0.6\textwidth]{figuras/Procedimiento_Simulaciones/Radiacion/set_distances_fullrange.png}
		\caption{Set de distancias}
		\label{fig:set_distances2}
	\end{subfigure}
	\caption{(\subref{fig:check_distances2}) Casilla para la selección de la opción de simular un rango de distancias. (\subref{fig:set_distances2}) Ventana para la selección del rango de distancias a simular.}%
	\label{fig:checkboxes2}%
	\end{figure}
	%%%%%%%%%%%%%%%%%%%%%%%%%%%%%%%%%%%%%%%%%%%%%%%%%%%
	%%%%%%%%%%%%%%%%%%%%%%%%%%%%%%%%%%%%%%%%%%%%%%%%%%%
	\section{Distancia fija y varios materiales}
	\begin{enumerate}
			\item Hacer click sobre el checkbox \textbf{Materials Range} (figura \ref{fig:check_materials2}).
			\item Hacer click sobre el botón \textbf{set} del \textbf{Materials Range}, que produce que aparezca la ventana de la figura \ref{fig:set_materials2}.
			\item Seleccionar los materiales para la cara superior (UpFace) (figura \ref{fig:set_materials2}).
			\item Seleccionar los materiales para la cara inferior (DownFace).
			\item Hacer clic en \textbf{Accept} de la ventana emergente.
			\item Seleccionar la distancia (figura \ref{unaDistancia}). 
			\item Hacer click sobre el botón \textbf{Calculate} y esperar a que termine de ejecutarse la simulación.
	\end{enumerate}
	%%%%%%%%%%%%%%%%%
	\begin{figure}[H]%
	\begin{subfigure}[b]{0.48\textwidth}
		\centering
			\includegraphics[width=0.6\textwidth]{figuras/Procedimiento_Simulaciones/Radiacion/set_materilas2.png}
		\caption{Set de materiales}
		\label{fig:set_materials2}
	\end{subfigure}\hfill
		\begin{subfigure}[b]{0.48\textwidth}
		\centering
			\includegraphics[width=0.6\textwidth]{figuras/Procedimiento_Simulaciones/Radiacion/check_materials2.png}
		\caption{Checkbox de materiales}
		\label{fig:check_materials2}
	\end{subfigure}
	\caption{(\subref{fig:set_materials2}) Ventana para la selección de las combinaciones de materiales a simula, siendo \textbf{UpFace} el emisor y \textbf{DownFace} la célula.	(\subref{fig:check_materials2}) Casilla para la selección de la opción de simular una combinación de materiales.}%
	\label{fig:sets2}%
	\end{figure}	
	%%%%%%%%%%%%%%%%%%%%%%%%%%%%%%%%%%%%%%%%%%%%%%%%%%%%
	%%%%%%%%%%%%%%%%%%%%%%%%%%%%%%%%%%%%%%%%%%%%%%%%%%%%
\section{Rango de materiales y varios materiales}
\begin{enumerate}
			\item Hacer click sobre el checkbox \textbf{Materials Range} (figura \ref{fig:check_materials2}).
			\item Hacer click sobre el checkbox \textbf{Distance Range} (figura \ref{fig:check_distances2}).
			%% MATERIALS
			\item Hacer click sobre el botón \textbf{set} del \textbf{Materials Range}, que produce que aparezca la ventana de la figura \ref{fig:set_materials2}.
			\item Seleccionar los materiales para la cara superior (UpFace) (figura \ref{fig:set_materials2}).
			\item Seleccionar los materiales para la cara inferior (DownFace).
			\item Hacer clic en \textbf{Accept} de la ventana emergente.
			%%%% DISTANCE
			\item Hacer click sobre el botón \textbf{set} del \textbf{Distance Range}, que produce que aparezca la ventana de la figura \ref{fig:set_distances2}.
			\item Seleccionar el rango de distancias deseado o el checkbox \textbf{Full Range}, según lo que se desee. Para este trabajo se selecciona el checkbox \textbf{Full Range}.
			\item Hacer click en \textbf{Accept} de la nueva ventana.
			\item Hacer click sobre el botón \textbf{Calculate} y esperar a que termine de ejecutarse la simulación.
\end{enumerate}

\chapter{Lista de siglas}

%\let\cleardoublepage\clearpage
\glsaddall
%\cleardoublepage
%\setglossarysection{chapter}
\setlength{\glsdescwidth}{\textwidth}
\printglossary[type=\acronymtype,title=Acr\'{o}nimos,style=longheader]%superheader]%myacronymstyle]
%\printnoidxglossary[type=\acronymtype,title=Acr\'{o}nimos]%,sort=use]%[type=\acronymtype]
\let\cleardoublepage\clearpage
\chapter{Tabla de Símbolos}
\setlength{\glsdescwidth}{15cm}
\printglossary[type=symbolslist,style=symbunitlong]
\let\cleardoublepage\clearpage
%\printglossary

\backmatter
%Glosario, lista de símbolos, notas, etc.

%estilo de bibliografía: plana, alfa...
\bibliographystyle{ieeetr}%plain}

%genera doble hoja en blanco
\cleardoublepage

%apartado de bibliografía
\addcontentsline{toc}{chapter}{Bibliografia}

%se incluye la bibliografía. Archivo de tipo .bib (bibtex)
\bibliography{bibliografia/bibliografia}


% Se incluye los anexos
%\include{capitulos/anexo}
%\printnoidxglossary[type=\acronymtype,title=Acronyms]
%fin del documento
\end{document}